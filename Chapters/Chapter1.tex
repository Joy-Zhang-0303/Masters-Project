% Chapter Template

\chapter{Basic Notions of Representation Theory} % Main chapter title

%\label{Chapter1} % Change X to a consecutive number; for referencing this chapter elsewhere, use \ref{ChapterX}



%----------------------------------------------------------------------------------------
%	SECTION 1
%----------------------------------------------------------------------------------------
%TODO: Add introduction/historical background 


%----------------------------------------------------------------------------------------
%	Group Actions
%
\section {Group Actions}
%----------------------------------------------------------------------------------------
\begin{defn}\label{def-grp-action}
A  \textbf{\textit{(left)} group action} of a group $G$ on a set $X$ is a map $\varphi \colon G \times X \to X$ (written as $g \cdot a$, for all $g \in G$ and $a \in A$) that satisfies the following two axoims:
\begin{align}
\label{grp-action-axiom-1}&1 \cdot  x = x && \forall x \in X\\
\label{grp-action-axiom-2}&(gh) \cdot x  = g \cdot (h \cdot x) && \forall g,h \in G, x \in X
\end{align}
\end{defn}
\begin{note}
We could likewise define the concept of a \textit{right} group action, where the set elements would be multiplied by group elements on the right instead of on the left.  Throughout we shall use the term \textit{group action} to mean a \textit{left} group action.
\end{note}


\begin{prop}\label{sigma-is-a-permutation}
Let $G$ act on the set $X$.  For any fixed $g \in G$, the map $\sigma_g$ from $X$ into $X$ defined by $\sigma_g (x) = g \cdot x$ is a \textit{permutation} of the set $X$, i.e. $\sigma_g \in S_X$.
\end{prop}
\begin{proof}
We show that $\sigma_g$ is a permutation of $X$ by finding a two-sided inverse map, namely $\sigma_{g^{-1}}$. Observe that for any $x \in X$, we have
\begin{align*}
(\sigma_{g^{-1}} \circ \sigma_g) (x) &= \sigma_{g^{-1}} ( \sigma_g (x) && \text{(by definition of function composition)} \\
					&= g^{-1} \cdot (g \cdot x) && \text{(by definition of $\sigma_g$ and $\sigma_{g^{-1}}$)} \\
					&= (g^{-1}g) \cdot x &&\text{(by axiom \ref{grp-action-axiom-1} of an action)} \\
					&= 1 \cdot x \\
					&= x &&\text{(by axiom \ref{grp-action-axiom-2} of an action)}.
\end{align*}
Thus $\sigma_{g^{-1}} \circ \sigma_g$ is the identity map on $X$. We can reverse the roles of $g$ and $g^{-1}$ to see that $\sigma_g \circ \sigma_{g^{-1}}$ is also the identity map on $X$.  Having a two-sided inverse, we conslude that $\sigma_g$ is a permutation of $X$.
\end{proof}

\begin{prop}\label{action-yields-hom}
Let $G$ act on the set $X$. The map from $G$ to the symmetric group $S_X$ defined by $g \mapsto \sigma_g (x) = g \cdot x$ is a group homomorphism.
\end{prop}
\begin{proof}
Define the map $\varphi \colon G \to S_X$ by $\varphi (g) = \sigma_g$.  We have seen from Proposition \ref{sigma-is-a-permutation} that $\sigma_g$ is indeed an element of $S_X$.  It remains to show that $\varphi(g_1 g_2) = \varphi(g_1) \circ \varphi(g_2)$ for any $g_1, g_2 \in G$.  Observe that

\begin{align*}
\varphi(g_1 g_2)(x) &= \sigma_{g_1 g_2} (x) && \text{(by definition of $\varphi$)} \\
			&= (g_1 g_2) \cdot x && \text{(by definition of $\sigma_{g_1 g_2}$)} \\
			&= g_1 \cdot (g_2 \cdot x) && \text{(by axiom \ref{grp-action-axiom-1} of an action)} \\
			&= \sigma_{g_1} ( \sigma_{g_2} (x)) && \text{(by definition of $\sigma_{g_1}$ and  $\sigma{g_2}$)} \\
			&= \varphi(g_1) ( \varphi(g_2) (x)) && \text{(by definition of $\varphi$)}\\
			&= (\varphi(g_1) \circ \varphi(g_2)) (x) && \text{(by definition of function composition)}.
\end{align*}
Since the values of $\varphi(g_1 g_2)$ and $\varphi(g_1) \circ \varphi(g_2)$ agree on every element $x \in X$, these two permutations are equal. We conclude that $\varphi$ is a homomorphism, since $g_1$ and $g_2$ were arbitrary elements of $G$.
\end{proof}


\begin{prop} \label{hom-yields-action}
Any homomorphism $\psi$ from the group $G$ into the symmetric group on $S_X$ on a set $X$ gives rise to an action of $G$ on $X$, defined by taking $g \cdot x = \psi(g)(x)$.
\end{prop}
\begin{proof}
Suppose  that we have a homomorphism $\psi$ from $G$ into $S_X$.  We can define a map from $G \times X$ to $X$  by $g \cdot x = \psi(g)(x)$. We verify that this map satisfies the definition of a group action of $G$ on $X$:
\\ (axiom \ref{grp-action-axiom-1}) \quad $1 \cdot x = \psi(1)(x) = id_X(x) = x$
\\(axiom \ref{grp-action-axiom-2}) \quad $(gh) \cdot x = \psi(gh)(x) = (\psi(g)\psi(h))(x) = \psi(g)(\psi(h)(x)) = g \cdot (h \cdot x)$
\end{proof}

\begin{prop} \label{equivalence-of-actions}
The actions of $G$ on the set $X$ are in bijective correspondence with the homomorphisms from $G$ into the symmetric group $S_X$.
\end{prop}
\begin{proof}
By Proposition \ref{action-yields-hom}, any action of $G$ on $X$ yields a homomorphism from $G$ into $S_X$.    Conversely, any homomorphism from $G$ into $S_X$ establishes an action of $G$ on $X$ by Proposition \ref{hom-yields-action}.
\end{proof}

%Todo\colon add examples?

%----------------------------------------------------------------------------------------
%	The Definition of a Representation
%
\section{The Definition of a Representation}
%----------------------------------------------------------------------------------------

\begin{defn}
\label{rep-def-1}
Let $G$ be a group, let $F$ be a field, and let $V$ be a vector space over $F$.  A \textbf{linear representation} of G is any group homomorphism $\varphi\colon G \to GL(V)$. \end{defn}
 
 %More explicitly, a representation is a map $\rho \colon G \rightarrow GL(V)$ such that \[ \rho (g_1 g_2) = \rho(g_1) \rho(g_2) \quad \forall g_1, g_2 \in G. \]

\begin{defn}\label{rep-def-2}Let $G$ be a group, let $F$ be a field, and let $V$ be a vector space over $F$. A \textbf{linear representation} of $G$ is any action of $G$ on $V$ which preserves the linear structure of $V$, that is, 
\begin{align}
\label{rep-axiom-1}&g \cdot (v_1+v_2)=g \cdot v_1+g \cdot v_2 \quad && \forall g \in G, v_1, v_2 \in V \\
\label{rep-axiom-2}&g \cdot (kv) = k (g \cdot v) \quad && \forall g \in G, v \in V, k \in F
\end{align}
 \end{defn}
 
 \begin{note}
 Unless otherwise specificed, we use \textit{representation} to mean \textit{finite-dimensional complex representation}.
 \end{note}
 
 
 \begin{prop}
The definitions of a linear representation given in \ref{rep-def-1} and \ref{rep-def-2} above are equivalent.
 \end{prop}
 \begin{proof}
%\leavevmode
 \begin{itemize}
\item[$(\rightarrow)$]  Suppose that we have a homomorphism $\varphi \colon G \to GL(V)$.  Note that $GL(V)$ is a subgroup of the symmetric group $S_V$ on $V$, so we can apply Proposition \ref{hom-yields-action} to obtain an action of $G$ on $V$ by $g \cdot v = \varphi(g)(v)$.  We check that this action preserves the linear structure of V.
\\\ref{rep-axiom-1} \quad For any $g \in G$, $v_1, v_2 \in V$ we have $g \cdot (v_1 +  v_2) = \varphi(g) (v_1 + v_2) = \varphi(g)(v_1) + \varphi(g)(v_2)= g \cdot v_1 + g \cdot v_2$.
\\\ref{rep-axiom-2} \quad For any $g \in G, v \in V, k \in F$ we have $g \cdot (kv) = \varphi(g)(kv) = k (\varphi(g)(v)) = k (g \cdot v)$.
\item[$(\leftarrow)$] Suppose that we have an action of $G$ on $V$ which preserves the linear structure of V in the sense of Definition \ref{rep-def-2}.  We can apply Proposition \ref{action-yields-hom} to obtain a homorphism $\varphi \colon G \to S_V$ given by $\varphi(g) = \sigma_g$ where $\sigma_g(v) = g \cdot v $.  It remains to show that the image $\varphi(G)$ of $G$ under $\varphi$ is actually contained in $GL(V)$, i.e. that for each $g \in G$ the map $\sigma_g$ is linear.  Fix an element $g \in G$. For any $k \in F$ and $v \in V$ we have
\begin{align*}
\sigma_g (kv) &= g \cdot (kv) && \text{(by definition of $\sigma_g$)} \\
		&= k (g \cdot v) && \text{(by property \ref{rep-axiom-1})} \\
		&= k (\sigma_g (v)) && \text{(by definition of $\sigma_g$)}.
\end{align*}
Also, for any $v_1, v_2 \in V$ we have
\begin{align*}
\sigma_g (v_1 + v_2) &= g \cdot (v_1 + v_2) && \text{(by definition of $\sigma_g$)} \\
		&= g \cdot v_1 + g \cdot v_2 && \text{(by property \ref{rep-axiom-2})} \\
		&= \sigma_g(v_1) + \sigma_g(v_2) && \text{(by definition of $\sigma_g$)}.
\end{align*}
Thus $\sigma_g$ is linear, and $\varphi(G) \subset GL(V)$ proves that we  have a homomorphism $\varphi \colon G \to GL(V)$.

\end{itemize}
 \end{proof}
 \begin{defn}Let $G$ be a group, let $F$ be a field, let $V$ be a vector space over $F$, and let $\varphi \colon G \to GL(V)$ be a representation of $G$.  The \textbf{dimension} of the representation is the dimension of $V$ over $F$.  
 \end{defn}
% ============== Examples of Representations ===========%
 \begin{example}
 \begin{enumerate}
\item Let $V$ be a $1$-dimensional vector space over the field $F$.  The map $\varphi \colon G \to GL(V)$ defined by $\varphi(g) = 1$ for all $g \in G$ is a representation called the \textit{trival representation} of $G$.  The trivial representation has dimension $1$.

\item If a finite group $G$ acts on a finite set $X$ and $F$ is any field, then there is an associated \textit{permutation representation}  on the vector space $V$ over $F$ with basis $\{e_x \colon x \in X\}$.  We let $G$ act on the basis elements by $g \cdot e_x = e_{gx}$ for all $x \in X$ and $g \in G$. Note that $G$ permutes the basis elements of $V$. 

\item A fundamental special case of a permutation representation is given by a finite group acting on itself by left multiplication.  In this case, the elements of $G$ form a basis for $V$, and each $g \in G$ permutes the basis elements by $g \cdot g_i = gg_i$.  This is called the \textit{regular representation} of $G$ and has dimension $|G|$. We shall see later that this representation encodes information about all other representations of $G$.

\item For any symmetric group $S_n$ the \textit{alternating representation} on $V=\mathbb{C}$ is given by the map $\varphi \colon S_n \to GL(\mathbb{C})=\mathbb{C}^\times$ defined by $\varphi(\sigma)=\text{sgn}(\sigma)$. More generally, for any group $G$ with a subgroup $H$ of index $2$, we can define an \textit{alternating representation} $\varphi \colon G \to GL(\mathbb{C})$ by letting $\varphi(g) = 1$ if $g \in H$ and $\varphi(g) = -1$ if $g \notin H$.  (We recover our original example  by taking $G= S_n$ and $H=A_n$.)

\end{enumerate}
 \end{example}

%=========Homomorphisms===================%

\begin{defn}
A \textbf{homomorphism} between two representations $\varphi_1 \colon G \to GL(V)$ and $\varphi_2 \colon G \to GL(W)$ is a linear map $\psi \colon V \to W$ that interwines with (respects) the $G$-action, i.e. such that \[ \psi ( \varphi_1 (g)(v)) = \varphi_2(g) (\psi(v)) \quad \forall v \in V, g \in G \]  An \textbf{isomorphism} of representations is a homomorphism of representations that is also an invertible map.
\end{defn}
\begin{note}
If we have representations $(\varphi_1, V)$ and $(\varphi_2, W)$ and an isomorphism of vector spaces $\psi \colon V \to W$ then we can rewrite the compatibility requirement above as $\varphi_2(g) = \psi \circ \varphi_1(g) \circ \psi^{-1}$ for all $g \in G$.
\end{note}

Given any representation $(\varphi, V)$ of $G$ on a vector space $V$ over a field $F$ of dimension $n$, we can fix a basis for $V$ to obtain an isomorphism of vector spaces $\psi \colon V \to F^n$.  We obtain a representation $\phi$ of $G$ on $F^n$ by defining $\phi = \psi \circ \varphi(g) \circ \psi^{-1}$ for all $g \in G$. Clearly, this representation is isomorphic to the original representation $(\varphi, V)$. In particular we can always choose to view $n$-dimensional complex representations as representations on $\mathbb{C}^n$ where each $\phi(g)$ is given by an $n \times n$ matrix with entries in $\mathbb{C}$.

Suppose that we have two representations $\varphi \colon G \to GL_n(F)$ and $\phi \colon G \to GL_m(F)$ given by $\varphi(g) = X_g$ and $\phi(g) = Y_g$.  A homomorphism between these representations is then an $m \times n$ matrix $A$ such that $A X_g = Y_g A$ for all $g \in G$.   An isomorphism is given precisely when such $A$ is square and invertible.  Thus, two representations $\varphi \colon G \to GL_n(F)$ and $\phi \colon G \to GL_n(F)$ are isomorphic if and only if there exists $A \in GL_n(F)$ such that $\varphi(g) = A \phi(g) A^{-1}$ for all $g \in G$.  This establishes the following proposition:
\begin{prop}
The isomorphism classes of $n$-dimensional representations of $G$ on $\mathbb{C}$ are in bijection with the quotient $Hom(G; GL_n(\mathbb{C})) / GL_n(\mathbb{C})$ of group homomorphisms $G \to GL_n(\mathbb{C})$ modulo the conjugation action of $GL_n(\mathbb{C})$.
\end{prop}



%----------------------------------------------------------------------------------------
%	Representations of Cyclic Groups
%
\section{Representations of Cyclic Groups}
%----------------------------------------------------------------------------------------
\begin{example}[Representations of $\mathbb{Z}$]
We want to classify all representations of the group $\mathbb{Z}$ under addition.  Consider an $n$-dimensional representation $\varphi \colon \mathbb{Z} \to GL_n$.  For $\varphi$ to be a group homomorphism requires that $\varphi(0) = \text{Id}$.  Observe that for any $0 \neq n \in \mathbb{Z}$, we have $\varphi(n) = \varphi( 1 + \ldots + 1) = \varphi(1)^n$.  Thus $\varphi$ is completely determined by the matrix $\varphi(1) \in GL_n(\mathbb{C})$, and any such matrix determines a representation of $\mathbb{Z}$.  It follows that the $n$-dimensional isomorphism classes of representations of $\mathbb{Z}$ are in bijection with the conjugacy classes in $GL_n(\mathbb{C})$.  These conjugacy classes can be parameterized by the \textit{Jordan canonical form}.
\end{example}

\begin{example}[Representations of the cyclic group of order $n$]
We shall classify all representations of the cyclic group $G = {1= g^n, g, \ldots, g^{n-1}}$ of order $n$. Consider a representation $\varphi \colon G \to GL(V)$.  As in the previous example, we know that $\varphi(1) = \text{Id}$ and $\varphi(g^k) = \varphi(g)^k$.  Thus our representation $\varphi$ is determined completely by the linear transformation $\varphi(g)$.   It will be helpful to fix a basis of $V$ so that we may view $\varphi(g)$ as a matrix $A$.  Recall from linear algebra that there exists a basis in which $\varphi(g)$ takes the \textit{Jordan normal form}.

\[ A = \begin{bmatrix}
    J_1 & 0 & \dots  &0 \\
  0 & J_2  & \dots & 0 \\
    \vdots & \vdots  & \ddots & \vdots \\
    0& 0&  \dots  & J_m
\end{bmatrix} \]
where each \textit{Jordan block} $J_k$ takes the form 
\[J_k =  \begin{bmatrix}
    \lambda & 1&0& \dots  &0 & 0 \\
     0 &\lambda& 1& \ddots & 0  & 0 \\
     0 & 0 & \lambda & \ddots& 0  & 0 \\
     0 & 0 & 0 & \ddots & 1 & 0 \\
    \vdots & \ddots & \ddots & \ddots & \ddots  & 1\\
    0& 0& 0 & \dots  & 0  &\lambda
\end{bmatrix}. \]
Now $I = A^n$ is a block-diagonal matrix with diagonal blocks $J_k^n$, so we must have that each block $J_k^n=\text{Id}$.  Observe that we can write each block as $J_k = \lambda \text{Id} + N$ where $N$ is the Jordan block with $\lambda = 0$.  Thus we have 
\[ \text{Id} = J_k^n = (\lambda \text{Id} + N)^n = \lambda ^n \text{Id} + \binom{n}{1} \lambda ^{n -1} N + \binom{n}{2} \lambda ^{n-2} N^2 + \ldots + \binom {n} {n -1} \lambda N^{n -1} + N^n \].  
\begin{lemma}
Let $N$ be the Jordan block with $\lambda = 0$ of size $n \times n$.  For any integer $p$ with $1 \leq p \leq n - 1$, then $N^p$ is the matrix with ones in the positions $(i,j)$ where $j = i + p$ and zeroes everywhere else.  (The ones lie along a line parallel to the diagonal, $p$ steps above it.)

\begin{proof}
(By induction.)
\begin{itemize}
\item \textit{Base case} This is simply the definition of $N$.
\item \textit{Inductive step} Suppose that the lemma holds for $N^p$.  We compute the $(i,j)$ entry of $N^{p+1}$:
\[ (N^{p+1})_{i,j} = \sum_{k=1}^{n} (N^{p})_{i,k} N_{k, j} = (N^p)_{i, i +p} N_{i +p, j} = N_{i +p, j} =\begin{cases} 
      1 & \text{if}  j = i + (p +1) \\
      0 & \text{otherwise} \\
   \end{cases}  \]
\end{itemize}
\end{proof}
\end{lemma}

Now, if $N \neq 0$ then each term $\binom {n}{k} \lambda ^ {n - k } N ^k$ for $k > 0$ would yield some non-zero non-diagonal entries (in the positions $(i,j)$ where $j= i + k$) and hence our sum could not equal the identity matrix.  We must conclude that $N = 0$, and $J_k = \lambda ^n$ is a $1 \times 1$ block.  Thus $\varphi(g)$ is a diagonal matrix with $n$th roots of unity as diagonal entries. 

To summarize, every $m$-dimensional representation $\varphi$ of the cyclic group $G = \langle g \rangle$ of order $n$ can be seen to act (in the right choice of basis) as $m \times m$ diagonal matrices with $n$th roots of unity along the diagonal.  In particular, these representations are determined completely by the value of $\varphi(g)$ and are classified up to isomorphism by unordered $m$-tuples of $n$th roots of unity.
\end{example}

\section{Constructions from Linear Algebra}
\begin{defn} A \textbf{subrepresentation} of $V$ is a $G$-invariant subspace $W \subseteq V$; that is a subspace $W \subseteq V$ with the property that $\varphi(g) (w) \in W$ for all $g \in G, w \in W$.  Note that $W$ effects a representation of $G$ under the action $\varphi(g) \restriction_W$.
\end{defn}

From elementary linear algebra, we know that given a subspace $W \subseteq V$, we can form the \textbf{quotient space} $V / W$ consisting of cosets $v + W$ in $V$.  If $W$ is a subrepresentation of $V$, we would like to define an action of $G$ on $V / W$ by the natural choice of $g (v + W) = \varphi(g)(v)+ W$.  We must that this action is well defined.  If we choose another $v' \in v + W$, then $v - v' \in W$ so that $\varphi(g)(v - v') \in W$ since $W$ is $G$-invariant.  Thus, the cosets $\varphi(g)(v) + W$ and $\varphi(g)(v') + W$ agree and this action is indeed well defined.

\begin{defn}
Let $W$ be a $G$-subrepresentation of $V$.  Then $V/W$ forms a representation of $G$ called the \textbf{quotient representation} of $V$ under $W$, with the action $g( v + W) = \varphi(g)(v) + W$.
\end{defn}

We recall also from linear algebra that given two vector spaces $V_1$ and $V_2$, we can form the \textbf{direct sum} $V_1 \oplus V_2$ consisting of ordered pairs $(v_1 ,v_2)$ where $v_1 \in V_1, v_2 \in V_2$.  

\begin{defn}
Let $V_1$ and $V_2$ be representations of $G$.  Then $V_1 \oplus V_2$ forms a representation of $G$ called the \textbf{direct sum representation}, with the action $g (v_1, v_2) = (g \cdot  v_1, g \cdot v_2)$.
\end{defn}


\section{Complete Reducibility and Unitarity}
\begin{defn}
A representation is called \textbf{irreducible} if it contains no proper invariant subspaces.  It is called \textbf{completely reducible} if it decomposes into a direct sum of irreducible subrepresentations.
\end{defn}

\begin{example}
\begin{enumerate}
\item Any irreducible representation is completely reducible.
\item Any $1$-dimensional representations has no proper subspaces, and is thus irreducible.
\end{enumerate}
\end{example}

\begin{thm}\label{simultaneous} If $A_1, A_2, \ldots, A_r$ are linear operators on $V$ and each $A_i$ is diagonalizable, they are simultaneously diagonalizable if and only if they commute.
\end{thm}
\begin{proof}
 See \cite[Theorem 5.1] {ConradMinPoly}.
\end{proof}

\begin{thm} Every complex representation of a finite abelian group is completely reducible, and every irreducible representation is $1$-dimensional.  
\end {thm}
\begin{proof}
Take an arbitrary element $g \in G$.  Since $G$ is finite, we can find an integer $n$ such that $g^n = 1$ and $\varphi(g)^n = Id$.    Hence the minimal polynomial of $\varphi(g)$ divides  $x^n -1$.  Recall that $x^n-1$ has $n$ distinct roots over $\mathbb{C}$, which are generated by taking powers of $\xi = e^{\frac{2 \pi i}{n}}$.  This means that the minimal polynomial $\varphi(g)$ factors into linear factors only over $\mathbb{C}$ so that $\varphi(g)$ is diagonalizable.  We conclude that each $\varphi(g)$ is (separately) diagonalizable since $g \in G$ was arbitrary.

Now, given any two elements $g_1, g_2 \in G$ we have 

\begin{align*}
\varphi(g_1) \varphi(g_2)&= \varphi(g_1 g_2)&& \text{(since $\varphi$ is a homomorphism)} \\
		&=  \varphi(g_2 g_1) && \text{(since $G$ is abeilian)} \\
		&= \varphi(g_2) \varphi(g_1) && \text{(since $\varphi$ is a homomorphism)}.
\end{align*}
Thus the matrices $\left\{ \varphi(g)\right\}$ commute, so we can apply \ref{simultaneous} to conclude that $\left\{ \varphi(g)\right\}$ are simultaneously diagonalizable. This basis $\left\{ e_1, ..., e_k \right\}$ yields the decomposition $V= \mathbb{C}e_1 \oplus \mathbb{C} e_2 \oplus \ldots \oplus \mathbb{C} e_n$.
\end{proof}

We recall the following definition from linear algebra:
\begin{defn} \label{hip}
Let $V$ be a complex vector space.  A \textbf{Hermitian inner product} on $V$ is a map $\langle \cdot{,}\cdot \rangle \colon  V \times V \to \mathbb{C}$ that satisfies the following properties for all  $u, v, w \in V$ and $c \in \mathbb{C}$:
\begin{enumerate}
\item \label{hip-1}  $\langle u + v, w \rangle = \langle u , w \rangle + \langle v, w \rangle$.
\item \label{hip-2} $\langle c u, v \rangle = c \langle u, v \rangle$.
\item \label{hip-3} $\langle u , v \rangle = \overline{\langle v , u \rangle }$.
\item \label{hip-4} $\langle v, v \rangle \geq 0$ with equality if and only if $v = 0$.
\end{enumerate}
\end{defn}

\begin{defn}
A representation $\varphi$ of $G$ on a complex vector space $V$ is \textbf{unitary} if $V$ has been equipped with a hermetian inner product $\langle \cdot{,}\cdot \rangle$ which is preserved by the action of $G$, that is, 
\[ \langle v , w \rangle = \langle \varphi(g)(v) , \varphi(g)(w) \rangle \quad \forall v,w \in V, g \in G. \]
A representation is said to be \textbf{unitarisable} if it can be equipped with such a product (even without one being specified).
\end{defn}

\begin{thm}\label{weyl-unitary-trick} [Weyl's unitary trick] Finite-dimensional represenations of finite groups are unitarisable.
\end {thm}
\begin{proof}
Take any Hermetian inner product on V, say $\langle \cdot{,}\cdot \rangle'$.  We define a new inner product on V by \textit{averaging over} $G$:
\[ \langle v {,}w  \rangle \defeq \frac{1}{|G|} \mathlarger{\sum}_{g \in G} \langle \varphi(g)v {,} \varphi(g)w \rangle'. \]

This new inner product satifies properties \ref{hip-1},  \ref{hip-2}, and  \ref{hip-3} of Definition \ref{hip} by linearity.  It remains to check positivity (\ref{hip-4}).  Clearly $\langle v {,} v \rangle = 0$ when $v =0$, since each term of the sum will equal zero.  In the case where $v \neq 0$,  observe that

\[ \langle v {,} v \rangle =\frac{1}{|G|} \mathlarger{\sum}_{g \in G} \langle \varphi(g)v {,} \varphi(g)v \rangle' \geq 0\]

since each term of the sum is non-negative by the positivity of $\langle \cdot{,}\cdot \rangle'$.  The only problem would occur if each term of this sum is equal to zero.  But $\langle \varphi(e) v {,} \varphi(e) v \rangle ' = \langle v {,} v\rangle' > 0$.  Thus $\langle v {,} v \rangle > 0$.  

Finally, we show that our new inner product is $G$-invariant.  For any $h \in G$, we have
\begin{align*}
\langle \varphi(h) v {,} \varphi(h) w  \rangle &= \frac{1}{|G|} \mathlarger{\sum}_{g \in G} \langle \varphi(g) \varphi(h)v {,}\varphi(g) \varphi(h)w \rangle' \\
&= \frac{1}{|G|} \mathlarger{\sum}_{g \in G} \langle \varphi(gh)v {,} \varphi(gh)w \rangle' && \text{(since $\varphi$ is a homomorphism)} \\
&= \frac{1}{|G|} \mathlarger{\sum}_{k \in G} \langle \varphi(k)v {,} \varphi(k)w \rangle' && \text{(by a change of variables)}\\
&= \langle v {,} w \rangle.
\end{align*}
\end{proof}

\begin{lemma}\label{unitary-ortho-complement}
Let $V$ be a unitary representation of $G$ and let $W \subseteq V$ be a $G$-invariant subspace.  Then the orthogonal complement $W^\perp$ is also $G$-invariant.
\end{lemma}
\begin{proof}
Choose arbitrary elements $v \in W^\perp$ and $g \in G$.  We need to show that $gv \in W^\perp$.  Now for any $w \in W$, we have $\langle v,w \rangle = 0$.  Thus $\langle gv, gw \rangle = g \overline{g} \langle v , w \rangle = 0$ for any $w \in W$.  Notice that $w' = gw \in W$ since $W$ is $G$-invariant.  This implies that $ \langle gv, w' \rangle =0$, i.e. $gv \in W^\perp$.
\end{proof}


\begin{thm}\label{full-red-of-unitary-reps}
A finite-dimensional unitary representation of a group is fully reducible into unitary irreducible subrepresentations.
\end{thm}
\begin{proof}
Let $V$ be a finite dimensional unitary representation of $G$. We proceed by induction on the dimension of $V$.  If $\text{dim}(V)=1$, then $V$ is necessarily irreducible.  Now, suppose the theorem holds for all $W$ with $\text{dim}(V) \leq n - 1$ and suppose $\text{dim}(V)=n$. If $V$ is irreducible, we are done. Otherwise, there exists a proper $G$-invariant subspace $W (\neq 0, V)$.  
We can write $V = W \oplus W^\perp$ by Lemma \ref{unitary-ortho-complement}.  Applying the inductive hypothesis to $W$ and $W^\perp$, we obtain a decomposition into irreducibles
\[ V = (W_1 \oplus \ldots \oplus W_j) \oplus (W_{j + 1} \oplus \ldots \oplus W_k). \]
\end{proof}

\begin{cor}
Every complex representation of a finite group is completely reducible.
\end{cor}
\begin{proof}
Any such representation is unitarisable y by Theorem \ref{weyl-unitary-trick}.  We can then apply Theorem \ref{full-red-of-unitary-reps} to obtain full reduciblility.
\end{proof}


\section{Schur's Lemma}
\begin{thm}\label{schur-lemma-over-c}[Schur's Lemma over $\mathbb{C}$.] If $V$ is an irreducible $G$-representation over $\mathbb{C}$, then evey linear operator $\phi \colon V \to V$ commuting with $G$ is a scalar.
\end{thm}
\begin{proof}
Let $\lambda$ be an eigenvalue of $\phi$.  Observe that the eigenspace $E_\lambda$ is $G$-invariant: If $v \in E_\lambda$, then $\phi(v) = \lambda v$.  This implies that $\phi(g v) = g \phi(v) = g (\lambda v) = \lambda (gv)$, i.e. $gv \in E_\lambda$. Since $g$ was arbitrary, $E_\lambda$ is indeed $G$-invariant.  Now $E_\lambda \neq 0$, so by irreducibility $E_\lambda = V$.  Thus $\phi = \lambda \text{Id}$.  
\end{proof}

\begin{defn}
Given two representations $V$ and $W$, we write $\textbf{Hom}^G(V,W)$ to  denote the vector space of \textit{intertwining operators} from $V$ to $W$, i.e. the linear operators from $V$ to $W$ which commute with the action of $G$.  In other words, this is the vector space consisting of all \textit{homomorphisms of representations} between $V$ and $W$.
\end{defn}

\begin{cor}
If $V$ and $W$ are irreducible, the space $\text{Hom}^G(V,W)$ is $1$-dimensional if the representations are isomorphic, and in this case any non-zero map is an isomorphism. Otherwise,  $\text{Hom}^G(V,W=\{0\}$.
\end{cor}
\begin{proof}
Let $0 \neq \phi \in \text{Hom}^G(V,W)$.  If $v \in \text{ker}(\phi)$, then $\phi(v) = 0$ implies that  $\phi (gv) = g  \phi(v) = g  0 =0$, i.e. $ gv \in \text{ker}(\phi)$. Similarly, if $v \in \text{im}(\phi)$, then $v = \phi(w)$ implies that $\phi (gw) = g \phi(w) = gv$, i.e. $gv \in \text{im}(\phi)$.  Thus $\text{ker}(\phi)$ and $\text{im}(\phi)$ are both $G$-invariant.  

Irreducibility yields $\text{ker}(\phi) = 0 $ or $V$ and $\text{im}(\phi) = 0$ or  $W$ as the only possibilities.  If $\phi \neq 0$, then $\text{ker}(\phi) = 0$.  This means that $\phi$ is injective,  $\text{im}(\phi) = W$, and $\phi$ is an isomorphism. 

Let $\psi$ be another interwining operator from $V$ to $W$.  Then $\phi ^{-1} \circ \psi$ is also an intertwining operator from $V$ to $V$.  We can apply Schur's Lemma over $\mathbb{C}$ to see that $\phi ^{-1} \circ \psi = \lambda \text{Id}$, hence $\psi = \lambda \phi$.
\end{proof}

More definitions are required before we can state a more general Schur's Lemma (not restricted to just $\mathbb{C}$).

\begin{defn}
An \textbf{algebra} over a field $K$ is a ring with unit, containing a distinguished copy of $K$ that commutes with every algebra element, and with $1 \in K$ begin the algebra unit.  A \textbf{division ring} is a ring where every non-zero element is invertible, and a \textbf{division algebra} is a division ring which is also a $K$-algebra.
\end{defn}

\begin{defn}
Let $V$ be a representation of $G$ over $K$.  The \textbf{endomorphism algebra} $\text{End}^G(V)$ is the space of linear self-maps $\phi \colon V \to V$ which commute with the group action, that is, $\varphi(g) \circ \phi = \phi \circ \varphi(g) \quad \forall g \in G$.  The addition is the usual addition of linear maps (pointwise), and the multiplication is function composition.  The distinguished copy of $K$ is given by $K \text{Id}$.
\end{defn}

\begin{thm}\label{schur-lemma}[Schur's Lemma] If $V$ is an irreducible finite-dimensional representation of $G$ over $K$, then $\text{End}^G(V)$ is a finite-dimensional division algebra over $K$.
\end{thm}