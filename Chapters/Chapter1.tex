% Chapter Template

\chapter{Basic Notions of Representation Theory} % Main chapter title

%\label{Chapter1} % Change X to a consecutive number; for referencing this chapter elsewhere, use \ref{ChapterX}



%----------------------------------------------------------------------------------------
%	SECTION 1
%----------------------------------------------------------------------------------------
%TODO: Add introduction/historical background 
\section{Introduction}
Groups arise naturally as sets of symmetries of some object which are closed under composition and taking inverses.  For example, 
\begin{enumerate}
\item The \textbf{symmetric group} of degree $n$, $S_n$, is the group of all symmetries of the set $\{ 1, \ldots, n \}$.
\item The \textbf{dihedral group} of order $2n$, $D_{n}$, is the group of all symmetries of the regular $n$-gon in the plane.
\end{enumerate}
In these two examples, $S_n$  acts on the set $\{ 1, \ldots, n \}$ and $D_{n}$ acts on the regular $n$-gon in a natural manner. One may wonder more generally:  Given an abstract group $G$, which objects $X$ does $G$ act on?
This is the basic question of representation theory, which attempts to classify all such $X$ up to isomorphism.

%----------------------------------------------------------------------------------------
%	Group Actions
%
\section {Group Actions}
%----------------------------------------------------------------------------------------
\begin{defn}\label{def-grp-action}
A  \textbf{\textit{(left)} group action} of a group $G$ on a set $X$ is a map $\rho \colon G \times X \to X$ that satisfies the following two axoims:
\begin{align}
\label{grp-action-axiom-1}&1 \cdot  x = x && \forall x \in X\\
\label{grp-action-axiom-2}&(gh) \cdot x  = g \cdot (h \cdot x) && \forall g,h \in G, x \in X
\end{align}
\end{defn}
\begin{note}
We could likewise define the concept of a \textit{right} group action, where the set elements would be multiplied by group elements on the right instead of on the left.  Throughout we shall use the term \textit{group action} to mean a \textit{left} group action.  We often write $g \cdot a$ in place of $\rho(g) (a)$. 
\end{note}


\begin{prop}\label{sigma-is-a-permutation}
Let $G$ act on the set $X$.  For any fixed $g \in G$, the map $\sigma_g$ from $X$ into $X$ defined by $\sigma_g (x) = g \cdot x$ is a \textit{permutation} of the set $X$.  That is, $\sigma_g \in S_X$.
\end{prop}
\begin{proof}
We show that $\sigma_g$ is a permutation of $X$ by finding a two-sided inverse map, namely $\sigma_{g^{-1}}$. Observe that for any $x \in X$, we have
\begin{align*}
(\sigma_{g^{-1}} \circ \sigma_g) (x) &= \sigma_{g^{-1}} ( \sigma_g (x)  \\
					&= g^{-1} \cdot (g \cdot x) && \text{(by definition of $\sigma_g$ and $\sigma_{g^{-1}}$)} \\
					&= (g^{-1}g) \cdot x &&\text{(by axiom \ref{grp-action-axiom-1} of an action)} \\
					&= 1 \cdot x \\
					&= x &&\text{(by axiom \ref{grp-action-axiom-2} of an action)}.
\end{align*}
Thus $\sigma_{g^{-1}} \circ \sigma_g$ is the identity map on $X$. We can reverse the roles of $g$ and $g^{-1}$ to see that $\sigma_g \circ \sigma_{g^{-1}}$ is also the identity map on $X$.  Having a two-sided inverse, we conclude that $\sigma_g$ is a permutation of $X$.
\end{proof}

\begin{prop}\label{action-yields-hom}
Let $G$ act on the set $X$. The map from $G$ into the symmetric group $S_X$ defined by $g \mapsto \sigma_g (x) = g \cdot x$ is a group homomorphism.
\end{prop}
\begin{proof}
Define the map $\rho \colon G \to S_X$ by $\rho (g) = \sigma_g$.  We have seen from Proposition \ref{sigma-is-a-permutation} that $\sigma_g$ is indeed an element of $S_X$.  It remains to show that $\rho(g_1 g_2) = \rho(g_1) \circ \rho(g_2)$ for any $g_1, g_2 \in G$.  Observe that

\begin{align*}
\rho(g_1 g_2)(x) &= \sigma_{g_1 g_2} (x) && \text{(by definition of $\rho$)} \\
			&= (g_1 g_2) \cdot x && \text{(by definition of $\sigma_{g_1 g_2}$)} \\
			&= g_1 \cdot (g_2 \cdot x) && \text{(by axiom \ref{grp-action-axiom-1} of an action)} \\
			&= \sigma_{g_1} ( \sigma_{g_2} (x)) && \text{(by definition of $\sigma_{g_1}$ and  $\sigma{g_2}$)} \\
			&= \rho(g_1) ( \rho(g_2) (x)) && \text{(by definition of $\rho$)}\\
			&= (\rho(g_1) \circ \rho(g_2)) (x) && \text{(by definition of function composition)}.
\end{align*}
Since the values of $\rho(g_1 g_2)$ and $\rho(g_1) \circ \rho(g_2)$ agree on every element $x \in X$, these two permutations are equal. We conclude that $\rho$ is a homomorphism, since $g_1$ and $g_2$ were arbitrary elements of $G$.
\end{proof}


\begin{prop} \label{hom-yields-action}
Any homomorphism $\psi$ from the group $G$ into the symmetric group $S_X$ on a set $X$ gives rise to an action of $G$ on $X$, defined by taking $g \cdot x = \psi(g)(x)$.
\end{prop}
\begin{proof}
Suppose  that we have a homomorphism $\psi$ from $G$ into $S_X$.  We can define a map from $G \times X$ to $X$  by $g \cdot x = \psi(g)(x)$. We verify that this map satisfies the definition of a group action of $G$ on $X$:
\\ (axiom \ref{grp-action-axiom-1}) \quad $1 \cdot x = \psi(1)(x) = id_X(x) = x$
\\(axiom \ref{grp-action-axiom-2}) \quad $(gh) \cdot x = \psi(gh)(x) = (\psi(g)\psi(h))(x) = \psi(g)(\psi(h)(x)) = g \cdot (h \cdot x)$
\end{proof}

\begin{cor} \label{equivalence-of-actions}
The actions of $G$ on the set $X$ are in bijective correspondence with the homomorphisms from $G$ into the symmetric group $S_X$.
\end{cor}
\begin{proof}
By Proposition \ref{action-yields-hom}, any action of $G$ on $X$ yields a homomorphism from $G$ into $S_X$.    Conversely, any homomorphism from $G$ into $S_X$ establishes an action of $G$ on $X$ by Proposition \ref{hom-yields-action}.
\end{proof}


%Todo\colon add examples?

%----------------------------------------------------------------------------------------
%	The Definition of a Representation
%
\section{The Definition of a Representation}
%----------------------------------------------------------------------------------------
\begin{defn} \label{rep-def-2}Let $G$ be a group, let $F$ be a field, and let $V$ be a vector space over $F$. A \textbf{linear representation} of $G$ is an action of $G$ on $V$ which preserves the linear structure of $V$, i.e. an action of $G$ on $V$ such that
\begin{align}
\label{rep-axiom-1}&g \cdot (v_1+v_2)=g \cdot v_1+g \cdot v_2 \quad && \forall g \in G, v_1, v_2 \in V \\
\label{rep-axiom-2}&g \cdot (kv) = k (g \cdot v) \quad && \forall g \in G, v \in V, k \in F
\end{align}
 \end{defn}

\begin{defn}[Alternative definition]
\label{rep-def-1}
Let $G$ be a group, let $F$ be a field, and let $V$ be a vector space over $F$.  A \textbf{linear representation} of G is any group homomorphism $\rho\colon G \to GL(V)$. If we fix a basis for $V$, we get a representation in the previous sense.\end{defn}



 
 \begin{note}
 Unless otherwise specificed, we will restrict our discussion to \textit{finite-dimensional} representations over $\mathbb{C}$.
 \end{note}
 
 
 \begin{prop}
The definitions of a linear representation given in \ref{rep-def-1} and \ref{rep-def-2} above are equivalent.
 \end{prop}
 \begin{proof}
%\leavevmode
 \begin{itemize}
\item[$(\rightarrow)$]  Suppose that we have a homomorphism $\rho \colon G \to GL(V)$.  Note that $GL(V)$ is a subgroup of the symmetric group $S_V$ on $V$, so we can apply Proposition \ref{hom-yields-action} to obtain an action of $G$ on $V$ by $g \cdot v = \rho(g)(v)$.  We check that this action preserves the linear structure of V.
\\\ref{rep-axiom-1} \quad For any $g \in G$, $v_1, v_2 \in V$ we have $g \cdot (v_1 +  v_2) = \rho(g) (v_1 + v_2) = \rho(g)(v_1) + \rho(g)(v_2)= g \cdot v_1 + g \cdot v_2$.
\\\ref{rep-axiom-2} \quad For any $g \in G, v \in V, k \in F$ we have $g \cdot (kv) = \rho(g)(kv) = k (\rho(g)(v)) = k (g \cdot v)$.
\item[$(\leftarrow)$] Suppose that we have an action of $G$ on $V$ which preserves the linear structure of V in the sense of Definition \ref{rep-def-2}.  We can apply Proposition \ref{action-yields-hom} to obtain a homorphism $\rho \colon G \to S_V$ given by $\rho(g) = \sigma_g$ where $\sigma_g(v) = g \cdot v $.  It remains to show that the image $\rho(G)$ of $G$ under $\rho$ is actually contained in $GL(V)$, i.e. that for each $g \in G$ the map $\sigma_g$ is linear.  Fix an element $g \in G$. For any $k \in F$ and $v \in V$, we have
\begin{align*}
\sigma_g (kv) &= g \cdot (kv) && \text{(by definition of $\sigma_g$)} \\
		&= k (g \cdot v) && \text{(by property \ref{rep-axiom-1})} \\
		&= k (\sigma_g (v)) && \text{(by definition of $\sigma_g$)}.
\end{align*}
Also, for any $v_1, v_2 \in V$ we have
\begin{align*}
\sigma_g (v_1 + v_2) &= g \cdot (v_1 + v_2) && \text{(by definition of $\sigma_g$)} \\
		&= g \cdot v_1 + g \cdot v_2 && \text{(by property \ref{rep-axiom-2})} \\
		&= \sigma_g(v_1) + \sigma_g(v_2) && \text{(by definition of $\sigma_g$)}.
\end{align*}
Thus $\sigma_g$ is linear, and $\rho(G) \subset GL(V)$ proves that we  have a homomorphism $\rho \colon G \to GL(V)$.

\end{itemize}
 \end{proof}
 \begin{defn}Let $G$ be a group, let $F$ be a field, let $V$ be a vector space over $F$, and let $\rho \colon G \to GL(V)$ be a representation of $G$.  The \textbf{dimension} of the representation is the dimension of $V$ over $F$.  
 \end{defn}
% ============== Examples of Representations ===========%
 \begin{example}
 \begin{enumerate}
\item Let $V$ be a $1$-dimensional vector space over the field $F$.  The map $\rho \colon G \to GL(V)$ defined by $\rho(g) = 1$ for all $g \in G$ is a representation called the \textit{trival representation} of $G$.  The trivial representation has dimension $1$.

\item If $G$ is a finite group that acts on a finite set $X$, and $F$ is any field, then there is an associated \textit{permutation representation}  on the vector space $V$ over $F$ with basis $\{e_x \colon x \in X\}$.  We let $G$ act on the basis elements by the permutation $g \cdot e_x = e_{gx}$ for all $x \in X$ and $g \in G$. This representation has dimension $|X|$. 

\item A fundamental special case of a permutation representation that we shall return to later on is that when a finite group acts on itself by left multiplication. We take the vector space $V_{\text{reg}}$ which has a basis given by the formal symbols $\{ e_g | g \in G \}$, and let $G$ act by $\rho_{\text{reg}}(h) (e_g) = e_{hg}$.  This representation is called the \textit{regular representation} of $G$, and has dimension $|G|$. We shall see later that this representation encodes information about all other representations of $G$.

\item For any symmetric group $S_n$, the \textit{alternating representation} on $V=\mathbb{C}$ is given by the map $\rho \colon S_n \to GL(\mathbb{C})=\mathbb{C}^\times$ defined by $\rho(\sigma)=\text{sgn}(\sigma)$. More generally, for any group $G$ with a subgroup $H$ of index $2$, we can define an \textit{alternating representation} $\rho \colon G \to GL(\mathbb{C})$ by letting $\rho(g) = 1$ if $g \in H$ and $\rho(g) = -1$ if $g \notin H$.  (We recover our original example  by taking $G= S_n$ and $H=A_n$.)

\end{enumerate}
 \end{example}

%=========Homomorphisms===================%

\begin{defn}
A \textbf{homomorphism} between two representations $\rho_1 \colon G \to GL(V)$ and $\rho_2 \colon G \to GL(W)$ is a linear map $\psi \colon V \to W$ that interwines with (respects) the $G$-action, i.e. 
\[ \psi \circ \rho_1 (g)= \rho_2(g) \circ \psi \quad \forall  g \in G. \]  
An \textbf{isomorphism} of representations is a homomorphism of representations that is also an invertible map.
\end{defn}
\begin{note}
If we have representations $(\rho_1, V)$ and $(\rho_2, W)$ and an isomorphism of vector spaces $\psi \colon V \to W$ then we can rewrite the compatibility requirement above as $\rho_2(g) = \psi \circ \rho_1(g) \circ \psi^{-1}$ for all $g \in G$.
\end{note}

Given any representation $(\rho, V)$ of a group $G$ on a vector space $V$ over a field $F$ of dimension $n$, we can fix a basis for $V$ to obtain an isomorphism of vector spaces $\psi \colon V \to F^n$.  This yields a representation $\phi$ of $G$ on $F^n$ by defining $\phi (g) = \psi \circ \rho(g) \circ \psi^{-1}$ for all $g \in G$. Clearly, this representation is isomorphic to our original representation $(\rho, V)$. In particular, this means we can always choose to view $n$-dimensional complex representations as representations on $\mathbb{C}^n$ where each $\phi(g)$ is given by an $n \times n$ matrix with entries in $\mathbb{C}$.

Suppose that we have two representations $\rho_1 \colon G \to GL_n(F)$ and $\rho_2 \colon G \to GL_m(F)$ given by $\rho_1(g) = X_g$ and $\rho_2(g) = Y_g$.  A homomorphism between these representations is then an $m \times n$ matrix $A$ such that $A X_g = Y_g A$ for all $g \in G$.   An isomorphism is given precisely when such $A$ is square and invertible.  Thus, two representations $\rho_1 \colon G \to GL_n(F)$ and $\rho_2 \colon G \to GL_n(F)$ are isomorphic if and only if there exists $A \in GL_n(F)$ such that $\rho_1(g) = A \rho_2(g) A^{-1}$ for all $g \in G$.  This establishes the following proposition:
\begin{prop} \label{iso-classes-of-reprns}
The isomorphism classes of $n$-dimensional representations of $G$ on $\mathbb{C}$ are in bijection with the quotient $Hom(G; GL_n(\mathbb{C})) / GL_n(\mathbb{C})$ of group homomorphisms $G \to GL_n(\mathbb{C})$ modulo the conjugation action of $GL_n(\mathbb{C})$.
\end{prop}
\begin{note}
In other words, 
\end{note}
\begin{defn} A \textbf{subrepresentation} of $V$ is a $G$-invariant subspace $W \subseteq V$; that is, a subspace $W \subseteq V$ with the property that $\rho(g) (w) \in W$ for all $g \in G$ and $w \in W$.  Note that $W$ itself is a representation of $G$ under the action $\rho(g) \restriction_W$.
\end{defn}


\begin{example}
Let $G = \{ (1), (123), (132) \} \subset S_3$.  Note that $G$ is a subgroup of $S_3$, and is isomorphic to the cyclic group of order $3$. Let $V= \mathbb{C}^3$.  Then $G$ acts on the standard basis by $g \cdot e_i = e_ {gi}$.  Thus, the permutation representation of $G$ (with respect to the standard basis) is given by:
\begin{align*}
\rho((1)) &= \begin{bmatrix} 1 & 0 & 0 \\ 0 & 1 & 0 \\ 0 & 0 & 1 \end{bmatrix} \\
\rho((123)) &= \begin{bmatrix} 0 & 0 & 1 \\ 1 & 0 & 0 \\ 0 & 1 & 0 \end{bmatrix} \\
\rho((132)) &= \begin{bmatrix} 0 & 1  & 0 \\ 0 & 0 & 1 \\ 1 & 0 & 0 \end{bmatrix}.
\end{align*}
\end{example}

\begin{example}
Let $G= C_2 \times C_2$ be the Klein four-group, i.e. the group generated by $\sigma$, $\tau$ such that $\sigma^2 = \tau^2 = e$ and $\sigma \tau = \tau \sigma$.  Let $V$ be the vector space with basis $\{ b_e, b_\sigma, b_\tau, b_{\sigma \tau} \}$.  Left multiplication by $\sigma$ gives a permutation 
\begin{align*}
b_e &\mapsto b_\sigma\\
b_\sigma &\mapsto b_e \\
b_ \tau &\mapsto b_{\sigma \tau}\\
b_{\sigma \tau} &\mapsto b_\tau
\end{align*}
Thus, with respect to our given basis of $V$, the regular representation $\rho_{\text{reg}} \colon G \to GL(V)$ evaluated at $\sigma$ is given by the matrix
\[  \rho_{\text{reg}}(\sigma) = \begin{bmatrix}0 & 1 & 0 & 0 \\  1 & 0 & 0 & 0 \\ 0 & 0 & 0 & 1 \\ 0 & 0 & 1 & 0 \end{bmatrix} \]
We similarly calculate
\[  \rho_{\text{reg}}(\tau) = \begin{bmatrix}0&0&1&0 \\ 0&0&0&1 \\ 1&0&0&0 \\ 0&1&0&0 \end{bmatrix} \]
\[  \rho_{\text{reg}}(\sigma \tau) = \begin{bmatrix}0&0&0&1 \\ 0&0&1&0 \\ 0&1&0&0 \\ 1&0&0&0 \end{bmatrix} \]
\end{example}

\begin{example}\label{rep-of-d8}
Let $G = D_4 = \langle \sigma, \tau |  \sigma^4 = \tau^2 = e, \tau \sigma \tau^{-1} = \sigma^{-1} \rangle$ be the symmetry group of the square.  We can use geometry to construct a representation of $G$.  Consider a square in the plane with vertices at $(1,1), (1,-1), (-1, -1)$, and $(-1, 1)$.  We let $\sigma$ act on the square as a rotation by $\frac{\pi}{2}$, and let $\tau$ act by reflecting over the $x$-axis.  This naturally gives rise to a linear action of $G$ on all of $\mathbb{C}^2$.  Under the standard basis, we get the matrices:
\begin{align*}
\rho( e) &= \text{Id}_2 \\
\rho (\sigma) &= \begin{bmatrix} 0 & -1 \\ 1 & 0  \end{bmatrix} \\
\rho (\sigma^2) &= \begin{bmatrix} -1 & 0 \\ 0 & -1  \end{bmatrix} \\ 
\rho (\sigma^3) &= \begin{bmatrix} 0 & 1 \\ -1 & 0  \end{bmatrix} \\
\rho (\tau) &= \begin{bmatrix} 1 & 0 \\ 0 & -1\end{bmatrix} \\
\rho (\sigma \tau ) &= \begin{bmatrix} 0 & 1 \\ 1 & 0\end{bmatrix} \\
\rho (\sigma^2 \tau) &= \begin{bmatrix} -1 & 0 \\ 0 & 1\end{bmatrix} \\
\rho (\sigma^3 \tau) &= \begin{bmatrix} 0 & -1 \\ -1 & 0\end{bmatrix}
\end{align*}

\end{example}

\begin{example} \label{basic-example-of-hom-of-reps}
Let $G = C_2 = \langle \tau | \tau^2 = e \rangle$ be the cyclic group of order $2$.  The regular representation of $G$, written in the standard basis, is 
\[ \rho_{\text{reg}}(\tau)= \begin{bmatrix} 0 & 1 \\ 1 & 0 \end{bmatrix} \]
and $\rho_{\text{reg}}(e) = \text{Id}_2$.  Let $\rho_{\text{sgn}}$ be the alternating representation of $G$ on $\mathbb{C}$, i.e.
\begin{align*}
 \rho_{\text{sgn}} \colon G &\to GL_1 (\mathbb{C}) = \mathbb{C} ^ {\times} \\
\tau &\mapsto -1 \\
e &\mapsto 1
 \end{align*}
Let $f \colon \mathbb{C}^2 \to \mathbb{C}$ be the linear map represented by the matrix $\begin{bmatrix} 1 & -1 \end{bmatrix}$ in the standard basis.  Then for any vector $x = \begin{bmatrix}x_1 \\ x_2 \end{bmatrix} \in \mathbb{C}^2$, we have 
\begin{align*}
f \circ \rho_{\text{reg}} (\tau) (x) &= \begin{bmatrix} 1 & -1 \end{bmatrix}  \begin{bmatrix} 0 & 1 \\ 1 & 0 \end{bmatrix}  \begin{bmatrix}x_1 \\ x_2 \end{bmatrix}  \\
&= - \begin{bmatrix} 1 & -1 \end{bmatrix}  \begin{bmatrix}x_1 \\ x_2 \end{bmatrix} \\
&= \rho_{\text{sgn}} (\tau) \circ f(x).
\end{align*}
Thus, $f$ is a $G$-linear map from $\rho_{\text{reg}}$ to $\rho_{\text{sgn}}$ (i.e. a homomorphism of representations).  

Now, let $W$ be the subspace of $\mathbb{C}^2$ spanned by the vector $\begin{bmatrix}1 \\ 1  \end{bmatrix}$. Then
\[ \rho_{\text{reg}}(\tau)  \begin{bmatrix}1 \\ 1  \end{bmatrix} =   \begin{bmatrix} 0 & 1 \\ 1 & 0 \end{bmatrix} = \begin{bmatrix}1 \\ 1  \end{bmatrix} = \begin{bmatrix}1 \\ 1  \end{bmatrix} \]
and  $\rho_{\text{reg}}(e) \begin{bmatrix}1 \\ 1  \end{bmatrix} = \begin{bmatrix}1 \\ 1  \end{bmatrix} $, so $W$ is a subrepresentation of $\rho_{\text{reg}}$.  (In fact, $W$ is isomorphic to the $1$-dimensional trivial representation of $G$.)
\end{example}

\begin{example}
We can generalize $G$-invariant subspace from the previous example.  Suppose we have a representation $\rho \colon G
\to GL_n
(\mathbb{C})$.  If
we can find a vector $x \in \mathbb{C}^n$ which is an eigenvector for every matrix $\rho(g), g \in G$,
i.e. an $x \in \mathbb{C}^n$ such that
\[ \rho(g) (x) = \lambda_g (x) \quad \forall g \in G\]
for some eigenvalues $\lambda_g \in \mathbb{C}$, then the span of $x$ is a $1$-dimensional $G$-invariant
subspace
 of $\mathbb{C}^n$.  It is isomorphic to the $1$-dimensional representation
\begin{align*}
 \rho_2 \colon G &\to GL_1 (\mathbb{C}) \\
g &\mapsto \lambda_g.
\end{align*}
\end{example}


\begin{prop}\label{ker-im-subreprns}
Let $f \colon V \to W$ be a homomorphism of representations of $G$.  Then $\text{Ker}(f)$ is a subrepresentation of $V$ and $\text{Im}(f)$ is a subrepresentation of $W$.
\end{prop}
\begin{proof}
Let $x \in \text{Ker}(f)$. Then $0 = g0 = g f(x) = f(gx)$ for every $g \in G$,
so $gx \in \text{Ker}(f)$ and $\text{Ker}(f)$ is $G$-invariant.

Now let $w \in \text{Im}(f)$. There exists $v \in V$ such that $w = f(v)$, so $g w = g f(v) = f
(gv)$ for
every $g \in G$. Thus $gw \in \text{Im}(f)$, and $\text{Im}(f)$ is $G$-invariant.
\end{proof}
\begin{note}
Recall the subrepresentation $W$ that we constructed in Example \ref{basic-example-of-hom-of-reps}.  This
subrepresnentation
 is
precisely equal
 to the
kernel of the map $f$ from that example.
\end{note}


%----------------------------------------------------------------------------------------
%	Representations of Cyclic Groups
%
\section{Representations of Cyclic Groups}
%----------------------------------------------------------------------------------------
\begin{example}[Representations of $\mathbb{Z}$]
Let's try to classify all representations of the group $\mathbb{Z}$ under addition.  Consider an $n$-dimensional representation $\rho \colon \mathbb{Z} \to GL_n (\mathbb{C})$.  For $\rho$ to be a group homomorphism requires that $\rho(0) = \text{Id}$.  Observe that for any $0 \neq n \in \mathbb{Z}$, we have $\rho(n) = \rho( 1 + \ldots + 1) = \rho(1)^n$.  Then $\rho$ is completely determined by the matrix $\rho(1) \in GL_n(\mathbb{C})$, and any such matrix determines a representation of $\mathbb{Z}$.  It follows that the $n$-dimensional isomorphism classes of representations of $\mathbb{Z}$ are in bijection with the conjugacy classes in $GL_n(\mathbb{C})$.  These conjugacy classes can be parameterized by the \textit{Jordan canonical form}.
\end{example}

\begin{example}[Representations of the cyclic group of order $n$]
We shall classify all representations of the cyclic group $G = \{ g, g^2, \ldots, g^{n-1}, g^n=1\}$ of order $n$. Consider a representation $\rho \colon G \to GL(V)$.  As in the previous example, we know that $\rho(1) = \text{Id}$ and $\rho(g^k) = \rho(g)^k$.  Thus our representation $\rho$ is determined completely by the linear transformation $\rho(g)$.   It will be helpful to fix a basis of $V$ so that we may view $\rho(g)$ as a matrix.  Recall from linear algebra that there exists a basis in which $\rho(g)$ takes the \textit{Jordan canonical form}

\[ \rho(g) = \begin{bmatrix}
    J_1 & 0 & \dots  &0 \\
  0 & J_2  & \dots & 0 \\
    \vdots & \vdots  & \ddots & \vdots \\
    0& 0&  \dots  & J_m
\end{bmatrix} \]
where each \textit{Jordan block} $J_k$ is of the form 
\[J_k =  \begin{bmatrix}
    \lambda & 1&0& \dots  &0 & 0 \\
     0 &\lambda& 1& \ddots & 0  & 0 \\
     0 & 0 & \lambda & \ddots& 0  & 0 \\
     0 & 0 & 0 & \ddots & 1 & 0 \\
    \vdots & \ddots & \ddots & \ddots & \ddots  & 1\\
    0& 0& 0 & \dots  & 0  &\lambda
\end{bmatrix}. \]
Now $I = \rho(g)^n$ is a block-diagonal matrix with diagonal blocks $J_k^n$, so we must have that each block $J_k^n=\text{Id}$.  Observe that we can write each block $J_k$ as $J_k = \lambda \text{Id} + N$ where $N$ is the Jordan block with $\lambda = 0$.  Thus we have 
\[ \text{Id} = J_k^n = (\lambda \text{Id} + N)^n = \lambda ^n \text{Id} + \binom{n}{1} \lambda ^{n -1} N + \binom{n}{2} \lambda ^{n-2} N^2 + \ldots + \binom {n} {n -1} \lambda N^{n -1} + N^n. \] The following lemma will show that in fact $N=0$. 
\begin{lemma}
Let $N$ be the Jordan block with $\lambda = 0$ of size $n \times n$.  For any integer $p$ with $1 \leq p \leq n - 1$, then $N^p$ is the matrix with ones in the positions $(i,j)$ where $j = i + p$ and zeroes everywhere else.  (The ones lie along a line parallel to the diagonal, $p$ steps above it.)

\begin{proof}
(By induction.)

 {\textit{Base case:}} This is simply the definition of $N$.

 \textit{Inductive step:} Suppose that the lemma holds for $N^p$.  We compute the $(i,j)$ entry of $N^{p+1}$:
\[ (N^{p+1})_{i,j} = \sum_{k=1}^{n} (N^{p})_{i,k} N_{k, j} = (N^p)_{i, i +p} N_{i +p, j} = N_{i +p, j} =\begin{cases} 
      1 & \text{if}  j = i + (p +1) \\
      0 & \text{otherwise} \\
   \end{cases}  \]

\end{proof}
\end{lemma}

Now, if $N \neq 0$ then each term $\binom {n}{k} \lambda ^ {n - k } N ^k$ for $k > 0$ would yield some non-zero non-diagonal entries (in the positions $(i,j)$ where $j= i + k$) and hence our sum could not equal the identity matrix.  We must conclude that $N = 0$, $J_k = \lambda \text{Id}$ is a $1 \times 1$ block, and $J_k ^n = \lambda ^ n \text{Id}$.  Thus $\rho(g)$ is a diagonal matrix with $n$th roots of unity as diagonal entries. 

To summarize, every $m$-dimensional representation $\rho$ of the cyclic group $G = \langle g \rangle$ of order $n$ can be seen to act (with the right choice of basis) as $m \times m$ diagonal matrices all with $n$th roots of unity along the diagonal.  In particular, these representations are determined completely by the value of $\rho(g)$ and are classified up to isomorphism by unordered $m$-tuples of $n$th roots of unity.
\end{example}

\section{Constructing New Representations from Old}
We know from linear algebra that given two vector spaces $V$ and $W$, we can form the \textbf{direct sum} $V \oplus W$ consisting of ordered pairs $(v ,w)$ where $v \in V, w \in W$.  

\begin{defn}
Let $V$ and $W$ be representations of $G$.  Then $V \oplus W$ admits a  natural representation of $G$, called the \textbf{direct sum representation} of $V$ and $W$, which we define by 
\begin{align*}
\rho_{V \oplus W} \colon G &\to GL(V \oplus W) \\
\rho_{V \oplus W}(g) \colon (x,y) &\mapsto (\rho_{V} (g)(x), \rho_{W}(g)(y)).
\end{align*}
\end{defn}


Recall from linear algebra that given a subspace $W \subseteq V$, we can form the \textbf{quotient space} $V / W$ consisting of cosets $v + W$ in $V$.  If $W$ is a subrepresentation of $V$, we would like to define an action of $G$ on $V / W$ by the natural choice of $g (v + W) = \rho(g)(v)+ W$.  It remains to verify that this action is well defined.  If we choose another $v' \in v + W$, then $v - v' \in W$, so that $\rho(g)(v - v') \in W$ since $W$ is $G$-invariant.  Thus, the cosets $\rho(g)(v) + W$ and $\rho(g)(v') + W$ agree and this action is indeed well defined. This justifies the following definition:

\begin{defn}
Let $W$ be a $G$-subrepresentation of $V$.  Then $V/W$ forms a representation of $G$ called the \textbf{quotient representation} of $V$ under $W$ with the action $g( v + W) = \rho(g)(v) + W$.
\end{defn}



\section{Complete Reducibility}
\begin{defn}
A representation is said to be \textbf{irreducible} if it has no subrepresentations other than the trivial subrepresentations $ 0 \subset V$ and $V \subset V$.  A representation is called \textbf{completely reducible} if it decomposes into a direct sum of irreducible subrepresentations.
\end{defn}

\begin{note}
\begin{enumerate}
\item Any $1$-dimensional representation $V$ has no subspaces other than $0$ and $V$ itself, and is thus irreducible.
\item Any irreducible representation is, in particular, completely reducible.
\end{enumerate}
\end{note}

\begin{example}[A $2$-dimensional irreducible representation] \label{2d-irrep-d3}
Let $G = D_3 = \langle \sigma, \tau | \sigma^3 = \tau^2 = e, \tau \sigma \tau^{-1} = \sigma^{-1} \rangle$. (Note that $D_3 \cong S_3)$.
Consider the regular triangle centered at the origin with vertices
\[(1,0), (-\frac{1}{2}, \frac{\sqrt{3}}{2}), (-\frac{1}{2}, - \frac{\sqrt{3}}{2}). \]
We can let $\sigma$ act as rotation by $\frac{2 \pi}{3}$ and let $\tau$ act as reflection over the $x$-axis to obtain an action of $G$ on $\mathbb{C}^2$ given (under the standard basis) by the matrices
\begin{align*}
\rho(\sigma) &= \begin{bmatrix}-\frac{1}{2} & -\frac{\sqrt{3}}{2} \\ \frac{\sqrt{3}}{2} & -\frac{1}{2}\end{bmatrix} \\
\rho(\tau) &= \begin{bmatrix}1 & 0 \\ 0 & -1 \end{bmatrix}
\end{align*}
Suppose $\rho$ has a non-trivial subrepresentation $W$.  We must have $\text{dim }W =1$.  Since $W$ is invariant under the action of both $\rho(\sigma)$ and $\rho(\tau)$, there must be some mutual eigenvector for $\rho(\sigma)$ and $\rho(\tau)$ that spans $W$.  The eigenvectors of $\rho(\sigma)$ are
\begin{align*}
\begin{bmatrix} 1 \\ -i \end{bmatrix} \quad (\lambda_1 = e^{\frac{2 \pi i}{3}}) \quad \text{and} \quad
\begin{bmatrix} 1 \\ i \end{bmatrix} \quad (\lambda_2 = e^{-\frac{2 \pi i}{3}}) .
\end{align*}  
The eigenvectors of $\rho(\tau)$ are 
\begin{align*}
\begin{bmatrix} 1 \\ 0 \end{bmatrix} \quad (\lambda_1 = 1) \quad \text{and} \quad
\begin{bmatrix} 0 \\ 1 \end{bmatrix} \quad (\lambda_2 = -1).
\end{align*}
Thus we see that there is no such $W$, and our representation is irreducible.
Finally, we note that under a change of basis to diagonalize $\rho(\sigma)$we have
\begin{align*}
\rho(\sigma) &= \begin{bmatrix}w & 0 \\ 0 &w^{2}  \end{bmatrix} \\
\rho(\tau) &= \begin{bmatrix} 0 & 1 \\ 1 & 0 \end{bmatrix}.
\end{align*} 
(Conjugate our original matrices by $A \mapsto S^{-1}AS$ where $S$ has columns given by the eigenvectors of $\rho(\sigma)$.)
\end{example}

\begin{thm}\label{simultaneous} If $A_1, A_2, \ldots, A_r$ are linear operators on $V$ and each $A_i$ is diagonalizable, they are simultaneously diagonalizable if and only if they commute.
\end{thm}
\begin{proof}
 See Conrad \cite[Theorem 5.1] {ConradMinPoly}.
\end{proof}

\begin{thm} Every complex representation of a finite abelian group is completely reducible into a direct sum of irreducible representations of dimension $1$.  
\end {thm}
\begin{proof}
Take an arbitrary element $g \in G$.  Since $G$ is finite, we can find an integer $n$ such that $g^n = 1$ and $\rho(g)^n = Id$.    Hence the minimal polynomial of $\rho(g)$ divides  $x^n -1$.  Recall that $x^n-1$ has $n$ distinct roots over $\mathbb{C}$, which are generated by taking powers of $\xi = e^{\frac{2 \pi i}{n}}$.  This means that the minimal polynomial $\rho(g)$ factors into linear factors only over $\mathbb{C}$ so that $\rho(g)$ is diagonalizable.  We conclude that each $\rho(g)$ is (separately) diagonalizable since $g \in G$ was arbitrary.

Now, given any two elements $g_1, g_2 \in G$ we have 

\begin{align*}
\rho(g_1) \rho(g_2)&= \rho(g_1 g_2)&& \text{(since $\rho$ is a homomorphism)} \\
		&=  \rho(g_2 g_1) && \text{(since $G$ is abeilian)} \\
		&= \rho(g_2) \rho(g_1) && \text{(since $\rho$ is a homomorphism)}.
\end{align*}
Thus the matrices $\left\{ \rho(g)\right\}$ commute, so we may apply Theorem \ref{simultaneous} to conclude that $\left\{ \rho(g)\right\}$ are simultaneously diagonalizable, say with basis $\left\{ e_1, ..., e_k \right\}$.  Then we have $V= \mathbb{C}e_1 \oplus \mathbb{C} e_2 \oplus \ldots \oplus \mathbb{C} e_n$, with each subspace $ \mathbb{C}e_1$ invariant under the action of $G$.
\end{proof}

\begin{defn}
Let $W$ be a subspace of $V$.  A \textbf{linear projection} $V$ onto $W$ is a linear map $f \colon V \to W$ such that $f \restriction_{W} = \text{Id}_W$.  If $W$ is a subrepresentation of $V$ and the map $f$ is $G$-invariant, then we say that $f$ is a $\mathbf{G}$\textbf{-linear projection}.
\end{defn}

\begin{lemma} \label{maschke-lemma}
Let $\rho \colon G \to GL(V)$ be a representation, and $W \subset V$ be a subrepresentation.  Suppose we have a $G$-linear projection 
\[ f \colon V \to W. \]
Then $\text{Ker}(f)$ is a complementary subrepresentation to $W$, i.e. $\text{Ker}(f)$ is a $G$-invariant subspace of $V$ such that
\[ V = \text{Ker}(f) \oplus W \]
\end{lemma}
\begin{proof}
We have seen that $\text{Ker}(f)$ is a subprepresentation of $V$ in Proposition \ref{ker-im-subreprns}.
Now if $y \in \text{Ker}(f) \cap W$ then $y=f(y)=0$, so $\text{Ker}(f) \cap W = 0$.  Finally $\text{Im}(f)=W$, so by the Rank-Nullity theorem
\[ \text{dim Ker}(f) + \text{dim }W = \text{dim }V. \]
Thus $V = \text{Ker}(f) \oplus W$.  
\end{proof}

\begin{thm}[Maschke's Theorem] \label{maschke}
Let $G$ be a finite group and let $F$ be a field such that $\text{char}(F) \nmid |G|$.  If $V$ is any finite-dimensional representation of $G$ over $F$, and $W \subset V$ is a subrepresentation of $V$, then there exists a complementary subrepresentation $U \subset V$, i.e. there is  a $G$-invariant subspace $U \subset V$ such that 
\[ V = W \oplus U. \]
\end{thm}
\begin{proof}
By the previous Lemma \ref{maschke-lemma} it will suffice to find a $G$-linear projection from $V$ onto $W$.  Fix a basis $\{ b_1, \ldots, b_m \}$ for $W$ and extend it to a basis  $\{ b_1, \ldots, b_m, b_{m+1}, \ldots, b_n \}$ for $V$.  Let $U = \langle b_{m+1}, \ldots, b_n \rangle$.  Then $U$ is certainly a complementary subspace to $W$, and we have a natural projection $f \colon W \oplus U \to W$ of $V$ onto $W$ with kernel $U$.  There is no reason to think that $f$ should be $G$-linear, but we can fix this by averaging over $G$.  Define $\widetilde{f} \colon V \to V$ by
\[\widetilde{f}(x) = \frac{1}{|G|} \sum_{g \in G} (\rho(g) \circ f \circ \rho(g^{-1}))(x). \]

We claim that $\widetilde{f}$ is a $G$-linear projection from $V$ onto $W$.  
First we check that $\text{Im}(f) \subset W$.  If $x \in V$ and $g \in G$, then
\[ f (\rho (g^{-1})(x)) \in W \]
and so
\[ \rho(g) ( f ( \rho( g^{-1})(x))) \in W \]
since $W$ is $G$-invariant.  Thus $\widetilde{f}(x) \in W$.  Next we check that $\widetilde{f} \restriction_{W} = \text{Id}_W$. Let $y \in W$.  For any $g \in G$, we know that $\rho(g^{-1})(y)$ is also in $W$, so
\[ f (\rho(g^{-1})(y)) = \rho (g^{-1})(y).\]
Then
\begin{align*}
\widetilde{f}(y) &=  \frac{1}{|G|} \sum_{g \in G} \rho(g) (f ( \rho(g^{-1})(y))) \\
&=\frac{1}{|G|} \sum_{g \in G} \rho(g) (\rho(g^{-1})(y)) \\
&= \frac{1}{|G|} \sum_{g \in G} \rho(g g^{-1}) (y) \\
&=\frac{1}{|G|} \sum_{g \in G} (y) \\
&= \frac{|G| y} {|G|}
\end{align*}
so indeed $\widetilde{f}$ is a linear projection of $V$ onto $W$.  Finally, we check that $\widetilde{f}$ is $G$-linear.  If $x \in V$ and $h \in G$, then
\begin{align*}
(\widetilde{f} \circ \rho(h))(x) &= \frac{1}{|G|} \sum_{g \in G} (\rho(g) \circ f \circ \rho(g^{-1}) \circ \rho(h))(x) \\
&= \frac{1}{|G|} \sum_{g \in G} (\rho(g) \circ f \circ \rho(g^{-1} h))(x) \\
&=\frac{1}{|G|} \sum_{g \in G} (\rho(hg) \circ f \circ \rho(g^{-1}))(x) \quad \text{(relabelling } g \mapsto hg \text{)} \\
&= (\rho(h) \circ \widetilde{f}) (x).
\end{align*}
\end{proof}

\begin{cor}
Let $G$ be a finite group and let $F$ be a field such that $\text{char}(F) \nmid |G|$. then any finite-dimensional representation of $G$ over $F$ is completely reducible.
\end{cor}
\begin{proof}
Let $V$ be a representation of $G$ over $F$ of dimension $n$.  If $V$ is irreducible, then $V$ is, in particular, completely reducible.  If not, then $V$ contains a proper subrepresentation $W \subset V$.  From Maschke's Theorem (\ref{maschke}), we know there exists a subrepresentation $U \subset V$ such that 
\begin{equation} \label{1.28.1} V = W \oplus U. \end{equation}
Both $W$ and $U$ have dimension less than $n$, so by induction we know that $W$ and $U$ are completely reducible. We deduce from \ref{1.28.1} that $V$ is completely reducible.
\end{proof}

\begin{example}
Let us once again return to Example \ref{basic-example-of-hom-of-reps}.  We had $G= C_2$, and found a $1$-dimensional subrepresentation 
\[ W = \left< \begin{bmatrix}1 \\ 1 \end{bmatrix} \right> \subset V_{\text{reg}}= \mathbb{C}^2 . \]
We know a complementary subrepresentation to $W$ exists by Machke's Theorem, so let's try to find one.  Consider
\[ U =  \left< \begin{bmatrix}1 \\ -1 \end{bmatrix} \right> \subset V_{\text{reg}}. \]
Then
\[ \rho_{\text{reg}}(\tau) \begin{bmatrix}1 \\ -1 \end{bmatrix} =  \begin{bmatrix} 0 & 1 \\ 1 & 0 \end{bmatrix} \begin{bmatrix}1 \\ -1 \end{bmatrix} = -  \begin{bmatrix}1 \\ -1 \end{bmatrix} \]
so $U$ is $G$-invariant.  (In fact $U$ is isomorphic to the alternating representation $\rho_{\text{sgn}}$ from the Example \ref{basic-example-of-hom-of-reps}.)  Finally we see that $V = W \oplus U$, since $W \cap U = \{ 0 \}$ and $\text{dim } U + \text{dim } W = 2 = \text{dim } V$.
\end{example}




\section{Vector Spaces of Linear Maps}
\begin{defn}
Let $V$ and $W$ be vector spaces.  Recall that the set $\textbf{Hom}\mathbf{(V,W)}$ of linear maps from $V$ to $W$ is itself a vector space.  If $f_1, f_2$ are two linear maps from $V$ to $W$, then we define their sum by
\begin{align*}		
(f_1 + f_2) \colon &V \to W \\		
&x \mapsto f_1(x) + f_2(x)		
\end{align*}		
and we define scalar multiplication of $\lambda \in \mathbb{C}$ by		
\begin{align*}		
(\lambda f_1) \colon &V \to W \\		
&x \mapsto \lambda f_1(x).		
\end{align*}		
%page 33 of grp 2014'%		
Now suppose we have representations $\rho_V \colon G \to GL(V)$ and $\rho_W \colon G \to GL(W)$ of $G$. Then there is a natural representation of $G$ on the vector space $\text{Hom}(V,W)$ given by
\begin{align*}		
 \rho_{\text{Hom}(V,W)}(g)  \colon \text{Hom}(V,W) &\to \text{Hom}(V,W) \\		
 f &\mapsto \rho_{W}(g) \circ f \circ \rho_{V}(g^{-1})
 \end{align*}		
for all $g \in G$. Note that $\rho_{\text{Hom}(V,W)}(g)(f)$ is certainly a linear map from $V$ to $W$ since the composition of linear maps is linear.
\end{defn}

\begin{prop} $\rho_{\text{Hom}(V,W)}$ is a representation of $G$.  That is, the map		
\begin{align*}		
  \rho_{\text{Hom}(V,W)} \colon G &\to \text{GL}(\text{Hom}(V,W)) \\		
g &\mapsto \rho_{\text{Hom}(V,W)}(g).		
\end{align*}		
 is a homomorphism.		
\end{prop}		
\begin{proof}		
We must check two things:		
\begin{enumerate}			
\item For every $g \in G$,  $\rho_{\text{Hom}(V,W)}(g)$ is invertible.	
\item The map $g \mapsto  \rho_{\text{Hom}(V,W)}(g)$ is a homomorphism.	
\end{enumerate}		
First, we check that
\begin{align*}
\rho_{\text{Hom}(V,W)}(g) \circ \rho_{\text{Hom}(V,W)}(h) \colon f &\mapsto \rho_{\text{Hom}(V,W)} (g) ( \rho_W (h) \circ f \circ \rho_V (h^ {-1})) \\
&= \rho_W (g) \circ \rho_w (h) \circ f \circ \rho_V (h^{-1} ) \circ \rho_V(g^{-1}) \\
&= \rho_W (gh) \circ f \circ \rho_V (g ^{-1} h ^{-1}) \\
&= \rho_{\text{Hom}(V,W)}(gh)(f)
\end{align*}
so indeed $\rho_{\text{Hom}(V,W)}$ is a homomorphism.  We can use this fact to see that $\rho_{\text{Hom}(V,W)}(g^{-1})$ is inverse to $\rho_{\text{Hom}(V,W)}(g)$ as
\begin{align*}
\rho_{\text{Hom}(V,W)}(g) \circ \rho_{\text{Hom}(V,W)}(g^{-1}) \ &= \rho_{\text{Hom}(V,W)} (e) \\ 
&= \text{Id}_{\text{Hom}(V,W)} \\
&= \rho_{\text{Hom}(V,W)}(g^{-1}) \circ \rho_{\text{Hom}(V,W)}(g).
\end{align*}
Thus $\rho_{\text{Hom}(V,W)}(g)$ is invertible for every $g \in G$, and $\rho_{\text{Hom}(V,W)}$ is a representation of $G$.
\end{proof}

\begin{defn}		
Let $V$ and $W$ be two representations of $G$.  The set of $G$-linear maps from $V$ to $W$ forms a subspace of $\text{Hom}(V,W)$, which we denote by $\textbf{Hom}_\mathbf{G}\mathbf{(V,W)}$.  In other words, $\text{Hom}_{G}(V,W)$ is the vector space consisting of all \textit{homomorphisms of representations} between $V$ and $W$. 
\end{defn} 

\begin{defn}
Let $\rho \colon G \to GL(V)$ be a representation.  We define the \textbf{invariant subrepresentation} $V^G \subset V$ to be the set 
\[ \{ v \in V  \mid \rho(g)(v) = v,  \quad \forall g \in G \}. \]
\end{defn}
Note that $V^G$ is a subspace of $V$, and is also clearly a subrepresentation.   It is isomorphic to a trivial representation of some dimension.

\begin{prop}\label{invariant-subrprn}
Let  $\rho_V \colon G \to GL(V)$ and $\rho_W \colon G \to GL(W)$ be representations of $G$.  Then the subrepresentation 
\[ \text{Hom}_G (V,W) \subset \text{Hom} (V,W) \]
is precisely the invariant subrepresentation $\text{Hom}(V,W) ^G$ of $\text{Hom}(V,W)$.
\end{prop}
\begin{proof}
Let $f \in \text{Hom}(V,W)$.  Then $f$ is an element of the invariant subrepresentation $\text{Hom}(V,W) ^G$ iff we have
\begin{align*}
&f = \rho_{\text{Hom}(V,W)} (g)(f)  \quad \forall g \in G \\
\iff &f = \rho_W (g) \circ f \circ \rho_V (g^{-1}) \quad \forall g \in G \\ 
\iff & f \circ \rho_V (g) = \rho_W (g) \circ f \quad \forall g \in G
\end{align*}
which is exactly the condition that $f$ is $G$-linear, i.e. that $f \in \text{Hom}_G (V,W)$.
\end{proof}

\begin{lemma}\label{invariant-subrprn-of-dir-sum}
Let $A$ and $B$ be two representations of $G$.  Then \[ (A \oplus B)^ G = A^G \oplus B^G. \]
\end{lemma}
\begin{proof}Observe that
\begin{align*}
(a,b) \in (A \oplus B)^G \iff & \rho_{A \oplus B} (g) (a,b) = (a,b) &  \forall g \in G \\
\iff & (\rho_A (g) (a), \rho_B (g) (b)) = (a,b) &  \forall g \in G \\
\iff & (a,b) \in A^G \oplus B^G.
\end{align*}
\end{proof}

\begin{lemma}\label{isomorphism-invariant-subrprns}
Let $\psi \colon A \to B$ be an isomorphism between representations of $G$. Then $\psi$ induces an isomorphism between their invariant subrepresentations 
\[ \psi \restriction_{A^G} \colon A^G \to B^G.\]
\end{lemma}
\begin{proof}
Clearly the restriction of $\psi$ to $A^G \subset A$ induces an isomorphism to some subrepresentation of $B$, but we must check that the image of this restriction actually equals $B^G$.  We verify that 
\begin{align*}
a \in A^G \iff & \rho_A (g) (a) = a & \forall g \in G \\
\iff & \psi( \rho_A (g)(a)) = \psi (a) & \forall g \in G \\
\iff & \rho_B(g) \psi(a) = \psi(a) & \forall g \in G \\
\iff & \psi(a) \in B^G.
\end{align*}
\end{proof}

\section{Schur's Lemma}
\begin{thm}\label{schur-lemma-over-c}[Schur's Lemma over $\mathbb{C}$.] If $V$ is an irreducible representation of $G$ over $\mathbb{C}$, then evey linear operator $\phi \colon V \to V$ commuting with $G$ is a scalar.
\end{thm}
\begin{proof}
Let $\phi \colon V \to V$  be a linear operator commuting with $G$, and let $\lambda$ be an eigenvalue of $\phi$.  Observe that the eigenspace $E_\lambda$ is $G$-invariant: If $v \in E_\lambda$, then $\phi(v) = \lambda v$.  This implies that $\phi(g v) = g \phi(v) = g (\lambda v) = \lambda (gv)$, i.e. $gv \in E_\lambda$. Since $g$ was arbitrary, $E_\lambda$ is indeed $G$-invariant.  Now $E_\lambda \neq 0$, so since $V$ is irreducible, $E_\lambda = V$.  Thus $\phi = \lambda \text{Id}$.  
\end{proof}

\begin{cor}\label{schur-cor}
If $V$ and $W$ are irreducible, the space $\text{Hom}_G(V,W)$ is $1$-dimensional if the representations are isomorphic, and in this case any non-zero map is an isomorphism. Otherwise,  $\text{Hom}_G(V,W)=\{0\}$.
\end{cor}
\begin{proof}
We claim $\text{ker}(\phi)$ and $\text{im}\phi$ are both $G$-invariant.  Let $0 \neq \phi \in \text{Hom}_G(V,W)$.  If $v \in \text{ker}(\phi)$, then $\phi(v)=0$ implies that $\phi(gv) = g\phi(v) = g0 = 0$, i.e. $gv \in \text{ker}(\phi)$.  Similarly, if $v \in \text{im}(\phi)$, then $v = \phi(w)$ implies that $\phi(gw) = g \phi(w) = gv$, i.e. $gv \in \text{Im} (\phi)$.
Irreducibility yields $\text{ker}(\phi) = 0$ or $V$ and $\text{im}(\phi) = 0$ or $W$ as the only possibilities.  If $\phi \neq 0$, then $\text{ker}(\phi)=0$, $\text{im}(\phi)=W$, and $\phi$ is an isomorphism.  
Let $\psi$ be another nonzero interwining operator from $V$ to $W$.  Then $\phi ^{-1} \circ \psi \in \text{Hom}_G (V,V)$.  We can apply Schur's Lemma over $\mathbb{C}$ to see that $\phi ^{-1} \circ \psi = \lambda \text{Id}$, hence $\psi = \lambda \phi$.  So $\phi$ spans $\text{Hom}_G(V,W)$.
\end{proof}

More definitions are required before we can state a more general Schur's Lemma (not restricted to just $\mathbb{C}$).

\begin{defn}
An \textbf{algebra} over a field $K$ is a ring with unit, containing a distinguished copy of $K$ that commutes with every algebra element, and with $1 \in K$ being the algebra unit.  A \textbf{division ring} is a ring where every non-zero element is invertible, and a \textbf{division algebra} is a division ring which is also a $K$-algebra.
\end{defn}

\begin{defn}
Let $V$ be a representation of $G$ over $K$.  The \textbf{endomorphism algebra} $\text{End}_G(V)$ is the space of linear self-maps $\phi \colon V \to V$ which commute with the group action, that is, $\rho(g) \circ \phi = \phi \circ \rho(g) \quad \forall g \in G$.  The addition is the usual addition of linear maps (pointwise), and the multiplication is function composition.  The distinguished copy of $K$ is given by $K \text{Id}$.
\end{defn}
\begin{note}
You may notice that $\text{End}_G(V)$ is precisely the space $\text{Hom}_G(V,V)$ we have already seen.  However, $\text{Hom}_G(V,W)$ is in general only a vector space over the base field, not an algebra.
\end{note}

\begin{thm}\label{schur-lemma}[Schur's Lemma] If $V$ is an irreducible finite-dimensional representation of $G$ over $K$, then $\text{End}_G(V)$ is a finite-dimensional division algebra over $K$.
\end{thm}
\begin{proof}
Let $0 \neq f \in \text{End}_G(V)$.  Then $\text{ker}(f)$ and $\text{im}(f)$ are both $G$-invariant subspaces of $V$.  Since $V$ is irreducible and $f \neq 0$, we must have $\text{ker}(f) = \{ 0\}$ and $\text{im}(f) = V$, i.e. $f$ is bijective.  
\end{proof}
\begin{note}
We recover Schur's Lemma over $\mathbb{C}$ by the fact that $\mathbb{C}$ is the only finite-dimensional division algebra over $\mathbb{C}$.
\end{note}


\section{Isotypical Decomposition}

\begin{lemma}\label{lemma-uvw}
Let $U, V, W$ be three vector spaces.  Then we have natural isomorphisms
\begin{enumerate}
\item \label{lemma-uvw1} $\text{Hom} (V, U \oplus W) = \text{Hom} (V,U) \oplus \text{Hom} (V,W)$
\item \label{lemma-uvw2} $\text{Hom} (U \oplus W, V) = \text{Hom} (U,V) \oplus \text{Hom} (W,V)$.
\end{enumerate}
Additionally, if $U,V,W$ carry representations of $G$, then (\ref{lemma-uvw1}) and (\ref{lemma-uvw2}) are isomorphisms of representations.
\end{lemma}
\begin{proof}
We have inclusion and projection maps
\[
\begin{tikzcd}
U \arrow[r, hookrightarrow,  shift left,"\iota_{U}"]
& U \oplus W \arrow[r, twoheadrightarrow, shift left,"\pi_{W}"]
\arrow[l, twoheadrightarrow, shift left,"\pi_{U}"]
& W
\arrow[l, hookrightarrow, shift left, "\iota_{W}"]
\end{tikzcd}
 \]
 given by
\begin{align*}
 \iota_U &\colon x \mapsto (x , 0) \\
 \pi_U &\colon (x , y) \mapsto x
 \end{align*}
 and similarly for $\iota_W$ and $\pi_W$.  It is clear that
 \[ \text{Id}_{U \oplus W} = \iota_U \circ \pi_U + \iota_W \circ \pi_W. \]
We also note that the four spaces under consideration all have dimension $(\text{dim } V)(\text{dim } W + \text{dim } U)$.

(\ref{lemma-uvw1}) We define a map
\begin{align*}
 \psi \colon \text{Hom}(V, U \oplus W) &\to \text{Hom}(V,U) \oplus \text{Hom}(V,W) \\
 f  &\mapsto (\pi_U \circ f, \pi_W \circ f). 
\end{align*}
We claim that this map has an inverse	given by 
\begin{align*}
 \psi^{-1} \colon \text{Hom}(V,U) \oplus \text{Hom}(V,W) &\to \text{Hom}(V, U \oplus W) \\
 (f_U ,f_W) &\mapsto \iota_U \circ f_U + \iota_W \circ f_W.
\end{align*}
Check that
\begin{align*}
\psi^{-1} \circ \psi \colon f &\mapsto \iota_U \circ \pi_U \circ f + \iota_W \circ \pi_W \circ f \\
&= (\iota_U \circ \pi_U + \iota_W \circ \pi_W) \circ f\\
&= \text{Id}_{\text{Hom}(V,W)} \circ f = f.
\end{align*}
Since both vector spaces have the same dimension, $\psi \circ \psi^{-1}$ must be the identity map as well, and $\psi$ is an isomorphism of vector spaces.  Now suppose we have representations $\rho_V, \rho_W, \rho_U$ of $G$ on $V, W$ and $U$.  Then we claim $\psi$ is $G$-linear.  Recall that by definition, 
\[ \rho_{\text{Hom}(V, U \oplus W)}(g)(f) = \rho_{U \oplus W} (g) \circ f \circ \rho_V (g^{-1}).\]
Observe that for any $g \in G$ and  $f \in \text{Hom}(V, U \oplus W)$,
\begin{align*}
 \pi_U \circ ( \rho_{\text{Hom}(V, U \oplus W)}(g)(f) )&= \pi_U \circ \rho_{U \oplus W} (g) \circ f \circ \rho_V (g^{-1}) \\
 &= \rho_U (g) \circ \pi_U  \circ f \circ \rho_V (g^{-1}) \quad (\text{since } \pi_U \text{ is }G\text{-linear}) \\
 &=\rho_{\text{Hom}(U,V)}(g)(f)
 \end{align*}
 and similarly for $W$, so that 
 \begin{align*}
\psi ( \rho_{\text{Hom}(V, U \oplus W)}(g)(f)) &= (\pi_U \circ \rho_{\text{Hom}(V, U \oplus W)}(g)(f), \pi_W \circ \rho_{\text{Hom}(V,U \oplus W)}(g)(f)) \\
&=(\rho_{\text{Hom}(V,U)}(g)(\pi_U \circ f), \rho_{\text{Hom}(V,W)}(g)(\pi_W \circ f)) \\
&= \rho_{\text{Hom}(V,U) \oplus \text{Hom}(V,W)}(g)(\pi_U \circ f, \pi_W \circ f).
 \end{align*}
Thus $\psi$ is $G$-linear, and we've proved (\ref{lemma-uvw1}).

(\ref{lemma-uvw2})  Define a map
\begin{align*}
\phi \colon \text{Hom}(U \oplus W, V) &\to \text{Hom}(U,V) \oplus \text{Hom}(W,V) \\
&= (f \circ \iota_U, f \circ \iota_W).
\end{align*}
The proof is similar to (1).
\end{proof}

\begin{cor}\label{invariant-subrprns-hom-spaces}
If  $U, V, W$ are representations of $G$, then there are natural isomorphisms
\begin{enumerate}
\item $\text{Hom}_G(V, U \oplus W) = \text{Hom}_G(V,U) \oplus \text{Hom}_G(V,W)$
\item $\text{Hom}_G(U \oplus W, V) = \text{Hom}_G(U, V) \oplus \text{Hom}_G(W ,V)$
\end{enumerate}
\end{cor}
\begin{proof}
(1). By Lemma (\ref{lemma-uvw}), we have an isomorphism of representations 
\[ \psi \colon \text{Hom}(V, U \oplus W) \to \text{Hom}(V, U) \oplus \text{Hom}(V,W). \]
We can apply Lemma (\ref{isomorphism-invariant-subrprns}) to obtain an isomorphism on the invariant subrepresentations
\[\text{Hom}(V, U \oplus W)^G \cong (\text{Hom}(V, U) \oplus \text{Hom}(V,W))^G.\]
Then Lemma (\ref{invariant-subrprn-of-dir-sum}) implies that 
\[\text{Hom}(V, U \oplus W)^G \cong \text{Hom}(V, U)^G \oplus \text{Hom}(V,W)^G. \]
The statement now follows from Proposition (\ref{invariant-subrprn}).

(2). The argument is similar to the one above.
\end{proof}

\begin{prop}\label{schurs-lemma-homvw}
Let $V$ and $W$ be irreducible representations of $G$.  Then
\[ \text{dim Hom}_G (V,W) =  \begin{cases} 
1 & \mbox{if $V$ and $W$ are isomorphic}  \\
0 &\mbox{if $V$ and $W$ are not isomorphic}
\end{cases} \]
\end{prop}
\begin{proof}
Suppose $V$ and $W$ are not isomorphic.  Then Schur's Lemma Corollary \ref{schur-cor} states that the only $G$-linear map from $V$ to $W$ is the zero map, hence $\text{Hom}_G(V,W) = \{ 0 \} $.

On the other hand, suppose that $f \colon V \to W$ is an isomorphism.  Then for any $h \in \text{Hom}_G(V,W)$, we have $f^{-1} \circ h \in \text{Hom}_G(V,V)$.  By Schur's Lemma, $f^{-1} \circ h = \lambda \text{Id}_V$ for some $\lambda \in \mathbb{C}$, i.e. $h = \lambda f$.  Thus $f$  spans $\text{Hom}_G(V,W)$.
\end{proof}

\begin{prop}\label{num-iso-subrprns}
Let $\rho \colon G \to GL(V)$ be a representation, and let \[ V = U_1 \oplus \ldots \oplus U_s \] be a decomposition of $V$ into irreducibles.  Let $W$ be any irreducible representation of $G$.  Then the number of irreducible representations in the set  $ \{ U_1, \ldots, U_s \}$ which are isomorphic to $W$ is equal to the dimension of $\text{Hom}_G(V,W)$, and also equal to the dimension of $\text{Hom}_G(W,V)$.
\end{prop}
\begin{proof}
We know from Proposition (\ref{schurs-lemma-homvw}) that the number of irreducibles representations in the set  $ \{ U_1, \ldots, U_s \}$ which are isomorphic to $W$ is equal to \[ \sum_{i=1}^s \text{dim Hom}_G(U_i,W). \]
By Corollary (\ref{invariant-subrprns-hom-spaces}), \[ \text{Hom}_G(V,W) = \bigoplus_{i=1}^s \text{Hom}_G(U_i, W) \]
so that \[  \text{dim Hom}_G(V,W) = \sum_{i=1}^s \text{dim Hom}_G(U_i, W). \]
The same argument works if we consider $\text{Hom}_G(W,V)$ and $\text{Hom}_G(W,U_i)$ in place of $\text{Hom}_G(V,W)$ and $\text{Hom}_G(U_i,W)$.
\end{proof}

\begin{thm}\label{uniqueness-of-irr-decomp}
Let $\rho \colon G \to GL(V)$ be a representation, and let
\begin{align*}
V = U_1 \oplus \ldots \oplus U_s \\
V = \widetilde{U_1} \oplus \ldots \oplus \widetilde{U_r}
\end{align*}
be two decompositions of $V$ into irreducible subrepresentations.  Then $s = r$, and (after reordering if necessary) $U_i$ and $\widetilde{U_i}$ are isomorphic for every $i \in \{1, \ldots, s\}$.
\end{thm}
\begin{proof}
Let $W$ be any irreducible representation of $G$.  By Proposition (\ref{num-iso-subrprns}), the number of irreducible subprepresentations in the first decomposition that are isomorphic to $W$ is equal to $\text{dim Hom}_G(V,W)$.  On the other hand, the number of irreducible subrepresentations in the second decomposition that are isomorphic to $W$ is also equal to $\text{dim Hom}_G(V,W)$.  So for any irreducible representation $W$, the two decompositions contain the same number of factors isomorphic to $W$.
\end{proof}

\begin{defn}
Let $V$ be a finite-dimensional representation of $G$. Decompose $V$ into a direct sum
\[ V = (V_{11} \oplus \ldots \oplus V_{1 n_1}) \oplus (V_{21} \oplus \ldots V_{2 n_2}) \oplus \ldots \oplus (V_{k 1} \oplus \ldots \oplus V_{k n_k}) \] 
where 
\begin{enumerate}
\item $V_{11}, V_{21}, \ldots, V_{k1}$ are pairwise non-isomorphic irreducible representations of $G$.
\item $V_{i 1} \cong \ldots \cong V_{i n_i} \quad \forall 1 \leq i \leq k.$
\end{enumerate}
Now, for any irreducible representation $S$ of $G$, we use the decomposition of $V$ above to define
\[ V_S = \bigoplus_{ \{(i,j) | V_{ij} \cong S\}} V_{ij}.\] 
Then we have
\[ V_S \cong  \begin{cases} S^{n_i} &\text{if } S \cong V_{1i} \text{ for some } 1 \leq i \leq k \\ 0 &\text{otherwise}. \end{cases} \]
This leaves us with the canonical decomposition
\[ V = \bigoplus_{S} V_S \]
where $S$ ranges over a set of representatives of the isomorphism classes of the irreducible representations of $G$.  We call this the \textbf{canonical decomposition of $V$ into isotypical components}. If follows from Theorem \ref{uniqueness-of-irr-decomp} that such a decomposition is unique, i.e. it does not depend on our initial decomposition of $V$ into irreducibles.
\end{defn}

\section{Duals and Tensor Products}
\begin{defn}
Let $V$ be a vector space.  Recall that we define the \textbf{dual vector space} to be
\[ V^{*} = \text{Hom}(V,\mathbb{C}).\]
This is a special case of $\text{Hom}(V,W)$ where $W = \mathbb{C}$. We know that if $\{ b_1, \ldots, b_n \} $ is a basis for $V$, then there is a \textbf{dual basis} $\{ f_1, \ldots, f_n \}$ for $V$ defined by
\[ f_i (b_j) = \begin{cases} 1 &\text{if } i=j \\ 0 &\text{if } i \neq j.  \end{cases} \]
Let $\rho_V \colon G \to GL(V)$ be a representation of $G$, and let $\mathbb{C}$ be the $1$-dimensional trival representation of $G$.  Then we have seen that $V^{*}$ carries a representation of $G$ defined by
\[ \rho_{\text{Hom}(V,\mathbb{C})} (g) (f) = f \circ \rho_V ( g^{-1}) \]
We call this the \textbf{dual representation} to $\rho_V$ and denote it by $\rho_V^{*}$.
\end{defn}
\begin{prop}\label{matrix-of-dual}
If we fix a basis for $V$, then $\rho_{V^*}(g)$ is given by the matrix 
\[( \rho_V (g^{-1}) )^T \]
with respect to the dual basis.
\end{prop}
\begin{proof}
Fix a basis $\{ a_1, \ldots, a_n \}$ for $V$.  Let $\rho_V (g^{-1})$ be described by the matrix $M$, so that 
\[ \rho_V (g^{-1} ) (a_j) = \sum_{1 \leq i \leq n} M_{ij} a_i. \]
Let $\rho_V^* (g)$ be described by the matrix $N$ with respect to the dual basis $\{ \alpha_1, \ldots, \alpha_n \}$, so that
\[ \rho_V^* (g) (\alpha_j)  = \sum_{1 \leq i \leq n} N_{ij} \alpha_i. \]
Then
\begin{align*}
N_{ji} &= \sum_{1 \leq k \leq n} N_{ki} \delta_{kj} \\
	&=\sum_{1 \leq k \leq n} N_{ki}( \alpha_k a_j )\\
	&= \left( \sum_{1 \leq k \leq n} N_{ki} \alpha_k \right) a_j \\
	&= (\rho_V^* (g) (\alpha_i )) (a_j) \\
	&= (\alpha_i \circ \rho_V (g^{-1}) ) (a_j) \quad \text{(by definition of the dual representation)} \\
	&= \alpha_i ( \rho_V(g^{-1}) (a_j)) \\
	&= \alpha_i \left(  \sum_{1 \leq k \leq n} M_{kj} a_k \right) \\
	&= \sum_{1 \leq k \leq n} M_{kj} \alpha_i a_k \\
	&= \sum_{1 \leq k \leq n} M_{kj} \delta_{ik} \\
	&= M_{ij}.
\end{align*}
That is,  $N=M^T$.
\end{proof}

\begin{defn}
Suppose $V$ and $W$ are two vector spaces over a field $K$. Then we define a new vector space called the \textbf{tensor product} of $V$ and $W$, denoted by $V \otimes_{K} W$.  This space is the quotient of the free vector space on $V \times W$ (with basis given by formal symbols $v \otimes w, v\in V, w \in W$), by the subspace $D$ spanned by all elements of the form
\begin{align*}
(v_1 + v_2, w) - (v_1 , w) - (v_2 , w) \\
(v , w_1 + w_2) - (v , w_1) - (v , w_2) \\
(k \cdot v , w) -( v , k \cdot w)
\end{align*}
for $v, v_1, v_2 \in V, w, w_1, w_2 \in W$, and $k \in K$. When the ground field $K$ is clear it can be omitted from the notation.  The elements of $V \otimes W$ are called \textbf{tensors}, and the coset $v \otimes w$ of $(v,w)$ in $V \otimes W$ is called a \textbf{simple tensor}.  We have the relations
\begin{align*}
(v_1 + v_2) \otimes w &= v_1 \otimes w + v_2 \otimes w \\
v \otimes (w_1 + w_2) &= v \otimes w_1 + v \otimes w_2 \\
(k \cdot v) \otimes w &= v \otimes (k \cdot w) = k \cdot (v \otimes w).
\end{align*}
\end{defn}

\begin{defn}
Let $V$ and $W$ be vector spaces over $K$.  A map $\phi \colon V \times W \to K$ is called $\mathbf{K}$\textbf{-balanced} if \begin{align*}
\phi( v_1 + v_2, w) &= \phi(v_1, w) + \phi(v_2, w) \\
\phi(v, w_1 + w_2) &= \phi (v, w_1) + \phi(v, w_2) \\
\phi(v, kw) &= \phi( kv, w)
\end{align*}
for all $v \in V, w \in W, k \in K$.
\end{defn}
\begin{example}
Mapping $V \times W$ to the free $K$-vector space on $V \times W$, and then passing to the quotient defines a map $\iota \colon V \times W \to V \otimes W$ with $\iota (v,w) = v \otimes w$.  From the relations satisfied by the tensor product, we see that the map $\iota$ is $K$-balanced.
\end{example}

%see pf. 364 of dummit-foote
\begin{thm}\label{universal-prop-tensor}[Universal property of the tensor product]  Suppose $V,W$, and $U$ are vector spaces over the field $K$.  Let $\varphi \colon V \times W \to U$ be a $K$-balanced map, and let $\iota$ be the map above.  Then there is a unique linear map $\varphi \colon V \otimes W \to U$ such that $\varphi$ factors through $\iota$, i.e., $\varphi = \varphi \circ \iota$.
\end{thm}
\begin{proof}
The map $\varphi$ extends by linearity to a linear transformation $\widetilde{\varphi}$ from the free vector space on $V \times W$ to $U$ such that $\widetilde{\varphi}(v , w) = \varphi (v,w)$ for all $v \in V, w \in W$.  Since $\varphi$ is $K$-balanced, $\widetilde{\varphi}$ maps each of the elements which span the subspace $D$ from the definition of the tensor product to 0.  For example,
\[ \widetilde{\varphi} ((kv, w) - (v, kw)) = \varphi(kv, w ) - \varphi (v, kw) = 0. \]
Thus the kernel of $\widetilde{\varphi}$ contains $D$, and so $\widetilde{\varphi}$ induces a linear map $\varphi \colon V \otimes W \to U$.  Then
\[ \varphi( v \otimes w) = \widetilde{\varphi}( v , w) = \varphi ( v , w) \]
i.e., $\varphi = \varphi \circ \iota$. Note that $\varphi$ is completely determined by this equation since the elements $v \otimes w$ span $V \otimes W$.
\end{proof}



\begin{prop}
Let $\{ e_i \}_{i \in I}$ and $\{ f_j \}_{j \in J}$ be bases for $V$ and $W$.  Then $\{ e_i \otimes f_j | i \in I, j \in J\}$  is a basis for $V \otimes W$.
\end{prop}
\begin{proof}
An elementary tensor in $V \otimes W$ has the form $v \otimes w$.  Write $v = \sum_i a_i e_i$ and $w = \sum_j b_j f_j$, where all but finitely many of $a_i$ and $b_j$ are 0.  Then 
\[ m \otimes n = \sum_i a_i e_i \otimes \sum_j b_j f_j  = \sum_{i,j} a_i b_j e_i \otimes f_j\]
is a linear combination of the tensors $e_i \otimes f_j$.  Since every tensor can be written as a sum of elementary tensors, the elements $e_i \otimes f_j$ span $V \otimes W$.

Now, we must show that this spanning set is linearly independent.  Suppose that $\sum_{i,j} c_{ij} e_i \otimes f_j = 0$, where all but finitely many $c_{ij}$ are 0.  We want to show that $c_{ij} = 0$ for every $i \in I, j \in J$. Fix two elements $ i_0 \in I$ and $j_0 \in J$.  To show that $c_{i_0 j_0} = 0$, consider the $K$-balanced map
\begin{align*}
V \times W &\to K \\
(v,w) &\mapsto a_{i_0} b_{j_0}
\end{align*}
 where $v = \sum_i a_i e_i$ and $w = \sum_j b_j f_j$. By the universal property of tensor products, there is a linear map $f_0 \colon V \otimes W \to K$ such that $f_0 (v \otimes w) = a_{i_0} b_{j_0}$ on any elementary tensor $v \otimes w$.  In particular, $f_0 (e_{i_0} \otimes f_{j_0}) = 1$ and  $f_0 (e_{i} \otimes f_{j}) = 0$ for $(i,j) \neq (i_0, j_0)$.  Applying $f_0$ to our assumption that $\sum_{i,j} c_{ij} e_i \otimes f_j = 0$ in $V \otimes W$ tells us that $c_{i_0 j_0} =0$ in $K$.  
\end{proof}

\begin{prop}
There are natural isomorphisms 
\begin{enumerate}
\item $(U \otimes V) \otimes W \cong U \otimes (V \otimes W)$ 
\item $( U \oplus V) \otimes W \cong (U \otimes W) \oplus (V \otimes W)$.
\end{enumerate}
\end{prop}
\begin{proof}
(1.) For each fixed $w \in W$, the mapping $(u,v) \mapsto u \otimes ( v \otimes w)$ is $K$-balanced, so by Theorem \ref{universal-prop-tensor} there is a unique linear map from $U \otimes V$ to $U \otimes ( V \otimes W)$ with $u \otimes v \mapsto u \otimes (v \otimes w)$.  This shows that the map from $(U \otimes V ) \times W$ to $U \otimes ( V \otimes W)$ given by $(u \otimes v, w ) \mapsto u \otimes (v \otimes w)$ is well defined.  This map is also $K$-balanced, and  thus another application of Theorem \ref{universal-prop-tensor} shows that it induces a linear map $(U \otimes V) \otimes W \to U \otimes ( V \otimes W)$ such that $(u \otimes v) \otimes w \mapsto u \otimes (v \otimes w)$.  In a similar manner, we can construct a map $U \otimes (V \otimes W) \to (U \otimes V) \otimes W$ with $u \otimes ( v \otimes w) \mapsto (u \otimes v) \otimes w$ which is inverse to our first map.  This proves the isomorphism.

(2.)  The map $(U \oplus V ) \times W \to (U \oplus W) \otimes (V \oplus W)$ defined by $((u, v ) , w) \mapsto (u \otimes w, v \otimes w)$ is clearly $K$-balanced.  Thus it induces a linear map $f \colon (U \oplus V) \otimes W \to (U \otimes W) \oplus (V \otimes W)$ with 
\[ f ( ( u, v ) \otimes w) = (u \otimes w, v \otimes w). \]
In the other direction, we use the $K$-balanced maps $U \times W \to (U \oplus V) \otimes W$ and $V \times W \to (U \oplus V) \otimes W)$ given by $(u, w) \mapsto (u, 0)\otimes w$ and $(v,w) \mapsto (0,v) \otimes w$ to obtain linear maps from $U \otimes W$ and $V \otimes W$ to $(U \oplus V) \otimes W$.  Together these maps give a linear transformation $g$ from the direct sum $(U \otimes W) \oplus ( V \otimes W)$ to $(U \oplus V) \otimes W)$ with 
\[ g ( u \otimes w_1, v \otimes w_2) = (u, 0) \otimes w_1 + (0,v) \otimes w_2. \]
It is straightforward to see that $f$ and $g$ are inverse linear transformations, and the isomorphism holds.
\end{proof}


Now let $V$ and $W$ be two representations of $G$.
\begin{defn}
We can define a representation of $G$ on $V \otimes W$ called the \textbf{tensor product representation}. We let
\[ \rho_{V \otimes W} (g) \colon V \otimes W \to V \otimes W \]
be the linear map given by
\[ \rho_{V \otimes W} (g) \colon a_i \otimes b_j \mapsto \rho_V (g) (a_i) \otimes \rho_W (g) (b_j).\]
\end{defn}
Suppose $\rho_V (g)$ is described by the matrix $M$ and 
$\rho_W (g)$ is described by the matrix $N$ in the given bases $\{ a_1, \ldots, a_n\}$ for $V$  and $\{b_1, \ldots, b_m \}$ for W.  Then
\begin{align*}
\rho_{V \otimes W} (g) \colon a_i \otimes b_t &\mapsto \left( \sum_{j=1}^n M_{ji} a_j \right) \otimes \left( \sum_{s=1}^m N_{st} b_s \right) \\
&= \sum_{\substack{j \in [1,n] \\  s \in [1,m]}} M_{ji} N_{st} a_j \otimes b_s.
\end{align*}
So $\rho_{V \otimes W}$ is described by the $nm \times nm$ matrix $M \otimes N$ whose entries are 
\[ [ M \otimes N ]_{js, it} = M_{ji} N_{st}. \]
This matrix has $nm$ rows, and to specify a row we need a pair of numbers $(j, s)$ where $j \in \{ 1, \ldots, n \}$ and $s \in \{ 1, \ldots, m \}$.  

\begin{prop}
Let $V$ and $W$ be representations of $G$.  Then $V \otimes W$ is isomorphic to $\text{Hom}(V^{*},W)$.
\end{prop}
\begin{proof}
Let $\{ a_1,  \ldots, a_n\}$ be a basis for $V$, let $\{\alpha_1, \ldots, \alpha_n \}$ be the corresponding dual basis for $V^{*}$, and let $\{b_1,  \ldots,b_m\}$ be a basis for $W$.  Then $\text{Hom}(V^*,W)$ has a basis $\{ f_{it} | 1 \leq i \leq n, 1 \leq t \leq m \}$ where
\begin{align*}
f_{it} (\alpha_j) = \begin{cases} b_t &\text{if } j = i\\ 0 &\text{if } j \neq i \end{cases}
\end{align*}
We obtain an isomorphism of vector spaces between $\text{Hom}(V^*,W)$ and $V \otimes W$ by the map
\[ \psi (f_{it}) = a_i \otimes b_t \]
extended to all of  $\text{Hom}(V^*,W)$ by linearity.  It remains to show that this isomorphism of vector spaces yields an isomorphism of representations, i.e. we need to check that 
\[ \psi \circ \rho_{\text{Hom}(V^*,W)} (g) = \rho_{V \otimes W} (g) \circ \psi \]
for all $g \in G$.  Fix $g \in G$, and let $M$ and $N$ denote the matrices which describe $\rho_V (g)$ and $\rho_W (g)$ in the given bases.  
By definition,
\[ \rho_{\text{Hom}(V^{*},W)} (g) (f_{it}) = \rho_W (g) \circ f_{it} \circ \rho_{V^{*}}(g^{-1}). \]
Now $\rho_{V^{*}} (g^ {-1})$ is given by the matrix $M^T$ in the dual basis by Proposition \ref{matrix-of-dual}, so
\[\rho_{V^{*}} (g^ {-1}) ( \alpha_k )= \sum_{j=1}^n M_{kj} \alpha_{j}. \]
Then
\[ f_{it} \circ \rho_{V^{*}} (g^ {-1}) (\alpha_k) = M_{ki} b_t \]
which means that 
\[ \rho_W (g) \circ f_{it} \circ \rho_{V^{*}} (g^ {-1}) ( \alpha_k) = M_{ki} \left( \sum_{s=1}^m N_{st} b_s \right). \]
Thus, if we write $\rho_W (g) \circ f_{it} \circ \rho_{V^{*}} (g^ {-1})$ in terms of the basis $\{ f_{js} \}$, we have 
\[\rho_W (g) \circ f_{it} \circ \rho_{V^{*}} (g^ {-1}) = \sum_{\substack{j \in [1,n] \\  s \in [1,m]}} M_{ji} N_{st} f_{js} \]
(since both sides agree on every basis vector $\alpha_k$).  Therefore,
\begin{align*}
\psi \circ \rho_{\text{Hom}(V^*,W)} (g) (f_{it}) &= \psi \left(\rho_W (g) \circ f_{it} \circ \rho_{V^{*}} (g^ {-1}) \right) \\
&= \psi \left(\sum_{\substack{j \in [1,n] \\  s \in [1,m]}} M_{ji} N_{st} f_{js} \right) \\
&=  \sum_{\substack{j \in [1,n] \\  s \in [1,m]}} M_{ji} N_{st} a_j \otimes b_s \\
&= \rho_{V \otimes W} (g) (a_i \otimes b_t) \quad \text{(by definition of the tensor product representation)} \\
&= \rho_{V \otimes W} (g) \circ \psi (f_{it})
\end{align*}
\end{proof}


\section{Character Theory}
\begin{defn}
The \textbf{character} of a representation $\rho \colon G \to GL(V)$ is the function $\chi_V \colon G \to \mathbb{C}$ defined by $\chi_V(g) = \text{Tr}(\rho(g))$.
\end{defn}
\begin{note}
The character is of a representation is not a homorphism in general, since $\text{Tr}(MN) \neq \text{Tr}(M) \text{Tr}(N)$ in general.
\end{note}

\begin{prop}\label{basic-props-of-char} (Basic Properties)
\begin{enumerate}
\item $\chi_V$ is conjugation invariant: $\chi_V (h g h^{-1}) = \chi_V (g)$ for all $g , h \in G$.
\item $\chi_V (1) = \text{\emph{dim }} V$.
\item \label{char-of-inverse} $\chi_V (g^{-1}) = \overline{\chi_V (g)}$ for all $g \in G$.
\item $\chi_{V^*} (g) =  \overline{\chi_V (g)}$ for all $g \in G$.
\end{enumerate}
\end{prop}

\begin{proof}
\begin{enumerate}
\item $\chi_V (h g h^{-1} = \text{Tr}(h g h^{-1}) = \text{Tr}(g h h^{-1}) = \text{Tr} (g) = \chi_V(g)$ for any $g,h \in G$.
\item $\chi_V(1) = \text{Tr}(\text{Id} _V) = \text{dim } V$.
\item Since $G$ is finite, we have seen that $\rho(g)$ is a diagonal matrix with roots of unity along the diagonal with the right choice of basis.  The inverse of a root of unity is given by its complex conjugate, so under this same basis, $\rho(g)^{-1}$ is clearly given by $\overline{\rho(g)}$.  Thus, $\chi_V(g^{-1}) = \text{Tr}(\rho(g^{-1})) = \text{Tr}(\rho(g)^{-1}) = \text{Tr}(\overline{\rho(g)}) = \overline{ \text{Tr} (\rho(g))} = \overline {\chi_V(g)}$.
\item \begin{align*}
 \chi_{V^*}(g) &= \text{Tr}(\rho_{V^*}(g)) \\
 &= \text{Tr}(\rho_V (g^{-1})^T) \quad \text{ (by Proposition \ref{matrix-of-dual})} \\
 &= \text{Tr}(\rho_V(g^{-1})) \\
 &= \overline{\chi_V (g)}  \quad \text{ (by \ref{char-of-inverse})}
\end{align*}
\end{enumerate}
\end{proof}

\begin{prop}\label{iso-reprns-same-char}
Isomorphic representations have the same character.
\end{prop}
\begin{proof}
We have seen in Proposition \ref{iso-classes-of-reprns} that isomorphic representations can be described by the same set of matrices with the right choice of bases.  Thus they have the same trace.
\end{proof}
We will see later that the converse is true - if two representations have the same character, then they are isomorphic.

\begin{prop}
Let $\rho_V \colon G \to GL(V)$ and $\rho_W \colon G \to GL(W)$ be representations of $G$ with characters $\chi_V$ and $\chi_W$.
\begin{enumerate}
\item $\chi _{V \oplus W} = \chi_V + \chi_W$.
\item $\chi_{V \otimes W} = \chi_V \cdot \chi_W$.
\end{enumerate}
\end{prop}
\begin{proof}
Pick bases for $V$ and $W$, so that $\rho_V (g)$ and $\rho_W (g)$ are described by matrices $M$ and $N$.  
\begin{enumerate}
\item $\rho_{V \oplus W} (g)$ is described by the block-diagonal matrix
\[ \begin{bmatrix}
M & 0 \\
0 & N \\
\end{bmatrix}\]
So we have $\text{Tr} (\rho_{V \oplus W} (g)) = \text{Tr} (M) + \text{Tr}(N) = \text{Tr}(\rho_V (g)) + \text{Tr} (\rho_W (g))$.

\item $\rho_ {V \otimes W}(g)$ is given by the matrix
\[ [M \otimes N ]_{js, it} = M_{ji} N_{st} \]
so
\begin{align*}
\text{Tr} (M \otimes N) &= \sum_{i,t} [M \otimes N]_{js,it} \\
	&= \sum_{i,t} (M_{ii} N_{tt}) \\
	&= \text{Tr} (M) \text{Tr} (N).
\end{align*}
Thus $\chi_{V \otimes W} (g)= \chi_V (g) \chi_W(g)$.
\end{enumerate}
\end{proof}

\begin{defn}
Let $\mathbb{C}^G$ denote the set of all functions from $G$ to $\mathbb{C}$.  Then $\mathbb{C}^G$ is a vector space with the sum of two functions defined pointwise and with scalar multiplication defined for $f \in \mathbb{C}^G, \lambda \in \mathbb{C}$ by 
\begin{align*}
\lambda f \colon G &\to \mathbb{C} \\
g &\mapsto \lambda f(g).
\end{align*}
A basis for $\mathbb{C}^G$ is clearly given by the set of functions 
\[\{ \delta_g | g \in G  \} \]
defined by 
\[ \delta_g \colon h \mapsto \begin{cases}  1 &\text{if } h = g \\
 0 &\text{if } h \neq g
\end{cases} \]
\end{defn}

\begin{defn}
Let $\varphi, \psi \in \mathbb{C}^G$.  We define an \textbf{inner product}  on $\mathbb{C}^G$ by 
\[ \langle \varphi | \psi \rangle = \frac{1}{|G|} \sum_{g \in G} \varphi(g) \overline{\psi(g)}.\]
\end{defn}
It is easy to see that $\langle \varphi | \psi \rangle$ is linear in the first variable, conjugate-linear in the second variable (i.e., $\langle \varphi | \lambda \psi \rangle = \overline{\lambda}	\langle \varphi | \psi \rangle$) , and that $\langle \varphi | \psi \rangle = \overline {\langle \psi | \varphi \rangle}$.  These three properties are the definition of a Hermitian inner product.  Note that our basis elements $\delta_g$ are orthogonal with respect to this inner product, but not orthonormal since 
\[ \langle \delta_g | \delta_g \rangle = \begin{cases} \frac{1}{|G|} &\mbox{if } h = g \\ 0 &\mbox{if } h \neq g \end{cases}. \]
The characters of $G$ are elements of $\mathbb{C}^G$, so we can evaluate this inner product on pairs of characters.  The answer turns out to be very useful, but before we can begin the proof we require two quick lemmas:

\begin{lemma}
Let $\rho \colon G \to GL(V)$ be any representation.  Consider the linear map 
\begin{align*}
\Psi \colon V &\to V \\
x &\mapsto \frac{1}{|G|} \sum_{g \in G} \rho(g)(x).
\end{align*}
Then $\Psi$ is a  projection from $V$ onto the invariant subspace $V^G$.
\end{lemma}
\begin{proof}
We need to check that $\Psi(x) \in V^G$ for all $x \in V$.  For any $h \in G$, 
\begin{align*}
\rho(h)(\Psi(x)) &= \frac{1}{|G|}  \sum_{g \in G} \rho(h)\rho(g)(x) \\
&= \frac{1}{|G|}\sum_{g \in G} \rho(h g) (x) \\
&= \frac{1}{|G|} \sum_{g \in G} \rho(g)(x) \quad (\text{by relabelling } g \mapsto h^{-1}g) \\
&= \Psi(x).
\end{align*}
Thus $\Psi$ is a linear map $V \to V^G$.  Finally we need to check that $\Psi \restriction_{V^G} = \text{Id}_{V^G}$.  Let $x \in V^G$.  Then
\begin{align*}
\Psi(x) &= \frac{1}{|G|} \sum_{g \in G} \rho(g)(x) \\
&= \frac{1}{|G|} \sum_{g \in G} (x) \\
&= \frac{|G|}{|G|} x= x.
\end{align*}
\end{proof}

\begin{lemma} \label{projection-lemma}
Let $V$ be a vector space with subspace $U \subset V$, and let $\pi \colon V \to V$ be a projection onto $U$.  Then 
\[ \text{Tr}(\pi) = \text{dim } U. \]
\end{lemma}
\begin{proof}
Recall that $V = U \oplus \text{Ker }(\pi)$ from Lemma \ref{maschke-lemma}. If we fix bases for $U$ and $\text{Ker }(\pi)$, which together give a basis for V, then $\pi$ is given by the block-diagonal matrix
\[   \begin{bmatrix}
\mathbf{1} & 0 \\
0 & \mathbf{0} \\
\end{bmatrix} \]
where $\text{dim U}$ is the size of the upper left block and $\text{dim Ker }(\pi)$ is the size of the bottom right block.  So $\text{Tr}(\pi) = \text{Tr}(\mathbf{1}_U) = \text{dim } U$.
\end{proof}

\begin{thm}
Let $\rho_V \colon G \to GL(V)$ and $\rho_W \colon G \to GL(W)$ be representations of $G$, and let $\chi_V, \chi_W$ be their characters.  Then 
\[ \langle \chi_W | \chi_V \rangle = \text{dim Hom}_G (V,W). \]
\end{thm}
In particular, the inner product of two characters is always a non-negative integer.  (Whereas in general, the inner product of two arbitrary functions can be any complex number.)

\begin{proof}
We have seen in Proposition \ref{invariant-subrprn} that
 \[ \text{Hom}_G (V,W) \subset \text{Hom}(V,W) \] 
 as the invariant subrepresentation, and by the previous lemma we have a projection
\begin{align*}
\Psi \colon \text{Hom}(V,W) &\to \text{Hom}(V,W) \\
f &\mapsto \frac{1}{|G|} \sum_{g \in G} \rho_{\text{Hom}(V,W)} (g) (f).
\end{align*}
We claim that \[ \text{Tr} (\Psi) = \langle \chi_W | \chi_V \rangle. \]  Once this claim is established, then Lemma \ref{projection-lemma} will prove the theorem.  We proceed by calculating $\text{Tr} (\Psi)$.  Fix bases $\{ a_1, \ldots, a_n \}$ for $V$ and $\{ b_1, \ldots, b_m \}$ for $W$.  Then $\text{Hom}(V,W)$ has an associated basis 
\[ \{ f_{ji} | 1 \leq i \leq n, 1 \leq j \leq m\} \]
where
\[ f_{ji} (a_i) = \begin{cases} b_j &\text{if } i=j \\ 0 &\text{if } i \neq j.  \end{cases} \] 
We may calculate $\text{Tr}(\Psi)$ as follows:  For each $i,j$, compute the expression of $\Psi(f_{ji})$ in this basis, and take the coefficient of the basis element $f_{ji}$.  This is a diagonal entry in the matrix for $\Psi$. Summing these values over all $i$ and $j$ will gives us $\text{Tr}(\Psi)$.

Let $\widetilde{\rho_V}, \widetilde{\rho_W}$ be the matrix representations obtained by writing $\rho_V$ and $\rho_W$ in the given bases.  We know that 
\[ \text{Hom}(V,W) = V^* \otimes W \]
so if we write $\rho_{\text{Hom}(V,W)}$ in the basis $\{ f_{ji} \}$ then we get the tensor product of  $\widetilde{\rho_{V^*}}$ and $\widetilde{\rho_W}$.  Thus
\begin{align*}
\rho_{\text{Hom}(V,W)}(g)(f_{ji}) &= \rho_W (g) \circ f_{ji} \circ \rho_V (g^{-1}) \\
&= \sum_{\substack{k \in [1,n] \\  t \in [1,m]}} \widetilde{\rho_V} (g^{-1})_{ik} \widetilde{\rho_W}(g)_{tj} f_{kt}. \quad \text{Show another step?}
\end{align*}
Then 
\begin{align*}
\Psi (f_{ji}) &= \frac{1}{|G|} \sum_{g \in G} \rho_{\text{Hom}(V,W)}(g)(f) \\
&= \frac{1}{|G|} \sum_{g \in G} \sum_{\substack{k \in [1,n] \\  t \in [1,m]}} \widetilde{\rho_V} (g^{-1})_{ik} \widetilde{\rho_W}(g)_{tj} f_{kt}.
\end{align*}
The coefficient of $f_{ji}$ in this expression is
\[ \frac{1}{|G|} \sum_{g \in G}\widetilde{\rho_V} (g^{-1})_{ii} \widetilde{\rho_W}(g)_{jj}. \]
Therefore
\begin{align*}
\text{Tr} (\Psi) &= \sum_{\substack{k \in [1,n] \\  t \in [1,m]}}  \frac{1}{|G|} \sum_{g \in G}\widetilde{\rho_V} (g^{-1})_{ii} \widetilde{\rho_W}(g)_{jj} \\
&=  \frac{1}{|G|} \sum_{g \in G} \left( \sum_{i=1}^n \widetilde{\rho_V}(g^{-1})_{ii} \right) \left( \sum_{j=1}^m \widetilde{\rho_W} (g)_{jj} \right) \\
&= \frac{1}{|G|} \sum_{g \in G} \chi_V (g^{-1}) \chi_W (g) \\
&= \frac{1}{|G|} \sum_{g \in G} \chi_W (g) \overline{\chi_V}(g)  \quad \text{(by Proposition \ref{basic-props-of-char}.\ref{char-of-inverse})}\\
&= \langle \chi_W | \chi_V \rangle.
\end{align*}
\end{proof}

\begin{cor}[Row orthogonality relations]\label{inner-product-of-irr-chars}
Let $\chi_1, \ldots, \chi_r$ be characters of pairwise non-isomorphic irreducible representations of $G$.  Then
\[ \langle \chi_i | \chi_j \rangle = \begin{cases}  1 &\text{if } i = j \\ 0 &\text{if } i \neq j\end{cases} \]
\end{cor}
\begin{proof}
Let $\chi_i$ and $\chi_j$ be the characters of the irreducible representations $U_i, U_j$.  Then by Proposition \ref{schurs-lemma-homvw},

\[ \langle \chi_i | \chi_j \rangle = \text{dim Hom}_G (U_i, U_j) = \begin{cases}  1 &\text{if }U_i, U_j \text{ are isomorphic} \\  0 &\text{if }U_i, U_j \text{ are not isomorphic}. \end{cases} \]
\end{proof}

\begin{cor}
Let $\chi$ be any character of $G$.    Then $\chi$ is irreducible if and only if \[ \langle \chi | \chi \rangle = 1\]
\end{cor}
\begin{proof}
Write $\chi$ as a linear combination of irreducible characters \[ \chi = m_1 \chi_1 + \ldots + m_k \chi_k \] where each $m_i$ is a non-negative integer.  Then by Corollary \ref{inner-product-of-irr-chars}, 
\begin{align*}
\langle \chi | \chi \rangle &= \sum_{i,j \in [1, k]} m_i m_j \langle \chi_i | \chi_j \rangle \\
&= m_1^2 + \ldots + m_k^2.
\end{align*}
So $\langle \chi | \chi \rangle = 1$ if and only if exactly one of the $m_i = 1$ and the rest are $0$.
\end{proof}

\begin{example}\label{irr-char-of-d8}
Let $G=D_4=  \langle \sigma, \tau |  \sigma^4 = \tau^2 = e, \tau \sigma \tau^{-1} = \sigma^{-1} \rangle$.  Recall the two dimensional representation $W$ of $D_4$ given in Example \ref {rep-of-d8}.
We compute the character of this representation by taking the trace of the matrices from that example:
\begin{align*}
\chi_W(e) &=2 & \chi_W(\tau) = 0 \\
\chi_W(\sigma) &= 0 &\chi_W (\sigma \tau) = 0 \\
\chi_W(\sigma^2) &= -2 &\chi_W (\sigma^2 \tau) = 0 \\
\chi_W(\sigma^3) &= 0 &\chi_W (\sigma^3 \tau) = 0.
\end{align*}
Then
\[ \langle \chi_W | \chi_W \rangle =\frac {1}{|G} \sum_{g \in G} \chi_W (g) \overline{\chi_W (g)} = \frac{1}{8} (4 + 4) = 1\]
so we conclude that $W$ is irreducible.
\end{example}


\begin{cor}
Let $\rho_V \colon G \to GL(V)$ and $\rho_W \colon G \to GL(W)$ be representations of $G$.  Then $V$ and $W$ are isomorphic if and only if $\chi_V = \chi_W$.
\end{cor}
\begin{proof}
We have already seen that isomorphic representations have the same character by Proposition \label{iso-reprns-same-char}.  On the other hand, suppose $\chi_V = \chi_W$.  Let $U_1, \ldots, U_r$ be the irreducible representations of $G$, and let $\chi_1, \ldots, \chi_r$ be their characters.  We can write
\[ V = U_1^{m_1} \oplus \ldots \oplus U_r^{m_r} \]
for some non-negative integers $m_1, \ldots, m_r$, and 
\[ W = U_1^{l_1} \oplus \ldots \oplus U_r^{l_r} \]
for some non-negative integers $l_1, \ldots, l_r$.  So
\[ \chi_V = m_1 \chi_1 + \ldots + m_r \chi_r \]
and 
\[ \chi_W = l_1 \chi_1 + \ldots + l_r \chi_r .\]
Thus we have 
\[ m_i = \langle \chi_V | \chi_i \rangle = \langle \chi_W | \chi_i \rangle = l_i \]
for all $i \in \{1, \ldots, r \}$ since $\chi_V = \chi_W$.  This proves $V$ and $W$ are isomorphic.
\end{proof}

\begin{lemma}
$\chi_{\text{reg}}(g) = \begin{cases} |G| &\text{if } g=1 \\ 0 &\text{if } g \neq 1 \end{cases}$
\end{lemma}
\begin{proof}
For each $g \in G$, we have $\rho_{\text{reg}} (g) (e_h) = e_{gh}$.  So $\rho_{\text{reg}}$ maps any basis element to another basis element.  
Thus \[ \text{Tr} (\rho_{\text{reg}} (g)) = | \{ g \in G | h=gh \} | = \begin{cases} |G| &\text{if } g=1 \\ 0 &\text{if } g \neq 1 \end{cases}. \]
\end{proof}

\begin{prop} \label{mult-of-irr-in-reg}
The multiplicity of any irreducible representation in the regular representation equals its dimension.
\end{prop}
\begin{proof}
Let $V$ be an irreducible representation of $G$.  Then
\begin{align*}
\langle \chi_{\text{reg}}, \chi_V \rangle &= \frac{1}{|G|} \chi_{\text{reg}}(1) \overline{\chi_V (1)} \\
&= \frac{1}{|G|} |G| (\text{dim } V) \\
&= \text{dim } V.
\end{align*}
Therefore the multiplicity of $V$ in the regular representation is $\text{dim } V$.
\end{proof}

\begin{cor}
There are finitely many irreducible representations of $G$, up to isomorphism.
\end{cor}
\begin{proof}
Since $\text{dim } FG = |G|$, there can be at most $|G|$ irreducible representations of $G$ up to isomorphism.
\end{proof}


\begin{cor}
Let $\chi_1, \ldots, \chi_r$ be the irreducible characters of $G$ with degrees $d_1, \ldots, d_r$.  Then
\[ |G| = \sum_{i=1}^n d_i^2 \]
\end{cor}
\begin{proof}
Let $U_1, \ldots, U_r$ be distinct representatives of the isomorphism classes of irreducible representations of $G$.  Then
\[ V_{\text{reg}} = U_1 ^ {d_1} \oplus \ldots \oplus U_r ^{d_r} \]
by Proposition \ref{mult-of-irr-in-reg}.  The claim follows from taking dimensions of each side of this equation.
\end{proof}


\begin{defn}
A \textbf{class function} on $G$ is a function on $G$ whose values are invariant by conjugation of elements in $G$.  The value of a class function at an element $g \in G$ depends only on the conjugacy class of $g$.  We may therefore view class functions as functions on the set of conjugacy classes of $G$.
\end{defn}
\begin{note}
The character $\chi_V$ of a representation $V$ of $G$ is a class function on $G$.
\end{note}
Since the character of a representation is a class function on $G$, we don't actually need to to compute $\chi_V (g) \overline {\chi_W (g)}$ for every group element to find the inner product of $\chi_V$ and $\chi_W$.  We just need to calculate it once on each conjugacy class, i.e. 
\begin{align*}
\langle \chi_V | \chi_W \rangle &= \frac{1}{|G|} \sum_{g \in G} \rho_V (g) \overline{\rho_W (g)} \\
&=  \frac{1}{|G|} \sum_{[g]} |[g]|   \rho_V (g) \overline{\rho_W (g)}
\end{align*}
where the latter sum ranges over the conjugacy classes $[g]$ of $G$.

\begin{defn}
We define \textbf{the character table of } $G$ to be the table of complex numbers whose:
\begin{itemize}\item rows are index by the isomorphism classes of irreducible representations of $G$, 
\item columns are indexed by the conjugacy classes of $G$,
\item $i,j$ entry is given by value of the character corresponding to row $i$ evaluated at the isomorphism class corresponding to column $j$.
\end{itemize}
\end{defn}
\begin{note}
It is also helpful to include the size of the conjugacy class associated with each column of the character table, so that we can easily compute the inner product of two characters by looking at the character table.
\end{note}

\begin{example}
Consider $G=D_3 = \langle \sigma, \tau | \sigma^3 = \tau^2 = e, \tau \sigma \tau^{-1} = \sigma^{-1} \rangle$.  We have seen three irreducible representations of $D_3$, namely the $1$-dimensional trivial representation, the $1$-dimensional alternating representation, and the $2$-dimensional irreducible representation $W$ constructed geometrically in Example \ref{2d-irrep-d3}.  Observe that
\[ |D_3| = 6 = 1^2 + 1^2 +2^2 \]
so these are all of the irreducible representations of $D_3$ up to isomorphism.  The conjugacy classes of $D_3$ are 
$\{e\}$, $\{ \sigma, \sigma ^2 \}$, and $\{\tau, \tau \sigma, \tau \sigma^2 \}$.  Thus, the character table of $D_3$ is given by 

\begin{tabular}{ | l | c | c | c |}\hline 
\multicolumn{4}{|c|}{Character table of $D_3$} \\ \hline
Size of conjugacy class $\vert [ g] \vert$ & 1 & 3 & 2 \\ \hline
\multicolumn{4}{|c|}{} \\ \hline

Conjugacy class representative $[g]$ & $[e]$ & $[\tau]$ & $[\sigma]$ \\ \hline
$\chi_1$  ($1$-d trivial reprn) & $1$ & 1 & 1 \\ \hline
$\chi_{\text{sgn}}$  ($1$-d sign reprn)  & 1 & -1 & 1 \\ \hline
$\chi_W$ ($2$-d reprn obtained geometrically) & 2 & 0 & -1 \\
\hline
\end{tabular}
\end{example}


\begin{example}
Let $G = D_4$.  Let $U_1, \ldots, U_r$ be the irreducible representations of $D_4$, with dimensions $d_1, \ldots, d_r$ respectively, and let $U_1$ be the $1$-dimensional trivial representation.  Then 
\[ d_2^2 + \ldots + d_r ^2 = |G| - d_1^2 = 8 - 1 = 7. \]
There are two possibilities:  

1.  $r=8$, and $d_i = 1$ for all $ 1 \leq i \leq 8$. 

2. or $r=5$, and $d_2 = d_3 = d_4 = 1$, $d_5 = 2$.

We saw in Example \ref{irr-char-of-d8} that $G$ has a two-dimensional irreducible representation, so in fact (2) holds. If we look at the relations in the presentation of $D_4=\langle \sigma, \tau |  \sigma^4 = \tau^2 = e, \tau \sigma \tau^{-1} = \sigma^{-1} \rangle$, we may construct four one-dimensional representations $\rho_1, \ldots, \rho_4$ by specifying the images of $\sigma$ and $\tau$:
 \begin{align*}
 \rho_{1 + i + 2j} \colon  \tau &\mapsto (-1)^i \\
	\sigma &\mapsto (-1)^j
 \end{align*}
where $i,j \in \{0,1\}$.  Thus the character table for $D_4$ is as follows:

\begin{tabular}{ | l | c | c | c |c | c |}\hline 
\multicolumn{6}{|c|}{Character table of $D_4$} \\ \hline
Conjugacy class  & $\{1\}$ & $ \{\sigma, \sigma^3\}$ & $\{\sigma^2\}$  & $\{\tau, \sigma^2 \tau\}$ & $\{\sigma\tau, \sigma^3 \tau\}$ \\ \hline
$\chi_1$ & $1$ & 1 & 1 & 1 & 1\\ \hline
$\chi_2$ & 1 & 1 & 1 & -1 & -1\\ \hline
$\chi_3$  & 1 & -1 & 1  & 1 & -1\\ \hline
$\chi_4$   & 1 & -1 & 1 & -1 & 1 \\ \hline
$\chi_W$ ($2$-d reprn obtained geometrically) & 2 & 0 & -2  & 0 & 0 \\
\hline
\end{tabular}
\end{example}

