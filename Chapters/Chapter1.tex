% Chapter Template

\chapter{Basics of Representation Theory} % Main chapter title

%\label{Chapter1} % Change X to a consecutive number; for referencing this chapter elsewhere, use \ref{ChapterX}



%----------------------------------------------------------------------------------------
%	SECTION 1
%----------------------------------------------------------------------------------------

%----------------------------------------------------------------------------------------
%	Group Actions
%
\section {Group Actions}
%----------------------------------------------------------------------------------------
\begin{defn}\label{def-grp-action}
A  \textbf{\textit{(left)} group action} of a group $G$ on a set $X$ is a map $\varphi : G \times X \to X$ (written as $g \cdot a$, for all $g \in G$ and $a \in A$) that satisfies the following two axoims:
\begin{align}
\label{grp-action-axiom-1}&id_G \cdot  x = x && \forall x \in X\\
\label{grp-action-axiom-2}&(gh) \cdot x  = g \cdot (h \cdot x) && \forall g,h \in G, x \in X
\end{align}
\end{defn}
\begin{note}
We could likewise define the concept of a \textit{right} group action, where the set elements would be multiplied by group elements on the right instead of on the left.  Throughout we shall use the term \textit{group action} to mean a \textit{left} group action.
\end{note}


\begin{prop}\label{sigma-is-a-permutation}
Let $G$ act on the set $X$.  For any fixed $g \in G$, the map $\sigma_g$ from $X$ into $X$ defined by $\sigma_g (x) = g \cdot x$ is a \textit{permutation} of the set $X$, i.e. $\sigma_g \in S_X$.
\end{prop}
\begin{proof}
We show that $\sigma_g$ is a permutation of $X$ by finding a two-sided inverse map, namely $\sigma_{g^{-1}}$. Observe that for any $x \in X$, we have
\begin{align*}
(\sigma_{g^{-1}} \circ \sigma_g) (x) &= \sigma_{g^{-1}} ( \sigma_g (x) && \text{(by definition of function composition)} \\
					&= g^{-1} \cdot (g \cdot x) && \text{(by definition of $\sigma_g$ and $\sigma_{g^{-1}}$)} \\
					&= (g^{-1}g) \cdot x &&\text{(by axiom \ref{grp-action-axiom-1} of an action)} \\
					&= id_G \cdot x \\
					&= x &&\text{(by axiom \ref{grp-action-axiom-2} of an action)}.
\end{align*}
Thus $\sigma_{g^{-1}} \circ \sigma_g$ is the identity map on $X$. We can reverse the roles of $g$ and $g^{-1}$ to see that $\sigma_g \circ \sigma_{g^{-1}}$ is also the identity map on $X$.  Having a two-sided inverse, we conslude that $\sigma_g$ is a permutation of $X$.
\end{proof}

\begin{prop}\label{action-yields-hom}
Let $G$ act on the set $X$. The map from $G$ to the symmetric group $S_X$ defined by $g \mapsto \sigma_g (x) = g \cdot x$ is a group homomorphism.
\end{prop}
\begin{proof}
Define the map $\varphi : G \to S_X$ by $\varphi (g) = \sigma_g$.  We have seen from Proposition \ref{sigma-is-a-permutation} that $\sigma_g$ is indeed an element of $S_X$.  It remains to show that $\varphi(g_1 g_2) = \varphi(g_1) \circ \varphi(g_2)$ for any $g_1, g_2 \in G$.  Observe that

\begin{align*}
\varphi(g_1 g_2)(x) &= \sigma_{g_1 g_2} (x) && \text{(by definition of $\varphi$)} \\
			&= (g_1 g_2) \cdot x && \text{(by definition of $\sigma_{g_1 g_2}$)} \\
			&= g_1 \cdot (g_2 \cdot x) && \text{(by axiom \ref{grp-action-axiom-1} of an action)} \\
			&= \sigma_{g_1} ( \sigma_{g_2} (x)) && \text{(by definition of $\sigma_{g_1}$ and  $\sigma{g_2}$)} \\
			&= \varphi(g_1) ( \varphi(g_2) (x)) && \text{(by definition of $\varphi$)}\\
			&= (\varphi(g_1) \circ \varphi(g_2)) (x) && \text{(by definition of function composition)}.
\end{align*}
Since the values of $\varphi(g_1 g_2)$ and $\varphi(g_1) \circ \varphi(g_2)$ agree on every element $x \in X$, these two permutations are equal. We conclude that $\varphi$ is a homomorphism, since $g_1$ and $g_2$ were arbitrary elements of $G$.
\end{proof}


\begin{prop} \label{hom-yields-action}
Any homomorphism $\psi$ from the group $G$ into the symmetric group on $S_X$ on a set $X$ gives rise to an action of $G$ on $X$, defined by taking $g \cdot x = \psi(g)(x)$.
\end{prop}
\begin{proof}
Suppose  that we have a homomorphism $\psi$ from $G$ into $S_X$.  We can define a map from $G \times X$ to $X$  by $g \cdot x = \psi(g)(x)$. We verify that this map satisfies the definition of a group action of $G$ on $X$:
\\ (axiom \ref{grp-action-axiom-1}) \quad $id_G \cdot x = \psi(id_G)(x) = id_X(x) = x$
\\(axiom \ref{grp-action-axiom-2}) \quad $(gh) \cdot x = \psi(gh)(x) = (\psi(g)\psi(h))(x) = \psi(g)(\psi(h)(x)) = g \cdot (h \cdot x)$
\end{proof}

\begin{prop} \label{equivalence-of-actions}
The actions of $G$ on the set $X$ are in bijective correspondence with the homomorphisms from $G$ into the symmetric group $S_X$.
\end{prop}
\begin{proof}
By Proposition \ref{action-yields-hom}, any action of $G$ on $X$ yields a homomorphism from $G$ into $S_X$.    Conversely, any homomorphism from $G$ into $S_X$ establishes an action of $G$ on $X$ by Proposition \ref{hom-yields-action}.
\end{proof}

%Todo: add examples?

%----------------------------------------------------------------------------------------
%	Definition of a Representation
%
\section{Definition of a Representation}
%----------------------------------------------------------------------------------------

\begin{defn}
\label{rep-def-1}
A \textbf{linear representation} of a group $G$ on a vector space $V$  is a group homomorphism from $G$ to $GL(V)$, the general linear group on V. \end{defn}
 
 %More explicitly, a representation is a map $\rho : G \rightarrow GL(V)$ such that \[ \rho (g_1 g_2) = \rho(g_1) \rho(g_2) \quad \forall g_1, g_2 \in G. \]

\begin{defn}\label{rep-def-2}A \textbf{linear representation} of a group $G$ on a vector space $V$ over a field $F$ is an action of $G$ on $V$ which preserves the linear structure of $V$, that is, 
\begin{align}
\label{rep-axiom-1}&g \cdot (v_1+v_2)=g \cdot v_1+g \cdot v_2 \quad && \forall g \in G, v_1, v_2 \in V \\
\label{rep-axiom-2}&g \cdot (kv) = k (g \cdot v) \quad && \forall g \in G, v \in V, k \in F
\end{align}
 \end{defn}
 
 \begin{prop}
The definitions of a linear representation given in \ref{rep-def-1} and \ref{rep-def-2} above are equivalent.
 \end{prop}
 \begin{proof}
%\leavevmode
 \begin{itemize}
\item[$(\rightarrow)$]  Suppose that we have a homomorphism $\varphi : G \to GL(V)$.  Note that $GL(V)$ is a subgroup of the symmetric group $S_V$ on $V$, so we can apply Proposition \ref{hom-yields-action} to obtain an action of $G$ on $V$ by $g \cdot v = \varphi(g)(v)$.  We check that this action preserves the linear structure of V.
\\\ref{rep-axiom-1} \quad For any $g \in G$, $v_1, v_2 \in V$ we have $g \cdot (v_1 +  v_2) = \varphi(g) (v_1 + v_2) = \varphi(g)(v_1) + \varphi(g)(v_2)= g \cdot v_1 + g \cdot v_2$.
\\\ref{rep-axiom-2} \quad For any $g \in G, v \in V, k \in F$ we have $g \cdot (kv) = \varphi(g)(kv) = k (\varphi(g)(v)) = k (g \cdot v)$.
\item[$(\leftarrow)$] Suppose that we have an action of $G$ on $V$ which preserves the linear structure of V in the sense of Definition \ref{rep-def-2}.  We can apply Proposition \ref{action-yields-hom} to obtain a homorphism $\varphi : G \to S_V$ given by $\varphi(g) = \sigma_g$ where $\sigma_g(v) = g \cdot v $.  It remains to show that the image $\varphi(G)$ of $G$ under $\varphi$ is actually contained in $GL(V)$, i.e. that for each $g \in G$ the map $\sigma_g$ is linear.  Fix an element $g \in G$. For any $k \in F$ and $v \in V$ we have
\begin{align*}
\sigma_g (kv) &= g \cdot (kv) && \text{(by definition of $\sigma_g$)} \\
		&= k (g \cdot v) && \text{(by property \ref{rep-axiom-1})} \\
		&= k (\sigma_g (v)) && \text{(by definition of $\sigma_g$)}.
\end{align*}
Also, for any $v_1, v_2 \in V$ we have
\begin{align*}
\sigma_g (v_1 + v_2) &= g \cdot (v_1 + v_2) && \text{(by definition of $\sigma_g$)} \\
		&= g \cdot v_1 + g \cdot v_2 && \text{(by property \ref{rep-axiom-2})} \\
		&= \sigma_g(v_1) + \sigma_g(v_2) && \text{(by definition of $\sigma_g$)}.
\end{align*}
Thus $\sigma_g$ is linear, and $\varphi(G) \subset GL(V)$ proves that we  have a homomorphism $\varphi : G \to GL(V)$.

\end{itemize}
 \end{proof}