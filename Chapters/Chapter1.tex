% Chapter Template

\chapter{Basic Notions of Representation Theory} % Main chapter title

%\label{Chapter1} % Change X to a consecutive number; for referencing this chapter elsewhere, use \ref{ChapterX}



%----------------------------------------------------------------------------------------
%	SECTION 1
%----------------------------------------------------------------------------------------
%TODO: Add introduction/historical background 


%----------------------------------------------------------------------------------------
%	Group Actions
%
\section {Group Actions}
%----------------------------------------------------------------------------------------
\begin{defn}\label{def-grp-action}
A  \textbf{\textit{(left)} group action} of a group $G$ on a set $X$ is a map $\varphi \colon G \times X \to X$ (written as $g \cdot a$, for all $g \in G$ and $a \in A$) that satisfies the following two axoims:
\begin{align}
\label{grp-action-axiom-1}&1 \cdot  x = x && \forall x \in X\\
\label{grp-action-axiom-2}&(gh) \cdot x  = g \cdot (h \cdot x) && \forall g,h \in G, x \in X
\end{align}
\end{defn}
\begin{note}
We could likewise define the concept of a \textit{right} group action, where the set elements would be multiplied by group elements on the right instead of on the left.  Throughout we shall use the term \textit{group action} to mean a \textit{left} group action.
\end{note}


\begin{prop}\label{sigma-is-a-permutation}
Let $G$ act on the set $X$.  For any fixed $g \in G$, the map $\sigma_g$ from $X$ into $X$ defined by $\sigma_g (x) = g \cdot x$ is a \textit{permutation} of the set $X$, i.e. $\sigma_g \in S_X$.
\end{prop}
\begin{proof}
We show that $\sigma_g$ is a permutation of $X$ by finding a two-sided inverse map, namely $\sigma_{g^{-1}}$. Observe that for any $x \in X$, we have
\begin{align*}
(\sigma_{g^{-1}} \circ \sigma_g) (x) &= \sigma_{g^{-1}} ( \sigma_g (x) && \text{(by definition of function composition)} \\
					&= g^{-1} \cdot (g \cdot x) && \text{(by definition of $\sigma_g$ and $\sigma_{g^{-1}}$)} \\
					&= (g^{-1}g) \cdot x &&\text{(by axiom \ref{grp-action-axiom-1} of an action)} \\
					&= 1 \cdot x \\
					&= x &&\text{(by axiom \ref{grp-action-axiom-2} of an action)}.
\end{align*}
Thus $\sigma_{g^{-1}} \circ \sigma_g$ is the identity map on $X$. We can reverse the roles of $g$ and $g^{-1}$ to see that $\sigma_g \circ \sigma_{g^{-1}}$ is also the identity map on $X$.  Having a two-sided inverse, we conslude that $\sigma_g$ is a permutation of $X$.
\end{proof}

\begin{prop}\label{action-yields-hom}
Let $G$ act on the set $X$. The map from $G$ to the symmetric group $S_X$ defined by $g \mapsto \sigma_g (x) = g \cdot x$ is a group homomorphism.
\end{prop}
\begin{proof}
Define the map $\varphi \colon G \to S_X$ by $\varphi (g) = \sigma_g$.  We have seen from Proposition \ref{sigma-is-a-permutation} that $\sigma_g$ is indeed an element of $S_X$.  It remains to show that $\varphi(g_1 g_2) = \varphi(g_1) \circ \varphi(g_2)$ for any $g_1, g_2 \in G$.  Observe that

\begin{align*}
\varphi(g_1 g_2)(x) &= \sigma_{g_1 g_2} (x) && \text{(by definition of $\varphi$)} \\
			&= (g_1 g_2) \cdot x && \text{(by definition of $\sigma_{g_1 g_2}$)} \\
			&= g_1 \cdot (g_2 \cdot x) && \text{(by axiom \ref{grp-action-axiom-1} of an action)} \\
			&= \sigma_{g_1} ( \sigma_{g_2} (x)) && \text{(by definition of $\sigma_{g_1}$ and  $\sigma{g_2}$)} \\
			&= \varphi(g_1) ( \varphi(g_2) (x)) && \text{(by definition of $\varphi$)}\\
			&= (\varphi(g_1) \circ \varphi(g_2)) (x) && \text{(by definition of function composition)}.
\end{align*}
Since the values of $\varphi(g_1 g_2)$ and $\varphi(g_1) \circ \varphi(g_2)$ agree on every element $x \in X$, these two permutations are equal. We conclude that $\varphi$ is a homomorphism, since $g_1$ and $g_2$ were arbitrary elements of $G$.
\end{proof}


\begin{prop} \label{hom-yields-action}
Any homomorphism $\psi$ from the group $G$ into the symmetric group on $S_X$ on a set $X$ gives rise to an action of $G$ on $X$, defined by taking $g \cdot x = \psi(g)(x)$.
\end{prop}
\begin{proof}
Suppose  that we have a homomorphism $\psi$ from $G$ into $S_X$.  We can define a map from $G \times X$ to $X$  by $g \cdot x = \psi(g)(x)$. We verify that this map satisfies the definition of a group action of $G$ on $X$:
\\ (axiom \ref{grp-action-axiom-1}) \quad $1 \cdot x = \psi(1)(x) = id_X(x) = x$
\\(axiom \ref{grp-action-axiom-2}) \quad $(gh) \cdot x = \psi(gh)(x) = (\psi(g)\psi(h))(x) = \psi(g)(\psi(h)(x)) = g \cdot (h \cdot x)$
\end{proof}

\begin{prop} \label{equivalence-of-actions}
The actions of $G$ on the set $X$ are in bijective correspondence with the homomorphisms from $G$ into the symmetric group $S_X$.
\end{prop}
\begin{proof}
By Proposition \ref{action-yields-hom}, any action of $G$ on $X$ yields a homomorphism from $G$ into $S_X$.    Conversely, any homomorphism from $G$ into $S_X$ establishes an action of $G$ on $X$ by Proposition \ref{hom-yields-action}.
\end{proof}

%Todo\colon add examples?

%----------------------------------------------------------------------------------------
%	The Definition of a Representation
%
\section{The Definition of a Representation}
%----------------------------------------------------------------------------------------

\begin{defn}
\label{rep-def-1}
Let $G$ be a group, let $F$ be a field, and let $V$ be a vector space over $F$.  A \textbf{linear representation} of G is any group homomorphism $\varphi\colon G \to GL(V)$. \end{defn}
 
 %More explicitly, a representation is a map $\rho \colon G \rightarrow GL(V)$ such that \[ \rho (g_1 g_2) = \rho(g_1) \rho(g_2) \quad \forall g_1, g_2 \in G. \]

\begin{defn}\label{rep-def-2}Let $G$ be a group, let $F$ be a field, and let $V$ be a vector space over $F$. A \textbf{linear representation} of $G$ is any action of $G$ on $V$ which preserves the linear structure of $V$, that is, 
\begin{align}
\label{rep-axiom-1}&g \cdot (v_1+v_2)=g \cdot v_1+g \cdot v_2 \quad && \forall g \in G, v_1, v_2 \in V \\
\label{rep-axiom-2}&g \cdot (kv) = k (g \cdot v) \quad && \forall g \in G, v \in V, k \in F
\end{align}
 \end{defn}
 
 \begin{note}
 Unless otherwise specificed, we use \textit{representation} to mean \textit{finite-dimensional complex representation}.
 \end{note}
 
 
 \begin{prop}
The definitions of a linear representation given in \ref{rep-def-1} and \ref{rep-def-2} above are equivalent.
 \end{prop}
 \begin{proof}
%\leavevmode
 \begin{itemize}
\item[$(\rightarrow)$]  Suppose that we have a homomorphism $\varphi \colon G \to GL(V)$.  Note that $GL(V)$ is a subgroup of the symmetric group $S_V$ on $V$, so we can apply Proposition \ref{hom-yields-action} to obtain an action of $G$ on $V$ by $g \cdot v = \varphi(g)(v)$.  We check that this action preserves the linear structure of V.
\\\ref{rep-axiom-1} \quad For any $g \in G$, $v_1, v_2 \in V$ we have $g \cdot (v_1 +  v_2) = \varphi(g) (v_1 + v_2) = \varphi(g)(v_1) + \varphi(g)(v_2)= g \cdot v_1 + g \cdot v_2$.
\\\ref{rep-axiom-2} \quad For any $g \in G, v \in V, k \in F$ we have $g \cdot (kv) = \varphi(g)(kv) = k (\varphi(g)(v)) = k (g \cdot v)$.
\item[$(\leftarrow)$] Suppose that we have an action of $G$ on $V$ which preserves the linear structure of V in the sense of Definition \ref{rep-def-2}.  We can apply Proposition \ref{action-yields-hom} to obtain a homorphism $\varphi \colon G \to S_V$ given by $\varphi(g) = \sigma_g$ where $\sigma_g(v) = g \cdot v $.  It remains to show that the image $\varphi(G)$ of $G$ under $\varphi$ is actually contained in $GL(V)$, i.e. that for each $g \in G$ the map $\sigma_g$ is linear.  Fix an element $g \in G$. For any $k \in F$ and $v \in V$ we have
\begin{align*}
\sigma_g (kv) &= g \cdot (kv) && \text{(by definition of $\sigma_g$)} \\
		&= k (g \cdot v) && \text{(by property \ref{rep-axiom-1})} \\
		&= k (\sigma_g (v)) && \text{(by definition of $\sigma_g$)}.
\end{align*}
Also, for any $v_1, v_2 \in V$ we have
\begin{align*}
\sigma_g (v_1 + v_2) &= g \cdot (v_1 + v_2) && \text{(by definition of $\sigma_g$)} \\
		&= g \cdot v_1 + g \cdot v_2 && \text{(by property \ref{rep-axiom-2})} \\
		&= \sigma_g(v_1) + \sigma_g(v_2) && \text{(by definition of $\sigma_g$)}.
\end{align*}
Thus $\sigma_g$ is linear, and $\varphi(G) \subset GL(V)$ proves that we  have a homomorphism $\varphi \colon G \to GL(V)$.

\end{itemize}
 \end{proof}
 \begin{defn}Let $G$ be a group, let $F$ be a field, let $V$ be a vector space over $F$, and let $\varphi \colon G \to GL(V)$ be a representation of $G$.  The \textbf{dimension} of the representation is the dimension of $V$ over $F$.  
 \end{defn}
% ============== Examples of Representations ===========%
 \begin{example}
 \begin{enumerate}
\item Let $V$ be a $1$-dimensional vector space over the field $F$.  The map $\varphi \colon G \to GL(V)$ defined by $\varphi(g) = 1$ for all $g \in G$ is a representation called the \textit{trival representation} of $G$.  The trivial representation has dimension $1$.

\item If a finite group $G$ acts on a finite set $X$ and $F$ is any field, then there is an associated \textit{permutation representation}  on the vector space $V$ over $F$ with basis $\{e_x \colon x \in X\}$.  We let $G$ act on the basis elements by $g \cdot e_x = e_{gx}$ for all $x \in X$ and $g \in G$. Note that $G$ permutes the basis elements of $V$. 

\item A fundamental special case of a permutation representation is given by a finite group acting on itself by left multiplication.  In this case, the elements of $G$ form a basis for $V$, and each $g \in G$ permutes the basis elements by $g \cdot g_i = gg_i$.  This is called the \textit{regular representation} of $G$ and has dimension $|G|$. We shall see that this representation encodes information about all other representations of $G$.

\item For any symmetric group $S_n$ the \textit{alternating representation} on $V=\mathbb{C}$ is given by the map $\varphi \colon S_n \to GL(\mathbb{C})=\mathbb{C}^\times$ defined by $\varphi(\sigma)=\text{sgn}(\sigma)$. More generally, for any group $G$ with a subgroup $H$ of index $2$, we can define an \textit{alternating representation} $\varphi \colon G \to GL(\mathbb{C})$ by letting $\varphi(g) = 1$ if $g \in H$ and $\varphi(g) = -1$ if $g \notin H$.  (We recover our original example  by taking $G= S_n$ and $H=A_n$.)

\end{enumerate}
 \end{example}

%=========Homomorphisms===================%

\begin{defn}
A \textbf{homomorphism} between two representations $\varphi_1 \colon G \to GL(V)$ and $\varphi_2 \colon G \to GL(W)$ is a linear map $\psi \colon V \to W$ that interwines with (respects) the $G$-action, i.e. such that \[ \psi ( \varphi_1 (g)(v)) = \varphi_2(g) (\psi(v)) \quad \forall v \in V, g \in G \]  An \textbf{isomorphism} of representations is a homomorphism of representations that is also an invertible map.
\end{defn}
\begin{note}
If we have representations $(\varphi_1, V)$ and $(\varphi_2, W)$ and an isomorphism of vector spaces $\psi \colon V \to W$ then we can rewrite the compatibility requirement above as $\varphi_2(g) = \psi \circ \varphi_1(g) \circ \psi^{-1}$ for all $g \in G$.
\end{note}

Given any representation $(\varphi, V)$ of $G$ on a vector space $V$ over a field $F$ of dimension $n$, we can fix a basis for $V$ to obtain an isomorphism of vector spaces $\psi \colon V \to F^n$.  We obtain a representation $\phi$ of $G$ on $F^n$ by defining $\phi = \psi \circ \varphi(g) \circ \psi^{-1}$ for all $g \in G$. Clearly, this representation is isomorphic to the original representation $(\varphi, V)$. In particular we can always choose to view $n$-dimensional complex representations as representations on $\mathbb{C}^n$ where each $\phi(g)$ is given by an $n \times n$ matrix with entries in $\mathbb{C}$.

Suppose that we have two representations $\varphi \colon G \to GL_n(F)$ and $\phi \colon G \to GL_m(F)$ given by $\varphi(g) = X_g$ and $\phi(g) = Y_g$.  A homomorphism between these representations is then an $m \times n$ matrix $A$ such that $A X_g = Y_g A$ for all $g \in G$.   An isomorphism is given precisely when such $A$ is square and invertible.  Thus, two representations $\varphi \colon G \to GL_n(F)$ and $\phi \colon G \to GL_n(F)$ are isomorphic if and only if there exists $A \in GL_n(F)$ such that $\varphi(g) = A \phi(g) A^{-1}$ for all $g \in G$.  This establishes the following proposition:
\begin{prop}
The isomorphism classes of $n$-dimensional representations of $G$ on $\mathbb{C}$ are in bijection with the quotient $Hom(G; GL_n(\mathbb{C})) / GL_n(\mathbb{C})$ of group homomorphisms $G \to GL_n(\mathbb{C})$ modulo the conjugation action of $GL_n(\mathbb{C})$.
\end{prop}



%----------------------------------------------------------------------------------------
%	The Definition of a Representation
%
\section{Representations of Cyclic Groups}
%----------------------------------------------------------------------------------------
\begin{example}[Representations of $\mathbb{Z}$]
We want to classify all representations of the group $\mathbb{Z}$ under addition.  Consider an $n$-dimensional representation $\varphi \colon \mathbb{Z} \to GL_n$.  For $\varphi$ to be a group homomorphism requires that $\varphi(0) = \text{Id}$.  Observe that for any $0 \neq n \in \mathbb{Z}$, we have $\varphi(n) = \varphi( 1 + \ldots + 1) = \varphi(1)^n$.  Thus $\varphi$ is completely determined by the matrix $\varphi(1) \in GL_n(\mathbb{C})$, and any such matrix determines a representation of $\mathbb{Z}$.  It follows that the $n$-dimensional isomorphism classes of representations of $\mathbb{Z}$ are in bijection with the conjugacy classes in $GL_n(\mathbb{C})$.  These conjugacy classes can be parameterized by the \textit{Jordan canonical form}.
\end{example}




 %Any n-dim reprn isomorphic to a reprn on GL_N(C) (teleman)
%then 2nd paragraph look at alex bartel page 4



