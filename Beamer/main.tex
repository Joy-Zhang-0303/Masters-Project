% This text is proprietary.
% It's a part of presentation made by myself.
% It may not used commercial.
% The noncommercial use such as private and study is free
% May 2007
% Author: Sascha Frank 
% University Freiburg 
% www.informatik.uni-freiburg.de/~frank/
%
% 
\usetheme{Frankfurt}
%\useoutertheme[subsection=false]{smoothbars}  % Dots for subsection


%\setbeamertemplate{footline}[page number]{}%gets rid of bottom navigation bars

\setbeamertemplate{navigation symbols}{} %gets rid of navigation symbols

%  \useoutertheme{default}   % empty
  %\useoutertheme{infolines}% simple but bland
  %\useoutertheme{split}    % ok if compress option used
%  \useoutertheme{shadow}   % way too much space used -- ok with option 'compress'
  %\useoutertheme{shadow}   
  %\setbeamercovered{transparent} % or whatever (possibly just delete it)
  %\useoutertheme[subsection=false]{miniframes}


\title{Character Tables for Representations of Finite Groups}  
\author{Jared Stewart}
\date{\today} 

\begin{document}

\begin{frame}
\titlepage
\end{frame}

\begin{frame}\frametitle{Table of contents}\tableofcontents
\end{frame} 

\section{Introduction}

\begin{frame}{Group Representations}
Groups arise naturally as sets of symmetries of some object which are closed under composition and taking inverses.   \pause For example, 
\begin{enumerate}
\item The \textbf{symmetric group} of degree $n$, $S_n$, is the group of all symmetries of the set $\{ 1, \ldots, n \}$. \pause
\item The \textbf{dihedral group} of order $2n$, $D_{n}$, is the group of all symmetries of the regular $n$-gon in the plane.
\end{enumerate}
\pause
One may wonder more generally:  Given an abstract group $G$, which objects $X$ does $G$ act on?
This is the basic question of representation theory, which attempts to classify all such $X$ up to isomorphism.
\end{frame}

\begin{frame}{Group Actions}
\begin{definition}\label{def-grp-action}
A  \textbf{\textit{(left)} group action} of a group $G$ on a set $X$ is a map $\rho \colon G \times X \to X$ (written as $g \cdot a$, for all $g \in G$ and $a \in A$) that satisfies the following two axoims:
\begin{align}
\label{grp-action-axiom-1}&1 \cdot  x = x && \forall x \in X\\
\label{grp-action-axiom-2}&(gh) \cdot x  = g \cdot (h \cdot x) && \forall g,h \in G, x \in X
\end{align}
\end{definition}
\end{frame}
\begin{note}
We could likewise define the concept of a \textit{right} group action, where the set elements would be multiplied by group elements on the right instead of on the left.  Throughout we shall use the term \textit{group action} to mean a \textit{left} group action.
\end{note}


\begin{frame}{The Definition of a Representation}
\begin{definition}\label{rep-def-2}Let $G$ be a group, let $F$ be a field, and let $V$ be a vector space over $F$. A \textbf{linear representation} of $G$ is an action of $G$ on $V$ which preserves the linear structure of $V$, i.e. an action of $G$ on $V$ such that
\begin{align}
\label{rep-axiom-1}&g \cdot (v_1+v_2)=g \cdot v_1+g \cdot v_2 \quad && \forall g \in G, v_1, v_2 \in V \\
\label{rep-axiom-2}&g \cdot (kv) = k (g \cdot v) \quad && \forall g \in G, v \in V, k \in F
\end{align}
\end{definition}
\end{frame}

\begin{frame}{The Definition of a Representation}
\begin{definition}[Alternative definition]
\label{rep-def-1}
Let $G$ be a group, let $F$ be a field, and let $V$ be a vector space over $F$.  A \textbf{linear representation} of G is any group homomorphism $\rho\colon G \to GL(V)$. If we fix a basis for $V$, we get a representation in the previous sense.\end{definition}
\end{frame}

\section{Section no.1} 
\begin{frame}\frametitle{Title} 
Each frame should have a title.
\end{frame}
\subsection{Subsection no.1.1  }


\section{Section no. 2} 
\subsection{Lists I}

\note[itemize]{
\item point 1 Ijkflasjkssdf
\item point 2
}
\begin{frame}\frametitle{unnumbered lists}
\begin{itemize}
\item Introduction to  \LaTeX  
\item Course 2 
\item Termpapers and presentations with \LaTeX 
\item Beamer class
\end{itemize} 
\end{frame}

\begin{frame}\frametitle{lists with pause}
\begin{itemize}
\item Introduction to  \LaTeX \pause 
\item Course 2 \pause 
\item Termpapers and presentations with \LaTeX \pause 
\item Beamer class
\end{itemize} 
\end{frame}

\subsection{Lists II}
\begin{frame}\frametitle{numbered lists}
\begin{enumerate}
\item Introduction to  \LaTeX  
\item Course 2 
\item Termpapers and presentations with \LaTeX 
\item Beamer class
\end{enumerate}
\end{frame}

\begin{frame}\frametitle{numbered lists with pause}
\begin{enumerate}
\item Introduction to  \LaTeX \pause 
\item Course 2 \pause 
\item Termpapers and presentations with \LaTeX \pause 
\item Beamer class
\end{enumerate}
\end{frame}

\section{Section no.3} 
\subsection{Tables}
\begin{frame}\frametitle{Tables}
\begin{tabular}{|c|c|c|}
\hline
\textbf{Date} & \textbf{Instructor} & \textbf{Title} \\
\hline
WS 04/05 & Sascha Frank & First steps with  \LaTeX  \\
\hline
SS 05 & Sascha Frank & \LaTeX \ Course serial \\
\hline
\end{tabular}
\end{frame}


\begin{frame}\frametitle{Tables with pause}
\begin{tabular}{c c c}
A & B & C \\ 
\pause 
1 & 2 & 3 \\  
\pause 
A & B & C \\ 
\end{tabular} 
\end{frame}


\section{Section no. 4}
\subsection{blocs}
\begin{frame}\frametitle{blocs}

\begin{block}{title of the bloc}
bloc text
\end{block}

\begin{exampleblock}{title of the bloc}
bloc text
\end{exampleblock}


\begin{alertblock}{title of the bloc}
bloc text
\end{alertblock}
\end{frame}

\section{Section no. 5}
\subsection{split screen}

\begin{frame}\frametitle{splitting screen}
\begin{columns}
\begin{column}{5cm}
\begin{itemize}
\item Beamer 
\item Beamer Class 
\item Beamer Class Latex 
\end{itemize}
\end{column}
\begin{column}{5cm}
\begin{tabular}{|c|c|}
\hline
\textbf{Instructor} & \textbf{Title} \\
\hline
Sascha Frank &  \LaTeX \ Course 1 \\
\hline
Sascha Frank &  Course serial  \\
\hline
\end{tabular}
\end{column}
\end{columns}
\end{frame}


\end{document}