% This text is propositionrietary.
% It's a part of presentation made by myself.
% It may not used commercial.
% The noncommercial use such as private and study is free
% May 2007
% Author: Sascha Frank 
% University Freiburg 
% www.informatik.uni-freiburg.de/~frank/
%
% 
%\usetheme{Madrid}
%\usetheme{Warsaw}
%\usetheme{Copenhagen}
\usetheme{Frankfurt}
\usefonttheme[onlymath]{serif}
%\usefonttheme{professionalfonts}
%\useoutertheme[subsection=false]{smoothbars}  % Dots for subsection
%\setbeamertemplate{theorems}[numbered]

%\setbeamertemplate{footline}[page number]{}%gets rid of bottom navigation bars

\setbeamertemplate{navigation symbols}{} %gets rid of navigation symbols
\newtheorem{proposition}[theorem]{Proposition} 
%  \useoutertheme{default}   % empty
  %\useoutertheme{infolines}% simple but bland
  %\useoutertheme{split}    % ok if compress option used
%  \useoutertheme{shadow}   % way too much space used -- ok with option 'compress'
  %\useoutertheme{shadow}   
  %\setbeamercovered{transparent} % or whatever (possibly just delete it)
  %\useoutertheme[subsection=false]{miniframes}


\title{Character Tables for Representations of Finite Groups}  
\author{Jared Stewart}
\date{\today} 

\begin{document}

\begin{frame}
\titlepage
\end{frame}

\begin{frame}\frametitle{Table of contents}\tableofcontents
\end{frame} 

\section{Basics of Representation Theory}

\subsection{Motivation}
\begin{frame}{Motivation}
Groups arise naturally as sets of symmetries of some object which are closed under composition and taking inverses.  For example, 
\begin{enumerate}
\item The \textbf{symmetric group} of degree $n$, $S_n$, is the group of all symmetries of the set $\{ 1, \ldots, n \}$.
\item The \textbf{dihedral group} of order $2n$, $D_{n}$, is the group of all symmetries of the regular $n$-gon in the plane.
\end{enumerate}
In these two examples, $S_n$  acts on the set $\{ 1, \ldots, n \}$ and $D_{n}$ acts on the regular $n$-gon in a natural manner. One may wonder more generally:  Given an abstract group $G$, which objects $X$ does $G$ act on?
This is the basic question of representation theory, which attempts to classify all such $X$ up to isomorphism.
\end{frame}

\subsection{Group Actions}
\begin{frame}{Group Actions}
\begin{definition}\label{def-grp-action}
A  \textbf{group action} of a group $G$ on a set $X$ is a map $\rho \colon G \times X \to X$ (written as $g \cdot x$, for all $g \in G$ and $x \in X$) that satisfies the following two axoims:
\begin{align}
\label{grp-action-axiom-1}&1 \cdot  x = x && \forall x \in X\\
\label{grp-action-axiom-2}&(gh) \cdot x  = g \cdot (h \cdot x) && \forall g,h \in G, x \in X
\end{align}
\end{definition}
\end{frame}
\begin{note}
We could likewise define the concept of a \textit{right} group action, where the set elements would be multiplied by group elements on the right instead of on the left.  Throughout we shall use the term \textit{group action} to mean a \textit{left} group action.
 \end{note}

\subsection{The Definition of a Representation}
\begin{frame}{The Definition of a Representation}
\begin{definition}\label{rep-def-2}Let $G$ be a group, let $F$ be a field, and let $V$ be a vector space over $F$. A \textbf{linear representation} of $G$ is an action of $G$ on $V$ that preserves the linear structure of $V$, i.e. an action of $G$ on $V$ such that
\begin{align}
\label{rep-axiom-1}&g \cdot (v_1+v_2)=g \cdot v_1+g \cdot v_2 \quad && \forall g \in G, v_1, v_2 \in V \\
\label{rep-axiom-2}&g \cdot (kv) = k (g \cdot v) \quad && \forall g \in G, v \in V, k \in F
\end{align}
\end{definition}
\begin{definition}[Alternative definition]
\label{rep-def-1}
Let $G$ be a group, let $F$ be a field, and let $V$ be a vector space over $F$.  A \textbf{linear representation} of G is any group homomorphism \[\rho\colon G \to GL(V).\]\end{definition}
\end{frame}


\begin{frame}
\begin{proposition}
The two definitions we have given of a linear representation are equivalent.
 \end{proposition}
 \begin{proof}
\begin{itemize}
\item[$(\rightarrow)$]  Suppose that we have a homomorphism $\rho \colon G \to GL(V)$.  We can obtain a linear action of $G$ on $V$ by defining $g \cdot v = \rho(g)(v)$. 

\item[$(\leftarrow)$] Suppose that we have a linear action of $G$ on $V$.  We obtain a homomorphism $\rho \colon G \to GL(V)$ by defining $\rho(g)(v) =g \cdot v$. 
\end{itemize}
 \end{proof}
\end{frame}

\begin{frame}{The Dimension of a Representation}
 \begin{definition}Let $\rho \colon G \to GL(V)$ be a representation of $G$.  The \textbf{dimension} of the representation is the dimension of the vector space $V$.  
 \end{definition}
\end{frame}

\begin{frame}{Examples of Representations}
 \begin{example}
\only<1>{Let $V$ be an $n$-dimensional vector space.  The map $\rho \colon G \to GL(V)$ defined by $\rho(g) = \text{Id}_V$ for all $g \in G$ is a representation of $G$ called the \textbf{trival representation} of dimension $n$. }

\only<2>{ If $G$ is a finite group that acts on a finite set $X$, and $F$ is any field, then there is an associated \textbf{permutation representation}  on the vector space $V$ over $F$ with basis $\{e_x \colon x \in X\}$.  We let $G$ act on the basis elements by the permutation $g \cdot e_x = e_{gx}$ for all $x \in X$ and $g \in G$. This representation has dimension $|X|$. }

\only<3>{ A special case of a permutation representation is that when a finite group acts on itself by left multiplication. We take the vector space $V_{\text{reg}}$ which has a basis given by the formal symbols $\{ e_g | g \in G \}$, and let $h \in G$ act by \[\rho_{\text{reg}}(h) (e_g) = e_{hg}.\]  This representation is called the \textbf{regular representation} of $G$, and has dimension $|G|$.  }

\only<4>{ For any symmetric group $S_n$, the \textbf{alternating representation} on $\mathbb{C}$ is given by the map 
\begin{align*}
\rho \colon S_n &\to GL(\mathbb{C})=\mathbb{C}^\times \\
\sigma & \mapsto \text{sgn}(\sigma).
\end{align*} More generally, for any group $G$ with a subgroup $H$ of index $2$, we can define an \textbf{alternating representation} $\rho \colon G \to GL(\mathbb{C})$ by letting $\rho(g) = 1$ if $g \in H$ and $\rho(g) = -1$ if $g \notin H$.  (We recover our original example  by taking $G= S_n$ and $H=A_n$.) }
 \end{example}
\end{frame}

\begin{frame}{$G$-linear maps}
\begin{definition}
A \textbf{homomorphism} between two representations $\rho_1 \colon G \to GL(V)$ and $\rho_2 \colon G \to GL(W)$ is a linear map $\psi \colon V \to W$ that interwines with the action of $G$, i.e. such that
\[ \psi \circ \rho_1 (g)= \rho_2(g) \circ \psi \quad \forall  g \in G. \]  
In this case, we also refer to $\psi$ as a $\mathbf{G}$\textbf{-linear map}.
\end{definition}
\begin{definition}
An \textbf{isomorphism} of representations is a $G$-linear map that is also invertible.
\end{definition}
\end{frame}

\begin{frame}{Representations as matrices}
\begin{example}
\only<1>{Given any representation $(\rho, V)$, where $V$ is a vector space of dimension $n$ over the field $K$, we can fix a basis for $V$ to obtain an isomorphism of vector spaces $\psi \colon V \to K^n$.  This yields a representation $\phi$ of $G$ on $K^n$ by defining \[\phi (g) = \psi \circ \rho(g) \circ \psi^{-1}\] for all $g \in G$. This representation is isomorphic to our original representation $(\rho, V)$. In particular, we can always choose to view complex $n$-dimensional representations of $G$ as representations on $\mathbb{C}^n$, where each $\phi(g)$ is given by an $n \times n$ matrix with entries in $\mathbb{C}$.}

\only<2>{
Let $G = \{ (1), (123), (132) \} \subset S_3$.  Let $V= \mathbb{C}^3$.  Then $G$ acts on the standard basis by $g \cdot e_i = e_ {gi}$.  Thus, the permutation representation of $G$ (with respect to the standard basis) is given by:
\begin{align*}
\rho((1)) &= \begin{bmatrix} 1 & 0 & 0 \\ 0 & 1 & 0 \\ 0 & 0 & 1 \end{bmatrix} \\
\rho((123)) &= \begin{bmatrix} 0 & 0 & 1 \\ 1 & 0 & 0 \\ 0 & 1 & 0 \end{bmatrix} \\
\rho((132)) &= \begin{bmatrix} 0 & 1  & 0 \\ 0 & 0 & 1 \\ 1 & 0 & 0 \end{bmatrix}.
\end{align*}
}
\end{example}
\end{frame}

\begin{frame}[plain]
\begin{example}
Let $G= C_2 \times C_2 = \langle \sigma, \tau | \sigma^2 = \tau^2 = e, \sigma \tau = \tau \sigma \rangle$ be the Klein four-group.  Let $V$ be the vector space with basis $\{ b_e, b_\sigma, b_\tau, b_{\sigma \tau} \}$.  Left multiplication by $\sigma$ gives a permutation 
\begin{align*}
b_e &\mapsto b_\sigma\\
b_\sigma &\mapsto b_e \\
b_ \tau &\mapsto b_{\sigma \tau}\\
b_{\sigma \tau} &\mapsto b_\tau.
\end{align*}
We can similarly compute $\rho_{\text{reg}}(\tau)$.  Thus, in our basis, the regular representation $\rho_{\text{reg}} \colon G \to GL(V)$  is given by the matrices
\begin{align*}
 \rho_{\text{reg}}(\sigma) = \begin{bmatrix}0 & 1 & 0 & 0 \\  1 & 0 & 0 & 0 \\ 0 & 0 & 0 & 1 \\ 0 & 0 & 1 & 0 \end{bmatrix} & \quad \rho_{\text{reg}}(\tau) = \begin{bmatrix}0&0&1&0 \\ 0&0&0&1 \\ 1&0&0&0 \\ 0&1&0&0 \end{bmatrix} \end{align*}
\end{example}
\end{frame}

\begin{frame}[plain]
\begin{example}\label{rep-of-d8}
Let $G = D_4 = \langle \sigma, \tau |  \sigma^4 = \tau^2 = e, \tau \sigma \tau^{-1} = \sigma^{-1} \rangle$ be the symmetry group of the square.  \pause Consider a square in the plane with vertices at $(1,1), (1,-1), (-1, -1)$, and $(-1, 1)$.  \pause We let $\sigma$ act on the square as a rotation by $\frac{\pi}{2}$, and let $\tau$ act by reflection over the $x$-axis.  This naturally gives rise to a linear action of $G$ on all of $\mathbb{C}^2$.  \pause  Under the standard basis, we get the matrices:
\begin{align*}
\rho( e) &= \begin{bmatrix} 1 & 0 \\ 0 & 1\end{bmatrix}  &\rho (\tau) &= \begin{bmatrix} 1 & 0 \\ 0 & -1\end{bmatrix} \\
\rho (\sigma) &= \begin{bmatrix} 0 & -1 \\ 1 & 0  \end{bmatrix} & \rho (\sigma \tau ) &= \begin{bmatrix} 0 & 1 \\ 1 & 0\end{bmatrix} \\
\rho (\sigma^2) &= \begin{bmatrix} -1 & 0 \\ 0 & -1  \end{bmatrix} & \rho (\sigma^2 \tau) &= \begin{bmatrix} -1 & 0 \\ 0 & 1\end{bmatrix} \\
\rho (\sigma^3) &= \begin{bmatrix} 0 & 1 \\ -1 & 0  \end{bmatrix} & \rho (\sigma^3 \tau) &= \begin{bmatrix} 0 & -1 \\ -1 & 0\end{bmatrix}
\end{align*}
\end{example}
\end{frame}

\subsection{Subrepresentations}
\begin{frame}{Subrepresentations}
\begin{definition} A \textbf{subrepresentation} of $V$ is a $G$-invariant subspace $W \subseteq V$; that is, a subspace $W \subseteq V$ with the property that $\rho(g) (w) \in W$ for all $g \in G$ and $w \in W$.  Note that $W$ itself is a representation of $G$ under the action $\rho(g) \restriction_W$.
\end{definition}
\end{frame}

\begin{frame}{Representations of ${C}^2$}
\only<1>{
\begin{example}
Let $G = C_2 = \langle \tau | \tau^2 = e \rangle$ be the cyclic group of order $2$.  The regular representation of $G$ written in the standard basis is given by 
\[ \rho_{\text{reg}}(\tau)= \begin{bmatrix} 0 & 1 \\ 1 & 0 \end{bmatrix} \]
and $\rho_{\text{reg}}(e) = \text{Id}_2$.  Let $\rho_{\text{sgn}}$ be the alternating representation of $G$ on $\mathbb{C}$, i.e.
\begin{align*}
 \rho_{\text{sgn}} \colon G &\to GL_1 (\mathbb{C}) = \mathbb{C} ^ {\times} \\
\tau &\mapsto -1 \\
e &\mapsto 1.
 \end{align*} \end{example} }
 
 \only<2>{
 \begin{example}[Cont.]
Let $f \colon \mathbb{C}^2 \to \mathbb{C}$ be the linear map represented by the matrix $\begin{bmatrix} 1 & -1 \end{bmatrix}$.  Then for any  $x = \begin{bmatrix}x_1 \\ x_2 \end{bmatrix} \in \mathbb{C}^2$, we have 
\begin{align*}
f \circ \rho_{\text{reg}} (\tau) (x) &= \begin{bmatrix} 1 & -1 \end{bmatrix}  \begin{bmatrix} 0 & 1 \\ 1 & 0 \end{bmatrix}  \begin{bmatrix}x_1 \\ x_2 \end{bmatrix}  \\
&= \begin{bmatrix} -1 & 1 \end{bmatrix}  \begin{bmatrix}x_1 \\ x_2 \end{bmatrix} \\
&= \rho_{\text{sgn}} (\tau) \circ f(x).
\end{align*}
Also note that $ f \circ \rho_{\text{reg}}(e) = \rho_{\text{sgn}}(e) \circ f$.  Thus $f$ is a $G$-linear map from $\rho_{\text{reg}}$ to $\rho_{\text{sgn}}$ (i.e. a homomorphism of representations).  
\end{example}
}

\only<3>{
 \begin{example}[Cont.]
Now let $W$ be the subspace of $\mathbb{C}^2$ spanned by the vector $\begin{bmatrix}1 \\ 1  \end{bmatrix}$. Then
\[ \rho_{\text{reg}}(\tau)  \begin{bmatrix}1 \\ 1  \end{bmatrix} =   \begin{bmatrix} 0 & 1 \\ 1 & 0 \end{bmatrix} \begin{bmatrix}1 \\ 1  \end{bmatrix} = \begin{bmatrix}1 \\ 1  \end{bmatrix} \]
and  $\rho_{\text{reg}}(e) \begin{bmatrix}1 \\ 1  \end{bmatrix} = \begin{bmatrix}1 \\ 1  \end{bmatrix} $, so $W$ is a $G$-invariant subpace, i.e. $W$ is a subrepresentation of $\rho_{\text{reg}}$.  Note that $W$ is precisely equal to the kernel of the map $f$, and that $W$ is isomorphic to the $1$-dimensional trivial representation of $G$.
\end{example}

}
\end{frame}

\begin{frame}
\begin{example}
We can generalize the $G$-invariant subspace from the previous example.  Suppose we have a representation $\rho \colon G
\to GL_n
(\mathbb{C})$.  If
we can find a vector $x \in \mathbb{C}^n$ which is an eigenvector for every matrix $\rho(g), g \in G$,
i.e. an $x \in \mathbb{C}^n$ such that
\[ \rho(g) (x) = \lambda_g (x) \quad \forall g \in G\]
for some eigenvalues $\lambda_g \in \mathbb{C}$, then the span of $x$ is a $1$-dimensional $G$-invariant
subspace
 of $\mathbb{C}^n$.  It is isomorphic to the $1$-dimensional representation
\begin{align*}
 \rho_2 \colon G &\to GL_1 (\mathbb{C}) \\
g &\mapsto \lambda_g.
\end{align*}
\end{example}
\end{frame}

\begin{frame}
\begin{proposition}\label{ker-im-subreprns}
Let $f \colon V \to W$ be a homomorphism of representations of $G$.  Then $\text{Ker}(f)$ is a subrepresentation of $V$ and $\text{Im}(f)$ is a subrepresentation of $W$.
\end{proposition}
\begin{proof}
\begin{itemize}
\item Let $x \in \text{Ker}(f)$. Then $0 = g0 = g f(x) = f(gx)$ for every $g \in G$.
So $gx \in \text{Ker}(f)$ and $\text{Ker}(f)$ is $G$-invariant.

\item Now let $w \in \text{Im}(f)$. There exists $v \in V$ such that $w = f(v)$, so $g w = g f(v) = f
(gv)$ for
every $g \in G$. Thus $gw \in \text{Im}(f)$, and $\text{Im}(f)$ is $G$-invariant.
\end{itemize}
\end{proof}
\end{frame}

\section{Complete Reducibility}
\begin{frame}{The direct sum of representations}
\begin{block}{Note}
We know from linear algebra that given two vector spaces $V$ and $W$, we can form the \textbf{direct sum} $V \oplus W$ consisting of ordered pairs $(v ,w)$ where $v \in V, w \in W$.  
\end{block} \pause
\begin{definition}
Let $V$ and $W$ be representations of $G$.  Then $V \oplus W$ admits a  natural representation of $G$, called the \textbf{direct sum representation} of $V$ and $W$, which we define by 
\begin{align*}
\rho_{V \oplus W} \colon G &\to GL(V \oplus W) \\
\rho_{V \oplus W}(g) \colon (x,y) &\mapsto (\rho_{V} (g)(x), \rho_{W}(g)(y)).
\end{align*}
\end{definition}
\end{frame}

\begin{frame}{Irreducible representations and complete reducibility}
\begin{definition}
A representation is said to be \textbf{irreducible} if it has no subrepresentations other than the trivial subrepresentations $ 0 \subset V$ and $V \subset V$.  A representation is called \textbf{completely reducible} if it decomposes into a direct sum of irreducible representations.
\end{definition}

\begin{block}{Note}
\begin{enumerate}
\item Any $1$-dimensional representation $V$ has no subspaces other than $0$ and $V$ itself, and is thus irreducible.
\item Any irreducible representation is, in particular, completely reducible.
\end{enumerate} \end{block}
\end{frame}

\begin{frame}
\only<1>{
\begin{example}[A $2$-dimensional irreducible representation] 
Let $G = D_3 = \langle \sigma, \tau | \sigma^3 = \tau^2 = e, \tau \sigma \tau^{-1} = \sigma^{-1} \rangle$. (Note that $D_3 \cong S_3$).
Consider the regular triangle centered at the origin with vertices
\[(1,0), (-\frac{1}{2}, \frac{\sqrt{3}}{2}), (-\frac{1}{2}, - \frac{\sqrt{3}}{2}). \]
We can let $\sigma$ act as rotation by $\frac{2 \pi}{3}$ and let $\tau$ act as reflection over the $x$-axis to obtain an action of $G$ on $\mathbb{C}^2$ given (under the standard basis) by the matrices
\begin{align*}
\rho(\sigma) &= \begin{bmatrix}-\frac{1}{2} & -\frac{\sqrt{3}}{2} \\ \frac{\sqrt{3}}{2} & -\frac{1}{2}\end{bmatrix} \\
\rho(\tau) &= \begin{bmatrix}1 & 0 \\ 0 & -1 \end{bmatrix}
\end{align*} \end{example} }

\only<2>{
\begin{example}[A $2$-dimensional irreducible representation cont.] 
Suppose $\rho$ has a non-trivial subrepresentation $W$.  We must have $\text{dim }W =1$.  Since $W$ is invariant under the action of both $\rho(\sigma)$ and $\rho(\tau)$, there must be some mutual eigenvector for $\rho(\sigma)$ and $\rho(\tau)$ that spans $W$.  The eigenvectors of $\rho(\sigma)$ are
\begin{align*}
\begin{bmatrix} 1 \\ -i \end{bmatrix} \quad (\lambda_1 = e^{\frac{2 \pi i}{3}}) \quad \text{and} \quad
\begin{bmatrix} 1 \\ i \end{bmatrix} \quad (\lambda_2 = e^{-\frac{2 \pi i}{3}}) .
\end{align*}  
The eigenvectors of $\rho(\tau)$ are 
\begin{align*}
\begin{bmatrix} 1 \\ 0 \end{bmatrix} \quad (\lambda_1 = 1) \quad \text{and} \quad
\begin{bmatrix} 0 \\ 1 \end{bmatrix} \quad (\lambda_2 = -1).
\end{align*}
Thus we see that there is no such $W$, and our representation is irreducible.
\end{example}}
\end{frame}

\begin{frame}{Representations of finite abelian groups}
\only<1>{\begin{theorem}If $A_1, A_2, \ldots, A_r$ are linear operators on $V$ and each $A_i$ is diagonalizable, then $\{A_i\}$ are simultaneously diagonalizable if and only if they commute.
\end{theorem}}

\only<2>{
\begin{theorem} Every complex representation of a finite abelian group is completely reducible into irreducible representations of dimension $1$.  
\end {theorem}
\begin{proof}
Take an arbitrary element $g \in G$.  Since $G$ is finite, we can find an integer $n$ such that $g^n = 1$ and $\rho(g)^n = Id$.    The minimal polynomial of $\rho(g)$ divides  $x^n -1$, which has $n$ distinct roots over $\mathbb{C}$.  So the minimal polynomial of $\rho(g)$ factors into linear factors only over $\mathbb{C}$,  i.e. $\rho(g)$ is diagonalizable.  We conclude that each $\rho(g)$ is (separately) diagonalizable since $g \in G$ was arbitrary.

Now, given any two elements $g_1, g_2 \in G$ we have $\rho(g_1)\rho(g_2)=\rho(g_2)\rho(g_1)$.
Since the matrices $\left\{ \rho(g)\right\}$ commute,  $\left\{ \rho(g)\right\}$ are simultaneously diagonalizable, say with basis $\left\{ e_1, ..., e_k \right\}$.  Then we have $V= \mathbb{C}e_1 \oplus \mathbb{C} e_2 \oplus \ldots \oplus \mathbb{C} e_n$, with each subspace $ \mathbb{C}e_1$ invariant under the action of $G$.
\end{proof}
}
\end{frame}


\begin{frame}
\begin{definition}
Let $W$ be a subspace of $V$.  A \textbf{linear projection} $V$ onto $W$ is a linear map $f \colon V \to W$ such that $f \restriction_{W} = \text{Id}_W$.  If $W$ is a subrepresentation of $V$ and the map $f$ is $G$-invariant, then we say that $f$ is a $\mathbf{G}$\textbf{-linear projection}.
\end{definition}

\begin{lemma} \label{maschke-lemma}
Let $\rho \colon G \to GL(V)$ be a representation, and $W \subset V$ be a subrepresentation.  Suppose we have a $G$-linear projection 
\[ f \colon V \to W. \]
Then $\text{Ker}(f)$ is a complementary subrepresentation to $W$, i.e. $\text{Ker}(f)$ is a $G$-invariant subspace of $V$ such that
\[ V = \text{Ker}(f) \oplus W \]
\end{lemma}
\end{frame}

\begin{frame}{Maschke's Theorem}
\begin{theorem}[Maschke's Theorem]
Let $G$ be a finite group and let $F$ be a field such that $\text{char}(F) \nmid |G|$.  If $V$ is any finite dimensional representation of $G$ over $F$, and $W \subset V$ is a subrepresentation of $V$, then there exists a complementary subrepresentation $U \subset V$ to $W$, i.e. there is  a $G$-invariant subspace $U \subset V$ such that 
\[ V = W \oplus U. \]
\end{theorem}
\end{frame}

\begin{frame}{Maschke's Theorem}
\begin{proof}
\only<1>{
It will suffice to find a $G$-linear projection from $V$ onto $W$.  Fix a basis $\{ b_1, \ldots, b_m \}$ for $W$ and extend it to a basis  $\{ b_1, \ldots, b_m, b_{m+1}, \ldots, b_n \}$ for $V$.  Let $U = \langle b_{m+1}, \ldots, b_n \rangle$.  Then $U$ is certainly a complementary subspace to $W$, and we have a natural projection $f \colon W \oplus U \to W$ of $V$ onto $W$ with kernel $U$.
There is no reason to think that $f$ should be $G$-linear, but we can fix this by averaging over $G$.  Define $\widetilde{f} \colon V \to V$ by
\[\widetilde{f}(x) = \frac{1}{|G|} \sum_{g \in G} (\rho(g) \circ f \circ \rho(g^{-1}))(x). \]
We claim that $\widetilde{f}$ is a $G$-linear projection from $V$ onto $W$.   } 

\only<2>{

First we check that $\text{Im}(\tilde{f}) \subset W$.  If $x \in V$ and $g \in G$, then
\[ f (\rho (g^{-1})(x)) \in W \]
and so
\[ \rho(g) ( f ( \rho( g^{-1})(x))) \in W \]
since $W$ is $G$-invariant.  Thus \[ \widetilde{f}(x) =  \frac{1}{|G|} \sum_{g \in G} (\rho(g) \circ f \circ \rho(g^{-1}))(x) \in W. \] }

\only<3>{ Next we check that $\widetilde{f} \restriction_{W} = \text{Id}_W$. Let $y \in W$.  For any $g \in G$, we know that $\rho(g^{-1})(y)$ is also in $W$, so
$ f (\rho(g^{-1})(y)) = \rho (g^{-1})(y)$.
Then
\begin{align*}
\widetilde{f}(y) &=  \frac{1}{|G|} \sum_{g \in G} \rho(g) (f ( \rho(g^{-1})(y))) \\
&=\frac{1}{|G|} \sum_{g \in G} \rho(g) (\rho(g^{-1})(y)) \\
&=\frac{1}{|G|} \sum_{g \in G} (y)  = \frac{|G| y} {|G|}  = y
\end{align*}
so indeed $\widetilde{f}$ is a linear projection of $V$ onto $W$. }

\only<4> { Finally, we check that $\widetilde{f}$ is $G$-linear.  If $x \in V$ and $h \in G$, then
\begin{align*}
(\widetilde{f} \circ \rho(h))(x) &= \frac{1}{|G|} \sum_{g \in G} (\rho(g) \circ f \circ \rho(g^{-1}) \circ \rho(h))(x) \\
&= \frac{1}{|G|} \sum_{g \in G} (\rho(g) \circ f \circ \rho(g^{-1} h))(x) \\
&=\frac{1}{|G|} \sum_{g \in G} (\rho(hg) \circ f \circ \rho(g^{-1}))(x) \quad \text{(} g \mapsto hg \text{)} \\
&= (\rho(h) \circ \widetilde{f}) (x).
\end{align*}}
\alt<4>{\qedhere}{\phantom\qedhere}
\end{proof}
	
\end{frame}

\end{document}