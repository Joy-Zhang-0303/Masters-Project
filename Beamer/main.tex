% This text is propositionrietary.
% It's a part of presentation made by myself.
% It may not used commercial.
% The noncommercial use such as private and study is free
% May 2007
% Author: Sascha Frank 
% University Freiburg 
% www.informatik.uni-freiburg.de/~frank/
%
% 
%\usetheme{Madrid}
%\usetheme{Warsaw}
%\usetheme{Copenhagen}
\usetheme{Frankfurt}
\usefonttheme[onlymath]{serif}
%\usefonttheme{professionalfonts}
%\useoutertheme[subsection=false]{smoothbars}  % Dots for subsection
%\setbeamertemplate{theorems}[numbered]

%\setbeamertemplate{footline}[page number]{}%gets rid of bottom navigation bars

\setbeamertemplate{navigation symbols}{} %gets rid of navigation symbols
\newtheorem{proposition}[theorem]{Proposition} 
%  \useoutertheme{default}   % empty
  %\useoutertheme{infolines}% simple but bland
  %\useoutertheme{split}    % ok if compress option used
%  \useoutertheme{shadow}   % way too much space used -- ok with option 'compress'
  %\useoutertheme{shadow}   
  %\setbeamercovered{transparent} % or whatever (possibly just delete it)
  %\useoutertheme[subsection=false]{miniframes}


\title{Character Tables for Representations of Finite Groups}  
\author{Jared Stewart}
\date{\today} 

\begin{document}

\begin{frame}
\titlepage
\end{frame}

\begin{frame}\frametitle{Table of contents}\tableofcontents
\end{frame} 

\section{Basics of Representation Theory}

\subsection{Motivation}
\begin{frame}{Motivation}
Groups arise naturally as sets of symmetries of some object which are closed under composition and taking inverses.  For example, 
\begin{enumerate}
\item The \textbf{symmetric group} of degree $n$, $S_n$, is the group of all symmetries of the set $\{ 1, \ldots, n \}$.
\item The \textbf{dihedral group} of order $2n$, $D_{n}$, is the group of all symmetries of the regular $n$-gon in the plane.
\end{enumerate}
In these two examples, $S_n$  acts on the set $\{ 1, \ldots, n \}$ and $D_{n}$ acts on the regular $n$-gon in a natural manner. One may wonder more generally:  Given an abstract group $G$, which objects $X$ does $G$ act on?
This is the basic question of representation theory, which attempts to classify all such $X$ up to isomorphism.
\end{frame}

\subsection{Group Actions}
\begin{frame}{Group Actions}
\begin{definition}\label{def-grp-action}
A  \textbf{group action} of a group $G$ on a set $X$ is a map $\rho \colon G \times X \to X$ (written as $g \cdot x$, for all $g \in G$ and $x \in X$) that satisfies the following two axoims:
\begin{align}
\label{grp-action-axiom-1}&1 \cdot  x = x && \forall x \in X\\
\label{grp-action-axiom-2}&(gh) \cdot x  = g \cdot (h \cdot x) && \forall g,h \in G, x \in X
\end{align}
\end{definition}
\end{frame}
\begin{note}
We could likewise define the concept of a \textit{right} group action, where the set elements would be multiplied by group elements on the right instead of on the left.  Throughout we shall use the term \textit{group action} to mean a \textit{left} group action.
 \end{note}

\subsection{The Definition of a Representation}
\begin{frame}{The Definition of a Representation}
\begin{definition}\label{rep-def-2}Let $G$ be a group, let $F$ be a field, and let $V$ be a vector space over $F$. A \textbf{linear representation} of $G$ is an action of $G$ on $V$ that preserves the linear structure of $V$, i.e. an action of $G$ on $V$ such that
\begin{align}
\label{rep-axiom-1}&g \cdot (v_1+v_2)=g \cdot v_1+g \cdot v_2 \quad && \forall g \in G, v_1, v_2 \in V \\
\label{rep-axiom-2}&g \cdot (kv) = k (g \cdot v) \quad && \forall g \in G, v \in V, k \in F
\end{align}
\end{definition}
\begin{definition}[Alternative definition]
\label{rep-def-1}
Let $G$ be a group, let $F$ be a field, and let $V$ be a vector space over $F$.  A \textbf{linear representation} of G is any group homomorphism \[\rho\colon G \to GL(V).\]\end{definition}
\end{frame}


\begin{frame}
\begin{proposition}
The two definitions we have given of a linear representation are equivalent.
 \end{proposition}
 \begin{proof}
\begin{itemize}
\item[$(\rightarrow)$]  Suppose that we have a homomorphism $\rho \colon G \to GL(V)$.  We can obtain a linear action of $G$ on $V$ by defining $g \cdot v = \rho(g)(v)$. 

\item[$(\leftarrow)$] Suppose that we have a linear action of $G$ on $V$.  We obtain a homomorphism $\rho \colon G \to GL(V)$ by defining $\rho(g)(v) =g \cdot v$. 
\end{itemize}
 \end{proof}
\end{frame}

\begin{frame}{The Dimension of a Representation}
 \begin{definition}Let $\rho \colon G \to GL(V)$ be a representation of $G$.  The \textbf{dimension} of the representation is the dimension of the vector space $V$.  
 \end{definition}
\end{frame}

\begin{frame}{Examples of Representations}
 \begin{example}
\only<1>{Let $V$ be an $n$-dimensional vector space.  The map $\rho \colon G \to GL(V)$ defined by $\rho(g) = \text{Id}_V$ for all $g \in G$ is a representation of $G$ called the \textbf{trival representation} of dimension $n$. }

\only<2>{ If $G$ is a finite group that acts on a finite set $X$, and $F$ is any field, then there is an associated \textbf{permutation representation}  on the vector space $V$ over $F$ with basis $\{e_x \colon x \in X\}$.  We let $G$ act on the basis elements by the permutation $g \cdot e_x = e_{gx}$ for all $x \in X$ and $g \in G$. This representation has dimension $|X|$. }

\only<3>{ A special case of a permutation representation is that when a finite group acts on itself by left multiplication. We take the vector space $V_{\text{reg}}$ which has a basis given by the formal symbols $\{ e_g | g \in G \}$, and let $h \in G$ act by \[\rho_{\text{reg}}(h) (e_g) = e_{hg}.\]  This representation is called the \textbf{regular representation} of $G$, and has dimension $|G|$.  }

\only<4>{ For any symmetric group $S_n$, the \textbf{alternating representation} on $\mathbb{C}$ is given by the map 
\begin{align*}
\rho \colon S_n &\to GL(\mathbb{C})=\mathbb{C}^\times \\
\sigma & \mapsto \text{sgn}(\sigma).
\end{align*} More generally, for any group $G$ with a subgroup $H$ of index $2$, we can define an \textit{alternating representation} $\rho \colon G \to GL(\mathbb{C})$ by letting $\rho(g) = 1$ if $g \in H$ and $\rho(g) = -1$ if $g \notin H$.  (We recover our original example  by taking $G= S_n$ and $H=A_n$.) }
 \end{example}
\end{frame}

\begin{frame}{$G$-linear maps}
\begin{definition}
A \textbf{homomorphism} between two representations $\rho_1 \colon G \to GL(V)$ and $\rho_2 \colon G \to GL(W)$ is a linear map $\psi \colon V \to W$ that interwines with the action of $G$, i.e. such that
\[ \psi \circ \rho_1 (g)= \rho_2(g) \circ \psi \quad \forall  g \in G. \]  

In this case, we also refer to $\psi$ as a $\mathbf{G}$\textbf{-linear map}.
\end{definition}
\begin{definition}
An \textbf{isomorphism} of representations is a $G$-linear map that is also invertible.
\end{definition}
\end{frame}

\section{Section no.1} 
\begin{frame}\frametitle{Title} 
Each frame should have a title.
\end{frame}
\subsection{Subsection no.1.1  }


\section{Section no. 2} 
\subsection{Lists I}

\note[itemize]{
\item point 1 Ijkflasjkssdf
\item point 2
}
\begin{frame}\frametitle{unnumbered lists}
\begin{itemize}
\item Introduction to  \LaTeX  
\item Course 2 
\item Termpapers and presentations with \LaTeX 
\item Beamer class
\end{itemize} 
\end{frame}

\begin{frame}\frametitle{lists with pause}
\begin{itemize}
\item Introduction to  \LaTeX \pause 
\item Course 2 \pause 
\item Termpapers and presentations with \LaTeX \pause 
\item Beamer class
\end{itemize} 
\end{frame}

\subsection{Lists II}
\begin{frame}\frametitle{numbered lists}
\begin{enumerate}
\item Introduction to  \LaTeX  
\item Course 2 
\item Termpapers and presentations with \LaTeX 
\item Beamer class
\end{enumerate}
\end{frame}

\begin{frame}\frametitle{numbered lists with pause}
\begin{enumerate}
\item Introduction to  \LaTeX \pause 
\item Course 2 \pause 
\item Termpapers and presentations with \LaTeX \pause 
\item Beamer class
\end{enumerate}
\end{frame}

\section{Section no.3} 
\subsection{Tables}
\begin{frame}\frametitle{Tables}
\begin{tabular}{|c|c|c|}
\hline
\textbf{Date} & \textbf{Instructor} & \textbf{Title} \\
\hline
WS 04/05 & Sascha Frank & First steps with  \LaTeX  \\
\hline
SS 05 & Sascha Frank & \LaTeX \ Course serial \\
\hline
\end{tabular}
\end{frame}


\begin{frame}\frametitle{Tables with pause}
\begin{tabular}{c c c}
A & B & C \\ 
\pause 
1 & 2 & 3 \\  
\pause 
A & B & C \\ 
\end{tabular} 
\end{frame}


\section{Section no. 4}
\subsection{blocs}
\begin{frame}\frametitle{blocs}

\begin{block}{title of the bloc}
bloc text
\end{block}

\begin{exampleblock}{title of the bloc}
bloc text
\end{exampleblock}


\begin{alertblock}{title of the bloc}
bloc text
\end{alertblock}
\end{frame}

\section{Section no. 5}
\subsection{split screen}

\begin{frame}\frametitle{splitting screen}
\begin{columns}
\begin{column}{5cm}
\begin{itemize}
\item Beamer 
\item Beamer Class 
\item Beamer Class Latex 
\end{itemize}
\end{column}
\begin{column}{5cm}
\begin{tabular}{|c|c|}
\hline
\textbf{Instructor} & \textbf{Title} \\
\hline
Sascha Frank &  \LaTeX \ Course 1 \\
\hline
Sascha Frank &  Course serial  \\
\hline
\end{tabular}
\end{column}
\end{columns}
\end{frame}


\end{document}