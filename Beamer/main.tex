% This text is propositionrietary.
% It's a part of presentation made by myself.
% It may not used commercial.
% The noncommercial use such as private and study is free
% May 2007
% Author: Sascha Frank 
% University Freiburg 
% www.informatik.uni-freiburg.de/~frank/
%
% 
%\usetheme{Madrid}
%\usetheme{Warsaw}
%\usetheme{Copenhagen}
\usetheme{Frankfurt}
\usefonttheme[onlymath]{serif}

%\useoutertheme{miniframes}
%\makeatletter
  %\beamer@compressfalse
%\makeatother


%\usefonttheme{professionalfonts}
%\useoutertheme[subsection=false]{smoothbars}  % Dots for subsection
%\setbeamertemplate{theorems}[numbered]

%\setbeamertemplate{footline}[page number]{}%gets rid of bottom navigation bars

\setbeamertemplate{navigation symbols}{} %gets rid of navigation symbols
\newtheorem{proposition}[theorem]{Proposition} 
%  \useoutertheme{default}   % empty
  %\useoutertheme{infolines}% simple but bland
  %\useoutertheme{split}    % ok if compress option used
%  \useoutertheme{shadow}   % way too much space used -- ok with option 'compress'
  %\useoutertheme{shadow}   
  %\setbeamercovered{transparent} % or whatever (possibly just delete it)
  %\useoutertheme[subsection=false]{miniframes}


\title{Character Tables for Representations of Finite Groups}  
\author{Jared Stewart \\Advised by Dr. Calin Chindris }

\date{\today \\ University of Missouri} 

\begin{document}

\begin{frame}
\titlepage
\end{frame}

\begin{frame}\frametitle{Table of contents}\tableofcontents
\end{frame} 

\section{Basics of Rep. Theory}
\subsection{Motivation and Definitions}
\begin{frame}{Motivation}
Groups arise naturally as sets of symmetries of some object which are closed under composition and taking inverses.  For example, 
\begin{enumerate}
\item The \textbf{symmetric group} of degree $n$, $S_n$, is the group of all symmetries of the set $\{ 1, \ldots, n \}$.
\item The \textbf{dihedral group} of order $2n$, $D_{n}$, is the group of all symmetries of the regular $n$-gon in the plane.
\end{enumerate}
In these two examples, $S_n$  acts on the set $\{ 1, \ldots, n \}$ and $D_{n}$ acts on the regular $n$-gon in a natural manner. One may wonder more generally:  Given an abstract group $G$, which objects $X$ does $G$ act on?
This is the basic question of representation theory, which attempts to classify all such $X$ up to isomorphism.
\end{frame}


\begin{frame}{The Definition of a Representation}
\begin{definition}
Let $G$ be a group, let $F$ be a field, and let $V$ be a vector space over $F$.  A \textbf{linear representation} of G is any group homomorphism \[\rho\colon G \to GL(V).\]\end{definition}
 \begin{definition}Let $\rho \colon G \to GL(V)$ be a representation of $G$.  The \textbf{dimension} of the representation is the dimension of the vector space $V$.  
 \end{definition}
\end{frame}


\subsection{Examples of Representations}
\begin{frame}{Examples of Representations}
\begin{example}
Let $V$ be an $n$-dimensional vector space.  The map $\rho \colon G \to GL(V)$ defined by $\rho(g) = \text{Id}_V$ for all $g \in G$ is a representation of $G$ called the \textbf{trival representation} of dimension $n$. 
\end{example}

\begin{example} If $G$ is a finite group that acts on a finite set $X$, and $F$ is any field, then there is an associated \textbf{permutation representation}  on the vector space $V$ over $F$ with basis $\{e_x \colon x \in X\}$.  We let $G$ act on the basis elements by the permutation $g \cdot e_x = e_{gx}$ for all $x \in X$ and $g \in G$. This representation has dimension $|X|$.  \end{example}
\end{frame}

\begin{frame}{The Regular Representation}
\begin{example}
A special case of a permutation representation is that when a finite group acts on itself by left multiplication. We take the vector space $V_{\text{reg}}$ which has a basis given by the formal symbols $\{ e_g | g \in G \}$, and let $h \in G$ act by \[\rho_{\text{reg}}(h) (e_g) = e_{hg}.\]  This representation is called the \textbf{regular representation} of $G$, and has dimension $|G|$.  
 \end{example}
\end{frame}

\begin{frame}{Examples of Representations}
\begin{example}
For any symmetric group $S_n$, the \textbf{alternating representation} on $\mathbb{C}$ is given by the map 
\begin{align*}
\rho \colon S_n &\to GL(\mathbb{C})=\mathbb{C}^\times \\
\sigma & \mapsto \text{sgn}(\sigma).
\end{align*} More generally, for any group $G$ with a subgroup $H$ of index $2$, we can define an \textbf{alternating representation} $\rho \colon G \to GL(\mathbb{C})$ by letting $\rho(g) = 1$ if $g \in H$ and $\rho(g) = -1$ if $g \notin H$.  (We recover our original example  by taking $G= S_n$ and $H=A_n$.) 
\end{example}
\end{frame}

\begin{frame}{$G$-linear maps}
\begin{definition}
A \textbf{homomorphism} between two representations $\rho_1 \colon G \to GL(V)$ and $\rho_2 \colon G \to GL(W)$ is a linear map $\psi \colon V \to W$ that interwines with the action of $G$, i.e. such that
\[ \psi \circ \rho_1 (g)= \rho_2(g) \circ \psi \quad \forall  g \in G. \]  
In this case, we also refer to $\psi$ as a $\mathbf{G}$\textbf{-linear map}.
\end{definition}
\begin{definition}
An \textbf{isomorphism} of representations is a $G$-linear map that is also invertible.
\end{definition}
\end{frame}

\begin{frame}{Representations as matrices}
\begin{example}
\only<1>{Given any representation $(\rho, V)$, where $V$ is a vector space of dimension $n$ over the field $K$, we can fix a basis for $V$ to obtain an isomorphism of vector spaces $\psi \colon V \to K^n$.  This yields a representation $\phi$ of $G$ on $K^n$ by defining \[\phi (g) = \psi \circ \rho(g) \circ \psi^{-1}\] for all $g \in G$. This representation is isomorphic to our original representation $(\rho, V)$. In particular, we can always choose to view complex $n$-dimensional representations of $G$ as representations on $\mathbb{C}^n$, where each $\phi(g)$ is given by an invertible $n \times n$ matrix with entries in $\mathbb{C}$.}

\only<2>{
Let $G = \{ (1), (123), (132) \} \subset S_3$.  Let $V= \mathbb{C}^3$.  Then $G$ acts on the standard basis by $g \cdot e_i = e_ {gi}$.  Thus, the permutation representation of $G$ (with respect to the standard basis) is given by:
\begin{align*}
\rho((1)) &= \begin{bmatrix} 1 & 0 & 0 \\ 0 & 1 & 0 \\ 0 & 0 & 1 \end{bmatrix} \\
\rho((123)) &= \begin{bmatrix} 0 & 0 & 1 \\ 1 & 0 & 0 \\ 0 & 1 & 0 \end{bmatrix} \\
\rho((132)) &= \begin{bmatrix} 0 & 1  & 0 \\ 0 & 0 & 1 \\ 1 & 0 & 0 \end{bmatrix}.
\end{align*}
}
\end{example}
\end{frame}

\begin{frame}[plain]
\begin{example}
Let $G= C_2 \times C_2 = \langle \sigma, \tau | \sigma^2 = \tau^2 = e, \sigma \tau = \tau \sigma \rangle$ be the Klein four-group.  Let $V$ be the vector space with basis $\{ b_e, b_\sigma, b_\tau, b_{\sigma \tau} \}$.  Left multiplication by $\sigma$ gives a permutation 
\begin{align*}
b_e &\mapsto b_\sigma\\
b_\sigma &\mapsto b_e \\
b_ \tau &\mapsto b_{\sigma \tau}\\
b_{\sigma \tau} &\mapsto b_\tau.
\end{align*}
We can similarly compute $\rho_{\text{reg}}(\tau)$.  Thus, in our basis, the regular representation $\rho_{\text{reg}} \colon G \to GL(V)$  is given by the matrices
\begin{align*}
 \rho_{\text{reg}}(\sigma) = \begin{bmatrix}0 & 1 & 0 & 0 \\  1 & 0 & 0 & 0 \\ 0 & 0 & 0 & 1 \\ 0 & 0 & 1 & 0 \end{bmatrix} & \quad \rho_{\text{reg}}(\tau) = \begin{bmatrix}0&0&1&0 \\ 0&0&0&1 \\ 1&0&0&0 \\ 0&1&0&0 \end{bmatrix} \end{align*}
\end{example}
\end{frame}

\begin{frame}[plain]
\begin{example}[$2$-dim rep of $D_4$.]
Let $G = D_4 = \langle \sigma, \tau |  \sigma^4 = \tau^2 = e, \tau \sigma \tau^{-1} = \sigma^{-1} \rangle$ be the symmetry group of the square.  \pause Consider a square in the plane with vertices at $(1,1), (1,-1), (-1, -1)$, and $(-1, 1)$.  \pause We let $\sigma$ act on the square as a rotation by $\frac{\pi}{2}$, and let $\tau$ act by reflection over the $x$-axis.  This naturally gives rise to a linear action of $G$ on all of $\mathbb{C}^2$.  \pause  Under the standard basis, we get the matrices:
\begin{align*}
\rho( e) &= \begin{bmatrix} 1 & 0 \\ 0 & 1\end{bmatrix}  &\rho (\tau) &= \begin{bmatrix} 1 & 0 \\ 0 & -1\end{bmatrix} \\
\rho (\sigma) &= \begin{bmatrix} 0 & -1 \\ 1 & 0  \end{bmatrix} & \rho (\sigma \tau ) &= \begin{bmatrix} 0 & 1 \\ 1 & 0\end{bmatrix} \\
\rho (\sigma^2) &= \begin{bmatrix} -1 & 0 \\ 0 & -1  \end{bmatrix} & \rho (\sigma^2 \tau) &= \begin{bmatrix} -1 & 0 \\ 0 & 1\end{bmatrix} \\
\rho (\sigma^3) &= \begin{bmatrix} 0 & 1 \\ -1 & 0  \end{bmatrix} & \rho (\sigma^3 \tau) &= \begin{bmatrix} 0 & -1 \\ -1 & 0\end{bmatrix}
\end{align*}
\end{example}
\end{frame}

\section{Reducibility}
\subsection{Irreducible representations and complete reducibility}
\begin{frame}{The direct sum of representations}
\begin{definition}
Let $V$ and $W$ be representations of $G$.  Then $V \oplus W$ admits a  natural representation of $G$, called the \textbf{direct sum representation} of $V$ and $W$, which we define by 
\begin{align*}
\rho_{V \oplus W} \colon G &\to GL(V \oplus W) \\
\rho_{V \oplus W}(g) \colon (x,y) &\mapsto (\rho_{V} (g)(x), \rho_{W}(g)(y)).
\end{align*}
\end{definition}
\end{frame}

\begin{frame}{Irreducible representations and complete reducibility}
\begin{definition} A \textbf{subrepresentation} of $V$ is a $G$-invariant subspace $W \subseteq V$; that is, a subspace $W \subseteq V$ with the property that $\rho(g) (w) \in W$ for all $g \in G$ and $w \in W$.  Note that $W$ itself is a representation of $G$ under the action $\rho(g) \restriction_W$.
\end{definition}
\begin{definition}
A representation is said to be \textbf{irreducible} if it has no subrepresentations other than the trivial subrepresentations $ 0 \subset V$ and $V \subset V$.  A representation is called \textbf{completely reducible} if it decomposes into a direct sum of irreducible representations.  We sometimes write \textbf{irrep} as shorthand for irreducible representation.
\end{definition}

\begin{block}{Note}
\begin{enumerate}
\item Any $1$-dimensional representation $V$ has no subspaces other than $0$ and $V$ itself, and is thus irreducible.
\item Any irreducible representation is, in particular, completely reducible.
\end{enumerate} \end{block}
\end{frame}

\begin{frame}
\only<1>{
\begin{example}[A $2$-dimensional irrep] 
Let $G = D_3 = \langle \sigma, \tau | \sigma^3 = \tau^2 = e, \tau \sigma \tau^{-1} = \sigma^{-1} \rangle$. (Note that $D_3 \cong S_3$).
Consider the regular triangle centered at the origin with vertices
\[(1,0), (-\frac{1}{2}, \frac{\sqrt{3}}{2}), (-\frac{1}{2}, - \frac{\sqrt{3}}{2}). \]
We can let $\sigma$ act as rotation by $\frac{2 \pi}{3}$ and let $\tau$ act as reflection over the $x$-axis to obtain an action of $G$ on $\mathbb{C}^2$ given (under the standard basis) by the matrices
\begin{align*}
\rho(\sigma) &= \begin{bmatrix}-\frac{1}{2} & -\frac{\sqrt{3}}{2} \\ \frac{\sqrt{3}}{2} & -\frac{1}{2}\end{bmatrix} \\
\rho(\tau) &= \begin{bmatrix}1 & 0 \\ 0 & -1 \end{bmatrix}
\end{align*} \end{example} }

\only<2>{
\begin{example}[A $2$-dimensional irrep cont.] 
Suppose $\rho$ has a non-trivial subrepresentation $W$.  We must have $\text{dim }W =1$.  Since $W$ is invariant under the action of both $\rho(\sigma)$ and $\rho(\tau)$, there must be some mutual eigenvector for $\rho(\sigma)$ and $\rho(\tau)$ that spans $W$.  The eigenvectors of $\rho(\sigma)$ are
\begin{align*}
\begin{bmatrix} 1 \\ -i \end{bmatrix} \quad (\lambda_1 = e^{\frac{2 \pi i}{3}}) \quad \text{and} \quad
\begin{bmatrix} 1 \\ i \end{bmatrix} \quad (\lambda_2 = e^{-\frac{2 \pi i}{3}}) .
\end{align*}  
The eigenvectors of $\rho(\tau)$ are 
\begin{align*}
\begin{bmatrix} 1 \\ 0 \end{bmatrix} \quad (\lambda_1 = 1) \quad \text{and} \quad
\begin{bmatrix} 0 \\ 1 \end{bmatrix} \quad (\lambda_2 = -1).
\end{align*}
Thus we see that there is no such $W$, and our representation is irreducible.
\end{example}}
\end{frame}

\begin{frame}{Representations of finite abelian groups}
\only<1>{\begin{theorem}If $A_1, A_2, \ldots, A_r$ are linear operators on $V$ and each $A_i$ is diagonalizable, then $\{A_i\}$ are simultaneously diagonalizable if and only if they commute.
\end{theorem}}

\only<2>{
\begin{theorem} Every complex representation of a finite abelian group is completely reducible into irreducible representations of dimension $1$.  
\end {theorem}
\begin{proof}
Take an arbitrary element $g \in G$.  Since $G$ is finite, we can find an integer $n$ such that $g^n = 1$ and $\rho(g)^n = Id$.    The minimal polynomial of $\rho(g)$ divides  $x^n -1$, which has $n$ distinct roots over $\mathbb{C}$, so it factors into linear factors only over $\mathbb{C}$,  i.e. $\rho(g)$ is diagonalizable.  We conclude that each $\rho(g)$ is (separately).
Now, given any two elements $g_1, g_2 \in G$ we have $\rho(g_1)\rho(g_2)=\rho(g_2)\rho(g_1)$.
Since the matrices $\left\{ \rho(g)\right\}$ commute,  $\left\{ \rho(g)\right\}$ are simultaneously diagonalizable, say with basis $\left\{ e_1, ..., e_k \right\}$.  Then we have $V= \mathbb{C}e_1 \oplus \mathbb{C} e_2 \oplus \ldots \oplus \mathbb{C} e_n$, with each subspace $ \mathbb{C}e_i$ invariant under the action of $G$.
\end{proof}
}
\end{frame}


\begin{frame}
\begin{definition}
Let $W$ be a subspace of $V$.  A \textbf{linear projection} $V$ onto $W$ is a linear map $f \colon V \to W$ such that $f \restriction_{W} = \text{Id}_W$.  If $W$ is a subrepresentation of $V$ and the map $f$ is $G$-invariant, then we say that $f$ is a $\mathbf{G}$\textbf{-linear projection}.
\end{definition}
\begin{lemma} \label{maschke-lemma}
Let $\rho \colon G \to GL(V)$ be a representation, and $W \subset V$ be a subrepresentation.  Suppose we have a $G$-linear projection 
\[ f \colon V \to W. \]
Then $\text{Ker}(f)$ is a complementary subrepresentation to $W$, i.e. $\text{Ker}(f)$ is a $G$-invariant subspace of $V$ such that
\[ V = \text{Ker}(f) \oplus W \]
\end{lemma}
\end{frame}

\subsection{Maschke's Theorem}
\begin{frame}{Maschke's Theorem}
\begin{theorem}[Maschke's Theorem]
Let $G$ be a finite group and let $F$ be a field such that $\text{char}(F) \nmid |G|$.  If $V$ is any finite dimensional representation of $G$ over $F$, and $W \subset V$ is a subrepresentation of $V$, then there exists a complementary subrepresentation $U \subset V$ to $W$, i.e. there is  a $G$-invariant subspace $U \subset V$ such that 
\[ V = W \oplus U. \]
\end{theorem}
\end{frame}

\begin{frame}{Maschke's Theorem}
\begin{proof}
\only<1>{
It will suffice to find a $G$-linear projection from $V$ onto $W$.  Fix a basis $\{ b_1, \ldots, b_m \}$ for $W$ and extend it to a basis  $\{ b_1, \ldots, b_m, b_{m+1}, \ldots, b_n \}$ for $V$.  Let $U = \langle b_{m+1}, \ldots, b_n \rangle$.  Then $U$ is certainly a complementary subspace to $W$, and we have a natural projection $f \colon W \oplus U \to W$ of $V$ onto $W$ with kernel $U$.
There is no reason to think that $f$ should be $G$-linear, but we can fix this by averaging over $G$.  Define $\widetilde{f} \colon V \to V$ by
\[\widetilde{f}(x) = \frac{1}{|G|} \sum_{g \in G} (\rho(g) \circ f \circ \rho(g^{-1}))(x). \]
We claim that $\widetilde{f}$ is a $G$-linear projection from $V$ onto $W$.   } 

\only<2>{

First we check that $\text{Im}(\tilde{f}) \subset W$.  If $x \in V$ and $g \in G$, then
\[ f (\rho (g^{-1})(x)) \in W \]
and so
\[ \rho(g) ( f ( \rho( g^{-1})(x))) \in W \]
since $W$ is $G$-invariant.  Thus \[ \widetilde{f}(x) =  \frac{1}{|G|} \sum_{g \in G} (\rho(g) \circ f \circ \rho(g^{-1}))(x) \in W. \] }

\only<3>{ Next we check that $\widetilde{f} \restriction_{W} = \text{Id}_W$. Let $y \in W$.  For any $g \in G$, we know that $\rho(g^{-1})(y)$ is also in $W$, so
$ f (\rho(g^{-1})(y)) = \rho (g^{-1})(y)$.
Then
\begin{align*}
\widetilde{f}(y) &=  \frac{1}{|G|} \sum_{g \in G} \rho(g) (f ( \rho(g^{-1})(y))) \\
&=\frac{1}{|G|} \sum_{g \in G} \rho(g) (\rho(g^{-1})(y)) \\
&=\frac{1}{|G|} \sum_{g \in G} (y)  = \frac{|G| y} {|G|}  = y
\end{align*}
so indeed $\widetilde{f}$ is a linear projection of $V$ onto $W$. }

\only<4> { Finally, we check that $\widetilde{f}$ is $G$-linear.  If $x \in V$ and $h \in G$, then
\begin{align*}
(\widetilde{f} \circ \rho(h))(x) &= \frac{1}{|G|} \sum_{g \in G} (\rho(g) \circ f \circ \rho(g^{-1}) \circ \rho(h))(x) \\
&= \frac{1}{|G|} \sum_{g \in G} (\rho(g) \circ f \circ \rho(g^{-1} h))(x) \\
&=\frac{1}{|G|} \sum_{g \in G} (\rho(hg) \circ f \circ \rho(g^{-1}))(x) \quad \text{(} g \mapsto hg \text{)} \\
&= (\rho(h) \circ \widetilde{f}) (x).
\end{align*}}
\alt<4>{\qedhere}{\phantom\qedhere}
\end{proof}
\end{frame}

\begin{frame}
\begin{block}{Corollary}
Let $G$ be a finite group and let $F$ be a field such that $\text{char}(F) \nmid |G|$. Then any finite-dimensional representation of $G$ over $F$ is completely reducible.
\end{block}
\begin{proof}
Let $V$ be a representation of $G$ over $F$ of dimension $n$.  If $V$ is irreducible, then $V$ is, in particular, completely reducible.  If not, then $V$ contains a proper subrepresentation $W \subset V$.  From Maschke's Theorem, we know there exists a subrepresentation $U \subset V$ such that 
\begin{equation}  V = W \oplus U. \end{equation}
Both $W$ and $U$ have dimension less than $n$, so by induction we know that $W$ and $U$ are completely reducible. We deduce that $V$ is completely reducible.
\end{proof}
\end{frame}

\begin{frame}
\begin{example}
Recall that for  $G= C_2$, we found a $1$-dim subrepresentation 
\[ W = \left< \begin{bmatrix}1 \\ 1 \end{bmatrix} \right> \subset V_{\text{reg}}= \mathbb{C}^2 . \]
We know a complementary subrepresentation to $W$ exists by Machke's Theorem, so let's try to find one.  Consider
\[ U =  \left< \begin{bmatrix}1 \\ -1 \end{bmatrix} \right> \subset V_{\text{reg}}. \]
Then
\[ \rho_{\text{reg}}(\tau) \begin{bmatrix}1 \\ -1 \end{bmatrix} =  \begin{bmatrix} 0 & 1 \\ 1 & 0 \end{bmatrix} \begin{bmatrix}1 \\ -1 \end{bmatrix} = -  \begin{bmatrix}1 \\ -1 \end{bmatrix} \]
so $U$ is $G$-invariant.  We see that $V = W \oplus U$, since $W \cap U = \{ 0 \}$ and $\text{dim } U + \text{dim } W = 2 = \text{dim } V$.   (Note $U$ is isomorphic to the alternating representation $\rho_{\text{sgn}}$.)
\end{example}
\end{frame}

\section{Schur's Lemma}
\subsection{Vector Spaces of Linear Maps}

\begin{frame}
\begin{proposition}
Suppose we have representations $\rho_V \colon G \to GL(V)$ and $\rho_W \colon G \to GL(W)$ of $G$. Then there is a natural representation of $G$ on the vector space $\text{Hom}(V,W)$ given for all $g \in G$ by
\begin{align*}		
 \rho_{\text{Hom}(V,W)}(g)  \colon \text{Hom}(V,W) &\to \text{Hom}(V,W) \\		
 f &\mapsto \rho_{W}(g) \circ f \circ \rho_{V}(g^{-1}).
 \end{align*}		
\end{proposition}		
\begin{proof}[Proof (sketch)]
\begin{enumerate}			
\item $\rho_{\text{Hom}(V,W)}(g)(f) \in \text{Hom}(V,W)$ since the composition of linear maps is linear.
\item For every $g \in G$,  $\rho_{\text{Hom}(V,W)}(g)$ is invertible.	
\item The map $g \mapsto  \rho_{\text{Hom}(V,W)}(g)$ is a homomorphism.	
\end{enumerate}		
\end{proof}
\end{frame}

\begin{frame}
\begin{definition}		
Let $V$ and $W$ be two representations of $G$.  The set of $G$-linear maps from $V$ to $W$ forms a subspace of $\text{Hom}(V,W)$, which we denote by $\textbf{Hom}_\mathbf{G}\mathbf{(V,W)}$.  In other words, $\text{Hom}_{G}(V,W)$ is the vector space consisting of all \textit{homomorphisms of representations} between $V$ and $W$. 
\end{definition} 

\begin{definition}
Let $\rho \colon G \to GL(V)$ be a representation.  We define the \textbf{invariant subrepresentation} $V^G \subset V$ to be the set 
\[ \{ v \in V  \mid \rho(g)(v) = v,  \quad \forall g \in G \}. \]
\end{definition}
\end{frame}
\begin{note}
Note that $V^G$ is a subspace of $V$, and is also clearly a subrepresentation.   It is isomorphic to a trivial representation of some dimension.
\end{note}

\begin{frame}
\begin{proposition}
Let  $\rho_V \colon G \to GL(V)$ and $\rho_W \colon G \to GL(W)$ be representations of $G$.  Then the subrepresentation 
\[ \text{Hom}_G (V,W) \subset \text{Hom} (V,W) \]
is precisely the invariant subrepresentation $\text{Hom}(V,W) ^G$ of $\text{Hom}(V,W)$.
\end{proposition}
\begin{proof}
Let $f \in \text{Hom}(V,W)$.  Then $f \in \text{Hom}(V,W) ^G$ iff we have
\begin{align*}
&f = \rho_{\text{Hom}(V,W)} (g)(f)  \quad \forall g \in G \\
\iff &f = \rho_W (g) \circ f \circ \rho_V (g^{-1}) \quad \forall g \in G \\ 
\iff & f \circ \rho_V (g) = \rho_W (g) \circ f \quad \forall g \in G
\end{align*}
which is exactly the condition that $f$ is $G$-linear.
\end{proof}
\end{frame}

\subsection{Schur's Lemma}
\begin{frame}
\begin{theorem}[Schur's Lemma over $\mathbb{C}$.] If $V$ is an irreducible representation of $G$ over $\mathbb{C}$, then evey linear operator $\phi \colon V \to V$ commuting with $G$ is a scalar.
\end{theorem}
\begin{proof}
Let $\phi \colon V \to V$  be a linear operator commuting with $G$, and let $\lambda$ be an eigenvalue of $\phi$.  Observe that the eigenspace $E_\lambda$ is $G$-invariant: If $v \in E_\lambda$, then $\phi(v) = \lambda v$.  This implies that $\phi(g v) = g \phi(v) = g (\lambda v) = \lambda (gv)$, i.e. $gv \in E_\lambda$. Since $g$ was arbitrary, $E_\lambda$ is indeed $G$-invariant.  Now $E_\lambda \neq 0$, so since $V$ is irreducible, $E_\lambda = V$.  Thus $\phi = \lambda \text{Id}$.  
\end{proof}
\end{frame}

\begin{frame}
\begin{block}{Corollary}
Suppose $V$ and $W$ are irreducible. The space $\text{Hom}_G(V,W)$ is $1$-dimensional if the representations are isomorphic, and in this case any non-zero map is an isomorphism. Otherwise,  $\text{Hom}_G(V,W)=\{0\}$.
\end{block}
\begin{proof}
Suppose  $\text{Hom}_G(V,W) \neq \{0\}$ and let $\phi \in \text{Hom}_G(V,W)$. We have seen $\text{ker}(\phi)$ and $\text{im}(\phi)$ are both $G$-invariant.
Irreducibility yields $\text{ker}(\phi) = 0$ or $V$ and $\text{im}(\phi) = 0$ or $W$ as the only possibilities.  Since $\phi \neq 0$, then $\text{ker}(\phi)=0$, $\text{im}(\phi)=W$, and $\phi$ is an isomorphism.  
Let $\psi$ be another nonzero interwining operator from $V$ to $W$.  Then $\phi ^{-1} \circ \psi \in \text{Hom}_G (V,V)$.  We can apply Schur's Lemma over $\mathbb{C}$ to see that $\phi ^{-1} \circ \psi = \lambda \text{Id}$, hence $\psi = \lambda \phi$.  So $\phi$ spans $\text{Hom}_G(V,W)$.
\end{proof}
\end{frame}

\section{Isotypical Decomp.}
\subsection{Isotypical decomposition}
\begin{frame}
\begin{proposition}
Let $V$ and $W$ be irreducible representations of $G$.  Then
\[ \text{dim Hom}_G (V,W) =  \begin{cases} 
1 & \mbox{if $V$ and $W$ are isomorphic}  \\
0 &\mbox{if $V$ and $W$ are not isomorphic}
\end{cases} \]
\end{proposition}
\begin{proof}
Suppose $V$ and $W$ are not isomorphic.  Then the Corollary to Schur's Lemma states that the only $G$-linear map from $V$ to $W$ is the zero map, hence $\text{Hom}_G(V,W) = \{ 0 \} $.

On the other hand, suppose that $f \colon V \to W$ is an isomorphism.  Then for any $h \in \text{Hom}_G(V,W)$, we have $f^{-1} \circ h \in \text{Hom}_G(V,V)$.  By Schur's Lemma, $f^{-1} \circ h = \lambda \text{Id}_V$ for some $\lambda \in \mathbb{C}$, i.e. $h = \lambda f$.  Thus $f$  spans $\text{Hom}_G(V,W)$.
\end{proof}
\end{frame}

\begin{frame}
\begin{proposition}
Let $\rho \colon G \to GL(V)$ be a representation, let \[ V = U_1 \oplus \ldots \oplus U_s \] be a decomposition of $V$ into irreps, and let $W$ be any irrep of $G$.  Then the number of irreps in the set  $ \{ U_1, \ldots, U_s \}$ which are isomorphic to $W$ equals the dimension of $\text{Hom}_G(V,W)$.
\end{proposition}
\begin{block}{Lemma}
If  $U, V,$ and $W$ are representations of $G$, then there are natural isomorphisms
\begin{itemize}
\item $\text{Hom}_G(V, U \oplus W) = \text{Hom}_G(V,U) \oplus \text{Hom}_G(V,W)$
\item $\text{Hom}_G(U \oplus W, V) = \text{Hom}_G(U, V) \oplus \text{Hom}_G(W ,V)$
\end{itemize}
\end{block}
\end{frame}

\begin{frame}
\begin{proof}
We use the previous proposition to see that t	he number of irreps in the set  $ \{ U_1, \ldots, U_s \}$ which are isomorphic to $W$ is equal to \[ \sum_{i=1}^s \text{dim Hom}_G(U_i,W). \]
Then \[ \text{Hom}_G(V,W) = \bigoplus_{i=1}^s \text{Hom}_G(U_i, W) .\]
by our lemma, so taking the dimension of both sides yields \[  \text{dim Hom}_G(V,W) = \sum_{i=1}^s \text{dim Hom}_G(U_i, W). \]
\end{proof}
\end{frame}
\begin{note}
The same argument works if we consider $\text{Hom}_G(W,V)$ and $\text{Hom}_G(W,U_i)$ in place of $\text{Hom}_G(V,W)$ and $\text{Hom}_G(U_i,W)$.
\end{note}

\begin{frame}
\begin{theorem}[Uniqueness of decomposition into irreducibles.]
Let $\rho \colon G \to GL(V)$ be a representation, and let
\begin{align*}
V = U_1 \oplus \ldots \oplus U_s \\
V = \widetilde{U_1} \oplus \ldots \oplus \widetilde{U_r}
\end{align*}
be two decompositions of $V$ into irreducible subrepresentations.  Then $s = r$, and (after reordering if necessary) $U_i$ and $\widetilde{U_i}$ are isomorphic for every $i \in \{1, \ldots, s\}$.
\end{theorem}
\begin{proof}
For any irrep $W$ of $G$, the number of irreps in either decomposition that are isomorphic to $W$ is equal to $\text{dim Hom}_G(V,W)$.  So the two decompositions contain the same number of factors isomorphic to $W$ for any irrep $W$ of $G$.
\end{proof}
\end{frame}

\section{Duals and Tensors}
\subsection{Dual Spaces}
\begin{frame}{The Dual Space}
\begin{definition}
Let $V$ be a vector space.  Recall that we define the \textbf{dual vector space} to be
\[ V^{*} = \text{Hom}(V,\mathbb{C}).\]  If we fix a basis  $\{ b_1, \ldots, b_n \} $ for $V$, then the \textbf{dual basis}$\{ f_1, \ldots, f_n \}$ for $V^*$ is defined by
\[ f_i (b_j) = \begin{cases} 1 &\text{if } i=j \\ 0 &\text{if } i \neq j.  \end{cases} \]
\end{definition}
\end{frame}
\begin{note}
Let $W$ be the $1$-dim trivial rep of $G$.  Then we have seen the rep  $\rho_{\text{Hom}(V,\mathbb{C})}$ given by 
\[ \rho_{\text{Hom}(V,\mathbb{C})} (g) (f) = \rho_W (g) f \circ \rho_V ( g^{-1})= f \circ \rho_V ( g^{-1}) \]
\end{note}


\begin{frame}{The Dual Representation}
\begin{definition}
Let $\rho_V \colon G \to GL(V)$ be a representation of $G$. Then we have seen that $V^{*}$ carries a representation of $G$ defined by
\[ \rho_{\text{Hom}(V,\mathbb{C})} (g) (f) = f \circ \rho_V ( g^{-1}) \]
We call this the \textbf{dual representation} to $\rho_V$, and denote it by $\rho_V^{*}$.
\end{definition}
\begin{proposition}
 If we fix a basis for $V$, then $\rho_{V^*}(g)$ is given by the matrix 
\[( \rho_V (g^{-1}) )^T \]
with respect to the dual basis.
\end{proposition}
\end{frame}

\subsection{Tensor Products}
\begin{frame}
\begin{definition}
Suppose $V$ and $W$ are two vector spaces over a field $K$. Then we define a new vector space called the \textbf{tensor product} of $V$ and $W$, denoted by $V \otimes_{K} W$.  This space is the quotient of the free vector space on $V \times W$ (with basis given by formal symbols $v \otimes w, v\in V, w \in W$), by the subspace $D$ spanned by all elements of the form
\begin{align*}
(v_1 + v_2, w) - (v_1 , w) - (v_2 , w) \\
(v , w_1 + w_2) - (v , w_1) - (v , w_2) \\
(k \cdot v , w) -( v , k \cdot w)
\end{align*}
for $v, v_1, v_2 \in V, w, w_1, w_2 \in W$, and $k \in K$. When the ground field $K$ is clear it can be omitted from the notation.  The elements of $V \otimes W$ are called \textbf{tensors}, and the coset $v \otimes w$ of $(v,w)$ in $V \otimes W$ is called a \textbf{simple tensor}.
\end{definition}
\end{frame}

\begin{frame}
\begin{definition}
We can define a representation of $G$ on $V \otimes W$ called the \textbf{tensor product representation}. We define
\[ \rho_{V \otimes W} (g) \colon V \otimes W \to V \otimes W \]
to be the linear map given by
\[ \rho_{V \otimes W} (g) \colon a_i \otimes b_j \mapsto \rho_V (g) (a_i) \otimes \rho_W (g) (b_j).\]
\end{definition}
\begin{proposition}
Let $V$ and $W$ be representations of $G$.  Then $V \otimes W$ is isomorphic to $\text{Hom}(V^{*},W)$.
\end{proposition}
\end{frame}

\begin{frame}
\begin{block}{Note}
Let $\{ a_1,  \ldots, a_n\}$ be a basis for $V$, let $\{\alpha_1, \ldots, \alpha_n \}$ be the corresponding dual basis for $V^{*}$, and let $\{b_1,  \ldots,b_m\}$ be a basis for $W$.  Then $\text{Hom}(V^*,W)$ has a basis $\{ f_{it} | 1 \leq i \leq n, 1 \leq t \leq m \}$ where
\begin{align*}
f_{it} (\alpha_j) = \begin{cases} b_t &\text{if } j = i\\ 0 &\text{if } j \neq i \end{cases} 
\end{align*}
Let $M$ and $N$ denote the matrices which describe $\rho_V (g)$ and $\rho_W (g)$ in the given bases.  If we write $\rho_W (g) \circ f_{it} \circ \rho_{V^{*}} (g^ {-1})$ in terms of the basis $\{ f_{js} \}$, we have 
\[\rho_W (g) \circ f_{it} \circ \rho_{V^{*}} (g^ {-1}) = \sum_{\substack{j \in [1,n] \\  s \in [1,m]}} M_{ji} N_{st} f_{js} \]
\end{block}
\end{frame}

\section{Character Theory}
\subsection{Basics}
\begin{frame}{The definition of a Character}
\begin{definition}
The \textbf{character} of a representation $\rho \colon G \to GL(V)$ is the function \[ \chi_V \colon G \to \mathbb{C}\] defined by \[\chi_V(g) = \text{Tr}(\rho(g)).\]
\end{definition}
\begin{block}{Note}
The character is of a representation is not a homomorphism in general, since $\text{Tr}(MN) \neq \text{Tr}(M) \text{Tr}(N)$ in general.
\end{block}
\end{frame}

\begin{frame}{Basic properties of Characters}
\begin{proposition}
Let $V$ be a representation of $G$.
\begin{itemize}
\item $\chi_V$ is conjugation invariant: $\chi_V (h g h^{-1}) = \chi_V (g) \quad \forall g , h \in G$.
\item $\chi_V (e) = \text{\emph{dim }} V$.
\item \label{char-of-inverse} $\chi_V (g^{-1}) = \overline{\chi_V (g)} \quad \forall g \in G$.
\item $\chi_{V^*} (g) =  \overline{\chi_V (g)}\quad \forall g \in G$.
\end{itemize}
\end{proposition}
\begin{proposition}
Let $V$ and $W$ be representations of $G$.
\begin{itemize}
\item $\chi _{V \oplus W} = \chi_V + \chi_W$.
\item $\chi_{V \otimes W} = \chi_V \cdot \chi_W$.
\end{itemize}
\end{proposition}
\end{frame}

\begin{frame}
\begin{proposition}
Isomorphic representations have the same character.
\end{proposition}
\begin{proof}
Isomorphic representations can be described by the same set of matrices with the right choice of bases.  Thus each $\rho(g)$ has the same trace.
\end{proof}
\end{frame}

\subsection{Inner products of characters}
\begin{frame}
\begin{definition}
Let $\mathbb{C}^G$ denote the vector space of all functions from $G$ to $\mathbb{C}$.
A basis for $\mathbb{C}^G$ is given by the set of functions 
\[\{ \delta_g | g \in G  \} \]
defined by 
\[ \delta_g \colon h \mapsto \begin{cases}  1 &\text{if } h = g \\
 0 &\text{if } h \neq g.
\end{cases} \]
\end{definition}
\begin{definition}
Let $\varphi, \psi \in \mathbb{C}^G$.  We define a \textbf{hermetian inner product}  on $\mathbb{C}^G$ by 
\[ \langle \varphi | \psi \rangle = \frac{1}{|G|} \sum_{g \in G} \varphi(g) \overline{\psi(g)}.\]
\end{definition}
\end{frame}

\begin{frame}{Inner product of Characters}
\begin{theorem}
Let $\rho_V \colon G \to GL(V)$ and $\rho_W \colon G \to GL(W)$ be representations of $G$, and let $\chi_V, \chi_W$ be their characters.  Then 
\[ \langle \chi_W | \chi_V \rangle = \text{dim Hom}_G (V,W). \]
\end{theorem}
\end{frame}

\begin{frame}
\begin{block}{Corollary}
Let $\chi_1, \ldots, \chi_r$ be characters of pairwise non-isomorphic irreducible representations of $G$.  Then
\[ \langle \chi_i | \chi_j \rangle = \begin{cases}  1 &\text{if } i = j \\ 0 &\text{if } i \neq j\end{cases} \]
\end{block}
\begin{proof}
Let $\chi_i$ and $\chi_j$ be the characters of the irreducible representations $U_i, U_j$.  Then
\[ \langle \chi_i | \chi_j \rangle = \text{dim Hom}_G (U_i, U_j) = \begin{cases}  1 &\text{if }U_i, U_j \text{ are isomorphic} \\  0 &\text{if }U_i, U_j \text{ are not isomorphic}. \end{cases} \]
\end{proof}
\end{frame}

\begin{frame}
\begin{block}{Corollary}
Let $\chi$ be any character of $G$.    Then $\chi$ is irreducible if and only if \[ \langle \chi | \chi \rangle = 1\]
\end{block}
\begin{proof}
Write $\chi$ as a linear combination of irreducible characters \[ \chi = m_1 \chi_1 + \ldots + m_k \chi_k \] where each $m_i$ is a non-negative integer.  Then
\begin{align*}
\langle \chi | \chi \rangle &= \sum_{i,j \in [1, k]} m_i m_j \langle \chi_i | \chi_j \rangle \\
&= m_1^2 + \ldots + m_k^2.
\end{align*}
So $\langle \chi | \chi \rangle = 1$ if and only if exactly one of the $m_i = 1$ and the rest are $0$.
\end{proof}
\end{frame}

\begin{frame}
\begin{example}
Let $G=D_4=  \langle \sigma, \tau |  \sigma^4 = \tau^2 = e, \tau \sigma \tau^{-1} = \sigma^{-1} \rangle$.  Recall the two dimensional representation $W$ of $D_4$ given earlier.
We compute the character of this representation by taking the trace of the matrices from that example:
\begin{align*}
\chi_W(e) &=2 & \chi_W(\tau) = 0 \\
\chi_W(\sigma) &= 0 &\chi_W (\sigma \tau) = 0 \\
\chi_W(\sigma^2) &= -2 &\chi_W (\sigma^2 \tau) = 0 \\
\chi_W(\sigma^3) &= 0 &\chi_W (\sigma^3 \tau) = 0.
\end{align*}
Then
\[ \langle \chi_W | \chi_W \rangle =\frac {1}{|G|} \sum_{g \in G} \chi_W (g) \overline{\chi_W (g)} = \frac{1}{8} (4 + 4) = 1\]
so we conclude that $W$ is irreducible.
\end{example}
\end{frame}

\begin{frame}{Proof that $\langle \chi_W | \chi_V \rangle = \text{dim Hom}_G (V,W)$}
\begin{lemma}
Let $\rho \colon G \to GL(V)$ be any representation.  Consider the linear map 
\begin{align*}
\Psi \colon V &\to V \\
x &\mapsto \frac{1}{|G|} \sum_{g \in G} \rho(g)(x).
\end{align*}
Then $\Psi$ is a  projection from $V$ onto the invariant subspace $V^G$.
\end{lemma}
\begin{lemma}
Let $V$ be a vector space with subspace $U \subset V$, and let $\pi \colon V \to V$ be a projection onto $U$.  Then 
\[ \text{Tr}(\pi) = \text{dim } U. \]
\end{lemma}
\end{frame}

\begin{frame}{Proof that $\langle \chi_W | \chi_V \rangle = \text{dim Hom}_G (V,W)$}
\begin{proof}
\only<1>{
We have seen that
 \[ \text{Hom}_G (V,W)=  \text{Hom}(V,W)^G  \subset \text{Hom}(V,W). \] By the first Lemma, we have a projection
\begin{align*}
\Psi \colon \text{Hom}(V,W) &\to \text{Hom}(V,W)^G\\
f &\mapsto \frac{1}{|G|} \sum_{g \in G} \rho_{\text{Hom}(V,W)} (g) (f).
\end{align*}
We claim that \[ \text{Tr} (\Psi) = \langle \chi_W | \chi_V \rangle. \]  Once this claim is established, then the theorem will follow from our second Lemma, since $\text{Tr} (\Psi)  = \text{dim Hom}_G (V,W)$. }
\only<2>{
We proceed by calculating $\text{Tr} (\Psi)$.  Fix bases $\{ a_1, \ldots, a_n \}$ for $V$ and $\{ b_1, \ldots, b_m \}$ for $W$.  Then $\text{Hom}(V,W)$ has an associated basis 
\[ \{ f_{ji} | 1 \leq i \leq n, 1 \leq j \leq m\} \]
where
\[ f_{ji} (a_i) = \begin{cases} b_j &\text{if } i=j \\ 0 &\text{if } i \neq j.  \end{cases} \] 
We may calculate $\text{Tr}(\Psi)$ as follows:  For each $i,j$, compute the expression of $\Psi(f_{ji})$ in this basis, and take the coefficient of the basis element $f_{ji}$.  This is a diagonal entry in the matrix for $\Psi$. Summing these values over all $i$ and $j$ will gives us $\text{Tr}(\Psi)$.
}
\only<3>{
Let $\widetilde{\rho_V}, \widetilde{\rho_W}$ be the matrix representations obtained by writing $\rho_V$ and $\rho_W$ in the given bases.  We have seen that 
\[ \text{Hom}(V,W) = V^* \otimes W \]
and if we write $\rho_{\text{Hom}(V,W)}$ in the basis $\{ f_{ji} \}$ then we get
\begin{align*}
\rho_{\text{Hom}(V,W)}(g)(f_{ji}) &= \rho_W (g) \circ f_{ji} \circ \rho_V (g^{-1}) \\
&= \sum_{\substack{k \in [1,n] \\  t \in [1,m]}} \widetilde{\rho_V} (g^{-1})_{ik} \widetilde{\rho_W}(g)_{tj} f_{kt}.
\end{align*}
}
\only<4>{
Now 
\begin{align*}
\Psi (f_{ji}) &= \frac{1}{|G|} \sum_{g \in G} \rho_{\text{Hom}(V,W)}(g)(f) \\
&= \frac{1}{|G|} \sum_{g \in G} \sum_{\substack{k \in [1,n] \\  t \in [1,m]}} \widetilde{\rho_V} (g^{-1})_{ik} \widetilde{\rho_W}(g)_{tj} f_{kt}.
\end{align*}
The coefficient of $f_{ji}$ in this expression is
\[ \frac{1}{|G|} \sum_{g \in G}\widetilde{\rho_V} (g^{-1})_{ii} \widetilde{\rho_W}(g)_{jj}. \]
(This is a diagonal entry of $\Psi$.)
}
\only<5>{
Therefore
\begin{align*}
\text{Tr} (\Psi) &= \sum_{\substack{k \in [1,n] \\  t \in [1,m]}}  \frac{1}{|G|} \sum_{g \in G}\widetilde{\rho_V} (g^{-1})_{ii} \widetilde{\rho_W}(g)_{jj} \\
&=  \frac{1}{|G|} \sum_{g \in G} \left( \sum_{i=1}^n \widetilde{\rho_V}(g^{-1})_{ii} \right) \left( \sum_{j=1}^m \widetilde{\rho_W} (g)_{jj} \right) \\
&= \frac{1}{|G|} \sum_{g \in G} \chi_V (g^{-1}) \chi_W (g) \\
&= \frac{1}{|G|} \sum_{g \in G} \chi_W (g) \overline{\chi_V}(g) = \langle \chi_W | \chi_V \rangle.
\end{align*} }
\alt<5>{\qedhere}{\phantom\qedhere}
\end{proof}
\end{frame}

\begin{frame}[plain]
\begin{block}{Corollary}
Let $V$ and $W$ be representations of $G$.  Then $V$ and $W$ are isomorphic if and only if $\chi_V = \chi_W$.
\end{block}
\begin{proof}
Suppose $\chi_V = \chi_W$. We can find non-negative integers $m_i$ and $l_j$ such that
\begin{align*}
V = U_1^{m_1} \oplus \ldots \oplus U_r^{m_r} \quad \text{ and } \quad W = U_1^{l_1} \oplus \ldots \oplus U_r^{l_r}
\end{align*}
where $U_1, \ldots, U_r$ are distinct irreps of $G$.  Then
\begin{align*}
\chi_V = m_1 \chi_1 + \ldots + m_r \chi_r \quad \text{ and } \quad  \chi_W = l_1 \chi_1 + \ldots + l_r \chi_r .
\end{align*}
It follows that
\[ m_i = \langle \chi_V | \chi_i \rangle   = \langle \chi_W | \chi_i \rangle =  l_i \]
for all $i \in \{1, \ldots, r \}$ since $\chi_V = \chi_W$.
\end{proof}
\end{frame}

\begin{frame}
\begin{block}{Lemma}
$\chi_{\text{reg}}(g) = \begin{cases} |G| &\text{if } g=e \\ 0 &\text{if } g \neq e \end{cases}$
\end{block}
\begin{proposition}
The multiplicity of any irreducible representation in the regular representation equals its dimension.
\end{proposition}
\begin{proof}
Let $V$ be an irreducible representation of $G$.  Then
\begin{align*}
\langle \chi_{\text{reg}}, \chi_V \rangle &= \frac{1}{|G|} \chi_{\text{reg}}(e) \overline{\chi_V (e)} \\
&= \frac{1}{|G|} |G| (\text{dim } V)= \text{dim } V.
\end{align*}
\end{proof}
\end{frame}

\begin{frame}
\begin{block}{Corollary}
There are finitely many irreducible representations of $G$, up to isomorphism.
\end{block}
\begin{block}{Corollary}
Let $U_1, \ldots, U_r$ be the irreducible representations of $G$ with degrees $d_1, \ldots, d_r$.  Then
\[ |G| = \sum_{i=1}^n d_i^2 \]
\end{block}
\end{frame}



\begin{frame}
\begin{definition}
A \textbf{class function} on $G$ is a function on $G$ whose values are invariant by conjugation of elements in $G$. 
\end{definition}
\begin{block}{Note}
The character $\chi_V$ of a representation $V$ of $G$ is a class function on $G$. To find the inner product of $\chi_V$ and $\chi_W$, we just need to calculate $\chi$ once on each conjugacy class, i.e. 
\begin{align*}
\langle \chi_V | \chi_W \rangle &= \frac{1}{|G|} \sum_{g \in G} \rho_V (g) \overline{\rho_W (g)} \\
&=  \frac{1}{|G|} \sum_{[g]} |[g]|   \rho_V (g) \overline{\rho_W (g)}
\end{align*}
where the latter sum ranges over the conjugacy classes $[g]$ of $G$.
\end{block}
\end{frame}
\begin{note}
 The value of a class function at an element $g \in G$ depends only on the conjugacy class of $g$, so we can view class functions as functions on the set of conjugacy classes of $G$.
 \end{note}

\begin{frame}
\begin{definition}
We define \textbf{the character table of} $G$ to be the table of complex numbers whose:
\begin{itemize}\item rows are index by the isomorphism classes of irreducible representations of $G$, 
\item columns are indexed by the conjugacy classes of $G$,
\item $i,j$ entry is given by value of the character corresponding to row $i$ evaluated at the isomorphism class corresponding to column $j$.
\end{itemize}
\end{definition}
\end{frame}

\begin{frame}{Character table of $D_3$}
\begin{example}
\only<1>{
Consider $G=D_3 = \langle \sigma, \tau | \sigma^3 = \tau^2 = e, \tau \sigma \tau^{-1} = \sigma^{-1} \rangle$.  We have seen three irreducible representations of $D_3$, namely the $1$-dimensional trivial representation, the $1$-dimensional alternating representation, and the $2$-dimensional irreducible representation $W$ constructed geometrically.  Observe that
\[ |D_3| = 6 = 1^2 + 1^2 +2^2 \]
so these are all of the irreducible representations of $D_3$ up to isomorphism.  
}
\only<2>{The conjugacy classes of $D_3$ are 
$\{e\}$, $\{ \sigma, \sigma ^2 \}$, and $\{\tau, \tau \sigma, \tau \sigma^2 \}$.  Thus, the character table of $D_3$ is given by 


\begin{tabular}{ | l | c | c | c |}\hline 
\multicolumn{4}{|c|}{Character table of $D_3$} \\ \hline

Conjugacy class representative $[g]$ & $[e]$ & $[\tau]$ & $[\sigma]$ \\ \hline
$\chi_1$  ($1$-d trivial reprn) & $1$ & 1 & 1 \\ \hline
$\chi_{\text{sgn}}$  ($1$-d sign reprn)  & 1 & -1 & 1 \\ \hline
$\chi_W$ ($2$-d reprn obtained geometrically) & 2 & 0 & -1 \\
\hline
\end{tabular}
}
\end{example}
\end{frame}

\begin{frame}{Character Table of $D_4$}
\begin{example}
\only<1>{
Let $G = D_4$.  Let $U_1, \ldots, U_r$ be the irreducible representations of $D_4$, with dimensions $d_1, \ldots, d_r$ respectively, and let $U_1$ be the $1$-dimensional trivial representation.  Then 
\[ d_2^2 + \ldots + d_r ^2 = |G| - d_1^2 = 8 - 1 = 7. \]
There are two possibilities:  

1.  $r=8$, and $d_i = 1$ for all $ 1 \leq i \leq 8$. 

2. or $r=5$, and $d_2 = d_3 = d_4 = 1$, $d_5 = 2$.

We saw earlier that $G$ has a two-dimensional irreducible representation, so in fact (2) holds. }
\only<2>{
 The remaining $1$-dimensional representations are easy to find, since they must satisfy the relations  $\rho(\sigma)^2 = 1$ and $ \rho(\tau)^2  = 1$.
Thus the character table for $D_4$ is as follows:

\begin{tabular}{ | l | c | c | c |c | c |}\hline 
\multicolumn{6}{|c|}{Character table of $D_4$} \\ \hline
Conjugacy class  & $\{1\}$ & $ \{\sigma, \sigma^3\}$ & $\{\sigma^2\}$  & $\{\tau, \sigma^2 \tau\}$ & $\{\sigma\tau, \sigma^3 \tau\}$ \\ \hline
$\chi_1$ & $1$ & 1 & 1 & 1 & 1\\ \hline
$\chi_2$ & 1 & 1 & 1 & -1 & -1\\ \hline
$\chi_3$  & 1 & -1 & 1  & 1 & -1\\ \hline
$\chi_4$   & 1 & -1 & 1 & -1 & 1 \\ \hline
$\chi_W$ ($2$-d reprn) & 2 & 0 & -2  & 0 & 0 \\
\hline
\end{tabular}
}
\end{example}
\end{frame}


\end{document}